


\def\codeTabSpace{\hspace*{6mm}}
\newenvironment{code}%
{\begin{tabbing}%
		\codeTabSpace \= \hspace*{70mm} \= \hspace*{42mm} \= \kill%
	}%
	{\end{tabbing}%
}
\newcounter{ind}
\newcommand{\n}{\addtocounter{ind}{7}\hspace*{7mm}}
\newcommand{\p}{\addtocounter{ind}{-7}\hspace*{-7mm}}
\newcommand{\nlc}{\\\stepcounter{linenum}{\scriptsize \arabic{linenum}}\>\hspace*{\value{ind}mm}}
\newcommand{\ul}{\\\>\hspace*{\value{ind}mm}}
\newcommand{\bl}{\\[-1.5mm]\>\hspace*{\value{ind}mm}}
\newcommand{\firstline}{\stepcounter{linenum}{\scriptsize \arabic{linenum}}\>}
\newcommand{\lref}[1]{\linenumref{#1}} % use this to refer to a line number
% End of stuff for entering source code=====================================

\newcommand{\postnotnull}{(1)}
\newcommand{\postl}{(2)}
\newcommand{\postpup}{(3)}
\newcommand{\postnonempty}{(4)}
\newcommand{\postgpnotnull}{(4a)}
\newcommand{\postp}{(4b)}
\newcommand{\postgpup}{(4c)}

\newcommand{\tabtabcom}{\>\>\com}
\newcommand{\tabcom}{\>\com}



\newcommand{\Key}{Key}
\newcommand{\R}{{\bf R}}
\newcommand{\NULL}{\mbox{$\bot$}}
\newcommand{\clean}{\mbox{\sc Clean}}
\newcommand{\mk}{\mbox{\sc Mark}}
\newcommand{\insertflag}{\mbox{I\Flag}}
\newcommand{\deleteflag}{\mbox{D\Flag}}
\newcommand{\record}{\mbox{record}}
\newcommand{\info}{\mbox{info}}

\newcommand{\Flag}{Flag}
\newcommand{\DFlag}{DInfo}
\newcommand{\IFlag}{IInfo}

\newcommand{\ie}{{\it i.e.}}
\newcommand{\TRUE}{\mbox{\sc True}}
\newcommand{\FALSE}{\mbox{\sc False}}

\newcommand{\func}[1]{\mbox{\sc #1}}

\newcommand{\CASB}{\func{CAS}}

\newcommand{\comnospace}{\mbox{$\triangleright$}}
\newcommand{\com}{\mbox{\comnospace\ }}

\newcommand{\doublespace}{\addtolength{\baselineskip}{.8\baselineskip}}
\newcommand{\ds}{\addtolength{\baselineskip}{.9\baselineskip}}

%%%%%%%%%%%%%%%%%%%%% COMMENTS %%%%%%%%%%%%%%%%%%%%%%%%%%%%%%%%%%%%%%%%%
%
%%%% New state-of-the-art method for comment insertion
%%%% by multiple distributed authors.
%%%% These comments are present in the LATEX file and
%%%% do not appear in print.
%
%% define debug: use first def for debug, second def for final version
\newcommand{\debug}[1]{#1}
%\newcommand{\debug}[1]{}
%

%% \newcommand{\comment}[1]{\debug{\marginpar {\sloppy\tiny #1}}}
\newcommand{\lcomment}[1]{\debug{\comment{$\rightarrow$ #1}}}
\newcommand{\incomment}[1]{\debug{[[[#1]]]}}
%
\newcommand{\fM}[1]{\comment{#1 5-14 M.}}
\newcommand{\fD}[1]{\comment{#1 5-21 D.}}
\newcommand{\upcom}[1]{#1}
%
%%%%%%%%%%%%%%%% end of comments %%%%%%%%%%%%%%%%%%%%%%%%%%%%%%%%%



\newenvironment{remark}{\begin{trivlist}
		\item[\hspace{\labelsep}{\bf\noindent Remark. }]}{\end{trivlist}}
%
%

\newenvironment{centre}{\begin{center}}{\end{center}}
% sets up centre as alternate name for center.



% marginal comment
\newcommand{\comment}[1]{\marginpar{\tiny #1}}
%\mnote{Example: only one algorithm is typed in so far.}
\newcommand{\mnote}[1]
{\marginpar%
	[{\tiny\begin{minipage}[t]{\marginparwidth}\raggedright#1%
	\end{minipage}}]%
	{\tiny\begin{minipage}[t]{\marginparwidth}\raggedright#1%
	\end{minipage}}%
}



%\newcommand{\here}[1]{[[[#1]]]\marginpar{***}}
\newcommand{\ignore}[1]{}
\newcommand{\floor}[1]{\left\lfloor #1 \right\rfloor}

\newcommand{\lft}{\mbox{\it left}}




\section{Robust BST}

The original BST algorithm does not support the crash-recovery model. It is clear from the code a process does not persist the operation's response in the non-volatile memory, and thus, once a process crash the response is lost. For example, assume a process $q$ apply $\func{Insert} (k)$ and assume $q$ performs a successful \CASB\ in line \ref{iflag-cas} and fails after completing the \func{HelpInsert} routine. In this case, the \func{Insert} operation took effect, that is, the new key appears as a leaf of the tree, and any $\func{Find} (k)$ operation will return it. However, even though the operation was already linearized at the time of the crash, upon recovery process $q$ is unaware of it. Moreover, looking for the new leaf in the tree is not healpfull, since it might be $k$ has been removed from the tree after the crush.

Moreover, if no recover routine is supplied, it may result an execution which is not well-formed. Consider for example the following scenario. A process $q$ invoke an $Op_1 = \func{Insert}(k_1)$ operation. After a successful \CASB\ in line \ref{iflag-cas} the process crush. After recovering, $q$ invoke an $Op_2 = \func{Insert}(k_2)$ operation. Assume $k_1$ and $k_2$ belongs to a different parts of the tree (do not share parent or grandparent). Then, $q$ can complete inserting $k_2$ without having any affect on $k_1$. Now, a process $q'$ performs $\func{Find}(k_1)$ which returns \NULL, as the insertion of $k_1$ is not completed, follows by an $\func{Find}(k_2)$, which returns the leaf of $k_2$. The $\func{Insert}(k_1)$ operation will be completed later by any \func{Insert} or \func{Delete} which needs to make changes to the flagged node. We get that $Op_2$ must be before $Op_1$ in the linearization, although $Op_1$ invoked first.

The kind of anomaly described above can be addressed by having the first \CASB\ of a successful attempt for \func{Insert} or \func{Delete} as the linearization point, as in the Linked-List. For that, the \func{Find} routine should take into consideration future unavoidable changes, for example, a node flagged with \insertflag\ ensures an insertion of some key. A simple solution is to change the \func{Find} routine, such that it also helps other operations, as described in figure \ref{robust find - solution 1}. The \func{Find} routine will search for key $k$ in the tree. If the \func{Search} routine returns grandparent or parent that flagged, then it might be that an insert or delete of $k$ is currently in a progress, thus we first help these operation to complete, and then search for $k$ again. Otherwise, if $gpupdate$ or $pupdate$ has been changed since the last read, it means some change already took affect, and there is a need to search for $k$ again. If none of the above holds, there is a point in time where $gp$ points to $p$ which points to $l$, and there is no attempt to change this part of the tree. As a result, if $k$ is in the tree at this point, it must be in $l$, and the find can return safely. 

The approach described above is not efficient in terms of time. We would like a solution which maintain the desirable behaviour of the original \func{Find} routine, where a single \func{Search} is needed. A more refined solution is given in figure \ref{robust find - solution 2}. The intuition for it is drown from the Linked-List algorithm.
In the Linked-List algorithm it was enough to consider a marked node as if it has been already deleted, without the need to complete the deletion. Nonetheless, the complex BST implementation is more challenging, as the \func{Delete} routine needs to successfully capture two nodes using \CASB\ in order to complete the deletion. Therefore, if a process $p$ executes $\func{Find}(k)$ procedure, and observes a node flagged with $\deleteflag$ attempting to delete the key $k$, it can not know whether in the future this delete attempt will succeed or fail, and thus does not know whether to consider the key $k$ as part of the tree or not. To overcome this problem, in such case the process will first try and validate the delete operation by marking the relevant node. According to whether the marking attempt was successful, the process can conclude if the delete operation is successful or not.
In order to easily implement the modified \func{Find} routine there is a need to conclude from \IFlag\ what is the new leaf (leaf $new$ in the \func{Insert} routine). For simplicity of presentation, we do not add this field, and abstractly refer to it in the code.

The correctness of the two solutions relies on the following argument.
Once a process flags a node during operation $Op$ with input key $k$ (either \func{Insert} or \func{Delete}), then if this attempt to complete the operation eventually succeed (i.e., the marking is also successful in the case of \func{Delete}), then any \func{Find}(k) operation invokes from this point consider $Op$ as if it is completed.

The suggested modification, although being simple and local, only guarantee the implementation satisfy R-linearzability. However, the problem of response being lost in case of a crush is not addressed. In general, the critical points in the code for recovery are the \CASB\ primitives, as a crush right after applying \CASB\ operation results the lost of the response, and in order to complete the operation the process needs to know the result of the \CASB. In addition, because of the helping mechanism, a suspended \func{Delete} operation which flagged a node, and yet to mark one, may be completed by other process in the future, and may not. Upon recovery, the process needs to distinguish between the two cases, in order to obtain the right response.

To address this issue, the expend the helping mechanism, so that the helping process needs also to update the info structure in case of a success. This is done by adding a boolean field to the Flag structure. This way, if a process crush along an operation $Op$, upon recovery it can check whether the operation was completed by some different process. 

Before a process attempt to perform an operation, as it creates the \Flag structure $op$ describing the operation and its affect on the data structure, the process stores $op$ in a designated location (according to its id). Upon recovery, the process reads this location, and if the operation is not complete, then it retries to perform it, starting from the point of the first flagging (the first \CASB). Otherwise, the operation was completed, and the response value is already known. Notice that there is a scenario in which process $q$ recovers and observes an operation $Op$ as not being complete, but just before it retries it, some other process complete the operation. We need to prove that even in such case, the operation will affect the data structure exactly once.

The supplied implementation does not specify what happens if a process crush outside of a BST routine. 

%\setcounter{linenum}{0}


\begin{figure*}
	\footnotesize
	
	\begin{code}
		\func{Find}($\Key\ k$) : Leaf* \{ \nlc
		\n Internal *$gp$, *$p$\nlc
		Leaf *$l$\nlc
		Update $pupdate, gpupdate$\bl
		\nlc
		
		%     Update $pupdate,gpupdate$\ul \nlc
		while (\TRUE) \{ \nlc \n
		$\langle gp, p, l, pupdate, gpupdate \rangle := \func{Search}(k)$\nlc
		if $gpupdate.state \neq \clean$ then $\func{Help}(gpupdate)$ \nlc
		else if $pupdate.state \neq \clean$ then $\func{Help}(pupdate)$ \nlc
		else if $gp \leftarrow update = gpupdate$ and $p \leftarrow update = pupdate$ then \{ \nlc \n
		if $l \rightarrow key = k$ then return $l$ \nlc
		else return \NULL \nlc
		\p \} \nlc
		\p \} \nlc
		\p \}
	\end{code}
	
	\caption{Solution 1: R-linearizability \func{Find} routine}
	\label{robust find - solution 1}
\end{figure*}



\begin{figure*}
	\footnotesize
	
	\begin{code}
		\func{Find}($\Key\ k$) : Leaf* \{ \nlc
			\n Internal *$gp$, *$p$\nlc
			Leaf *$l$\nlc
			Update $pupdate, gpupdate$\bl
			\nlc
			
			%     Update $pupdate,gpupdate$\ul \nlc
			$\langle gp, p, l, pupdate, gpupdate \rangle := \func{Search}(k)$\nlc
			if $l\rightarrow key \neq k$ then \{ \nlc
				\n if ($pupdate.state = \insertflag$ and $pupdate.info$ attempt to add key $k$) then \nlc
					\n return leaf with key $k$ from $pupdate.info$ \nlc \p
				else return \NULL \nlc \p
			\} \nlc
			if ($pupdate.state = \mk$ and $pupdate.info \leftarrow l \leftarrow key = k$) then return \NULL \nlc
			if ($gpupdate.state = \deleteflag$ and $gpupdate.info \leftarrow l \leftarrow key = k$) then \{ \nlc
				\n $op := gpupdate.info$ \nlc
				$result := \CASB(op\rightarrow p\rightarrow update, op\rightarrow pupdate, \langle \mk, op\rangle)$ \tabtabcom {\bf mark \CASB}\label{mark-cas}\nlc     
				if ($result = op\rightarrow pupdate$ or $result = \langle \mk, op\rangle$) then return \NULL \label{checkmark} \tabtabcom $op\rightarrow p$ is successfully marked \nlc \p
			\} \nlc
			return $l$ \nlc \p
		\}
	\end{code}

	\caption{Solution 2: R-linearizability \func{Find} routine}
	\label{robust find - solution 2}
\end{figure*}




\begin{figure}[bt]
	\footnotesize
	\begin{code}
		\firstline
		type Update \{ \hspace*{14mm} \com stored in one CAS word\nlc
		\n  $\{\clean,\deleteflag,\insertflag,\mk\}$ $state$ \nlc
		\Flag\ *{\it info}\nlc
		\p
		\}
		\nlc
		type Internal \{ \hspace*{13.25mm} \com subtype of Node\nlc
		\n $\Key \cup \{\infty_1,\infty_2\}$ $key$\nlc
		Update $update$\nlc
		Node *\lft, *$right$\nlc
		\p\}
		\nlc
		type Leaf \{ \hspace*{18.3mm} \com subtype of Node\nlc
		%\hspace*{4mm} \com could also store auxiliary information\nlc 
		\n $\Key \cup \{\infty_1,\infty_2\}$  $key$\nlc
		\p\}
		\nlc
		type \IFlag\ \{ \hspace*{17.7mm} \com subtype of \Flag\nlc
		\n Internal *$p$, *$newInternal$\nlc
		Leaf *$l$\nlc
		\textcolor{blue}{boolean $complete$} \nlc
		\p\}
		\nlc
		type \DFlag\ \{ \hspace*{16.4mm} \com subtype of \Flag\nlc
		\n Internal *$gp$, *$p$\nlc
		Leaf *$l$\nlc
		Update $pupdate$\nlc
		\textcolor{blue}{boolean $complete$} \nlc
		\p\}   
		\ul
		\com Initialization:\nlc
		shared Internal *$Root$ := pointer to new Internal node \ul
		\n with $key$ field $\infty_2$, $update$ field $\langle \clean,\NULL\rangle$, and\ul
		pointers to new Leaf nodes with keys $\infty_1$ and\ul 
		$\infty_2$, respectively, as \lft\ and $right$ fields.
		\p 
	\end{code}
	\caption{\label{code1}Type definitions and initialization.}
\end{figure}



\begin{figure*}
	\footnotesize
	
	\begin{code}
		\func{Recover}() \{ \nlc \n
			Flag $*op$ = Announce[id] \bl \nlc
			if $op$ of type \IFlag\ then \nlc \n
				if $op \leftarrow complete = \TRUE$ then return \TRUE \nlc
				else go to line \ref{iflag-cas} \nlc \p
			if $op$ of type \DFlag\ then \nlc \n
				if $op \leftarrow complete = \TRUE$ then return \TRUE \nlc
				else go to line \ref{dflag-cas} \nlc \p
		\p \}
	\end{code}

\caption{\func{Recover} routine}
\end{figure*}



\begin{figure*}
	\scriptsize
	\begin{code}
		\firstline
		\func{Search}($\Key\ k$) : $\langle \mbox{Internal*}, \mbox{Internal*}, \mbox{Leaf*}, \mbox{Update}, \mbox{Update}\rangle$  \{\ul
		\n \com Used by \func{Insert}, \func{Delete} and \func{Find} to traverse a branch of the BST; satisfies following {\it postconditions}:\ul
		\com \postnotnull\ $l$ points to a Leaf node and $p$ points to an Internal node\ul
		\com \postl\ Either $p\rightarrow \lft$ has contained $l$ (if $k<p\rightarrow key$) or $p\rightarrow right$ has contained $l$ (if $k\geq p\rightarrow key$)\ul
		\com \postpup\ $p\rightarrow update$ has contained $pupdate$\ul
		\com \postnonempty\ if $l\rightarrow key\neq \infty_1$, then the following three statements hold:\ul
		\com \hspace*{3mm}\postgpnotnull\ $gp$ points to an Internal node\ul
		\com \hspace*{3mm}\postp\ either $gp\rightarrow \lft$ has contained $p$ (if $k<gp\rightarrow key$) or $gp\rightarrow right$ has contained $p$ (if $k\geq gp\rightarrow key$)\ul
		\com \hspace*{3mm}\postgpup\ $gp\rightarrow update$ has contained $gpupdate$\nlc
		Internal *$gp$, *$p$\nlc
		Node *$l:= Root$  \label{restart-search}\nlc
		
		Update $gpupdate, pupdate$ \tabtabcom Each stores a copy of an $update$ field\bl\nlc
		%\com The initial values of $p,gp,pupdate$ and $gpupdate$ are unimportant
		%$p :=\NULL$ 
		%$pupdate := \langle \clean, \NULL\rangle$\nlc
		while $l$ points to an internal node \{ \nlc%\com Advance down the tree\nlc
		\n         $gp := p$ \tabtabcom Remember parent of $p$\nlc
		$p := l$ \tabtabcom Remember parent of $l$\nlc
		$gpupdate := pupdate$ \tabtabcom Remember $update$ field of $gp$\nlc
		$pupdate := p\rightarrow update$\label{store-pupdate}\tabtabcom Remember $update$ field of $p$\nlc  
		%           if $pupdate.state = \mk$ then \label{checkmark} \{\nlc
		%\n              \func{HelpMarked}$(pupdate.info)$ \label{call-hm3}\nlc
		%                goto line \lref{restart-search} \com Restart traversal\nlc
		%\p         \}\nlc        
		if $k < l\rightarrow key$ then $l:= p\rightarrow \lft$ else $l:=p \rightarrow right$ \label{read-child}\tabtabcom Move down to appropriate child\nlc
		%           if $pupdate \neq p\rightarrow update$ then goto line \lref{restart-search} \com Restart traversal \label{reread-pupdate}\nlc
		\p \} \nlc
		return $\langle gp, p, l, pupdate, gpupdate \rangle$ \nlc
		\p 
		\}\bl
		\nlc
		\func{Find}($\Key\ k$) : Leaf* \{ \nlc
		\n   Leaf *$l$\bl
		\nlc
		%     Update $pupdate,gpupdate$\ul \nlc
		$\langle -,-,l,-,-\rangle := \func{Search}(k)$\nlc
		if $l\rightarrow key = k$ then return $l$\nlc
		else return \NULL\nlc
		\p
		\}\bl
		\nlc
		\func{Insert}($\Key\ k$) : boolean \{ \nlc
		\n Internal *$p$, *$newInternal$\nlc 
		Leaf *$l$, *$newSibling$\nlc 
		Leaf *$new :=$ pointer to a new Leaf node whose $key$ field is $k$  \nlc
		Update $pupdate, result$\nlc
		\IFlag\ *$op$\bl\nlc%:=$ pointer to a new \IFlag\ \record\ \ul\nlc
		while \TRUE\ \{  \nlc
		\n $\langle -, p, l, pupdate, - \rangle := \func{Search}(k)$ \label{ins-search}\nlc
		if $l \rightarrow key = k$ then return \FALSE\ \tabtabcom Cannot insert duplicate key\label{insert-false}\nlc
		if $pupdate.state \neq \clean$ then \func{Help}$(pupdate)$ \tabtabcom Help the other operation \label{ins-help-unclean}\nlc
		else \{\nlc
		\n        $newSibling :=$ pointer to a new Leaf whose key is $l\rightarrow key$\nlc
		$newInternal :=$ pointer to a new Internal node with $key$ field $\max(k, l \rightarrow key)$,\label{create-internal}\ul      
		\n      $update$ field $\langle \clean, \NULL\rangle$, and with two
		child fields equal to $new$ and $newSibling$\ul 
		(the one with the smaller key is the left child)\nlc
		\p        $op :=$ pointer to a new \IFlag\ \record\  containing $\langle p, l, newInternal, \textcolor{blue}{\FALSE}\rangle$\label{new-IFlag}\nlc
		\textcolor{blue}{$Announce[id] := op$} \nlc
		$result := \CASB(p\rightarrow update, pupdate, \langle \insertflag, op\rangle)$ \tabtabcom {\bf iflag \CASB}\label{iflag-cas} \nlc
		if $result = pupdate$ \textcolor{blue}{or $result = \langle \insertflag,op \rangle$} then \{ \tabtabcom The iflag \CASB\ was successful\nlc
		\n            \func{HelpInsert}$(op)$ \tabtabcom Finish the insertion\label{finish-insert}\nlc
		return \TRUE\ \label{insert-true}\nlc
		\p        \}\nlc 
		else \func{Help}$(result)$ \tabcom The iflag \CASB\ failed; help the operation that caused failure\label{ins-help-after-failure}\nlc
		\p    \}\nlc
		\textcolor{blue}{if $op \rightarrow complete = \TRUE$ then} \nlc
		\n		\textcolor{blue}{return \TRUE} \nlc \p
		\p\}\nlc 
		\p
		\}\bl
		\nlc
		
		\func{HelpInsert}(\IFlag\ *$op$) \{\ul
		\n     \com {\it Precondition}:  $op$ points to an \IFlag\ \record\  (\ie, it is not $\NULL$)\nlc
		\func{CAS-Child}$(op\rightarrow p, op\rightarrow l, op\rightarrow newInternal)$ \tabtabcom {\bf ichild \CASB}\label{ichild-cas}\nlc
		$\CASB(op\rightarrow p\rightarrow update, \langle \insertflag, op \rangle, \langle \clean, op\rangle)$ \tabtabcom {\bf iunflag \CASB} \label{iunflag-cas}\nlc
		\textcolor{blue}{$op \rightarrow complete := \TRUE$} \tabtabcom {mark the operation as completed}\nlc 
		\p
		\}
	\end{code}
	\caption{\label{code2}Pseudocode for \func{Search}, \func{Find} and \func{Insert}.}
\end{figure*}

\begin{figure*}
	\scriptsize
	\begin{code}
		\firstline
		\func{Delete}($\Key\ k$) : boolean \{\nlc
		\n Internal *$gp$, *$p$\nlc
		Leaf *$l$\nlc
		Update $pupdate, gpupdate, result$\nlc
		\DFlag\ *$op$\bl\nlc
		while \TRUE\ \{ \nlc
		\n     $\langle gp, p, l, pupdate, gpupdate \rangle := \func{Search}(k)$\label{del-search}\nlc
		if $l\rightarrow key \neq k$ then return \FALSE\ \tabtabcom Key $k$ is not in the tree\label{delete-false}\nlc
		%       assert: $gp$ points to an internal node\label{assert-gpnotnull}\nlc
		if $gpupdate.state \neq \clean$ then \func{Help}$(gpupdate)$ \label{del-help-unclean-1}\nlc
		else if $pupdate.state \neq \clean$ then \func{Help}$(pupdate)$\label{del-help-unclean-2}\nlc
		else \{ \tabtabcom Try to flag $gp$\nlc
		\n          $op :=$ pointer to a new \DFlag\ \record\  containing $\langle gp, p, l, pupdate, \textcolor{blue}{\FALSE} \rangle$\label{new-DFlag}\nlc
		\textcolor{blue}{$Announce[id] := op$} \nlc
		$result := \CASB(gp\rightarrow update, gpupdate, \langle \deleteflag, op \rangle)$ \tabtabcom {\bf dflag \CASB}\label{dflag-cas}\nlc
		if $result = gpupdate$ \textcolor{blue}{ or $result = \langle \deleteflag, op \rangle$} then \{ \tabtabcom \CASB\ successful \nlc
		\n             if \func{HelpDelete}$(op)$ then return \TRUE \label{delete-true} \tabtabcom Either finish deletion or unflag\nlc
		\p          \}\nlc                 
		else \func{Help}$(result)$ \tabcom The dflag \CASB\ failed; help the operation that caused the failure \label{del-help-after-failure}\nlc%; help operation that is in the way 
		\p     \}\nlc
		\textcolor{blue}{if $op \rightarrow complete = \TRUE$ then} \nlc
		\n		\textcolor{blue}{return \TRUE} \nlc \p
		\p \}\nlc
		\p
		\}\bl
		\nlc
		
		\func{HelpDelete}(\DFlag\ *$op$) : boolean \{\ul
		\n   \com {\it Precondition}:  $op$ points to a \DFlag\ \record\  (\ie, it is not \NULL)\nlc%$op\rightarrow pupdate.state = \clean$\nlc
		Update $result$ \tabtabcom Stores result of mark \CASB\bl\nlc
		%\com First, try to mark the node that $op\rightarrow p$ points to\nlc
		%$pupdate = op\rightarrow pupdate$\nlc
		$result := \CASB(op\rightarrow p\rightarrow update, op\rightarrow pupdate, \langle \mk, op\rangle)$ \tabtabcom {\bf mark \CASB}\label{mark-cas}\nlc     
		if $result = op\rightarrow pupdate$ or $result = \langle \mk, op\rangle$ then \label{checkmark}\{\tabtabcom $op\rightarrow p$ is successfully marked\nlc
		\n          \func{HelpMarked}$(op)$ \label{call-hm1} \tabtabcom Complete the deletion\nlc
		return \TRUE\tabtabcom Tell \func{Delete} routine it is done\nlc
		\p       \}\nlc
		else \{\tabtabcom The mark \CASB\ failed \nlc
		\n              
		\func{Help}$(result)$\label{help-after-failed-mark}\tabtabcom Help operation that caused failure\nlc
		\CASB$(op\rightarrow gp\rightarrow update, \langle \deleteflag, op\rangle, \langle \clean, op\rangle)$ \tabtabcom {\bf backtrack \CASB}\label{backtrack-cas}\nlc
		
		return \FALSE\tabtabcom Tell \func{Delete} routine to try again\nlc
		\p       \}\nlc
		\p
		\}\bl
		\nlc
		
		\func{HelpMarked}(\DFlag\ *$op$) \{\ul
		\n   \com {\it Precondition}:  $op$ points to a \DFlag\ \record\  (\ie, it is not $\NULL$)\nlc
		Node *$other$\bl\ul
		\com Set $other$ to point to the sibling of the node to which  $op\rightarrow l$ points \nlc
		if $op\rightarrow p\rightarrow right = op\rightarrow l$ then $other := op\rightarrow p \rightarrow \lft$ else $other:=op\rightarrow p\rightarrow right$\label{read-other}\ul 
		\com Splice the node to which $op\rightarrow p$ points out of the tree, replacing it by $other$\nlc
		\func{CAS-Child}$(op\rightarrow gp, op\rightarrow p, other)$ \tabtabcom {\bf dchild \CASB}\label{dchild-cas}\nlc
		%     \com Unflag the node $op\rightarrow gp$ points to\nlc
		\CASB$(op\rightarrow gp\rightarrow update, \langle \deleteflag, op\rangle, \langle \clean, op\rangle)$ \tabtabcom {\bf dunflag \CASB}\label{dunflag-cas}\nlc
		\textcolor{blue}{$op \rightarrow complete := \TRUE$} \tabtabcom {mark the operation as completed} \nlc
		\p
		\}\bl\nlc
		
		\func{Help}(Update $u$) \{ \tabtabcom General-purpose helping routine\ul
		\n    \com {\it Precondition}:  $u$ has been stored in the $update$ field of some internal node\nlc
		if $u.state = \insertflag$ then \func{HelpInsert}$(u.\info)$\label{call-HelpInsert}\nlc
		else if $u.state = \mk$ then \func{HelpMarked}$(u.\info)$\label{call-hm2}\nlc
		else if $u.state = \deleteflag$ then \func{HelpDelete}$(u.\info)$\label{call-HelpDelete}\nlc
		\p
		\}\bl
		\nlc
		\func{CAS-Child}(Internal *$parent$, Node *$old$, Node *$new$) \{\label{CAS-Child}\ul
		\n  \com {\it Precondition}:  $parent$ points to an Internal node and $new$ points to a Node (\ie, neither is \NULL)\ul
		\com This routine tries to change one of the child fields of the node that $parent$ points to from $old$ to $new$.\nlc
		if $new \rightarrow key < parent\rightarrow key$ then\label{which-child}\nlc
		\n       \CASB$(parent\rightarrow \lft, old, new)$\label{child-cas-1}\nlc
		\p  else\nlc
		\n       \CASB$(parent\rightarrow right, old, new)$\label{child-cas-2}\nlc
		\p\p 
		\}
	\end{code}
	\caption{\label{code3}Pseudocode for \func{Delete} and some auxiliary routines.}
\end{figure*}