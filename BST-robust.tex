

The original BST algorithm does not support the crash-recovery model. It is clear from the code a process does not persist an operation response in the non-volatile memory, and thus, once a process crash the response is lost. For example, assume a process $p$ apply $\func{Insert} (k)$ and assume $p$ performs a successful \CASB\ in line \ref{iflag-cas} and fails after completing the \func{HelpInsert} routine. In this case, the \func{Insert} operation took effect, that is, the new key appears in a leaf of the tree, and a $\func{Find} (k)$ operation will return it. However, even though the operation was already linearized at the time of the crash, upon recovery process $p$ is unaware for it. Moreover, looking for $k$ in the tree, or even looking for the new leaf in the tree, is not healpfull, since it might be $k$ has been removed from the tree followed the crush.

Moreover, if no recover is supplied, it may result an execution which is not well-formed. Consider for example the following scenario. A process $p$ invoke an $Op_1 = \func{Insert}(k_1)$ operation. After a successful \CASB\ at line \ref{iflag-cas} the process crush. After recovering, $p$ invoke an $Op_2 = \func{Insert}(k_2)$ operation. Assume $k_1$ and $k_2$ belongs to a completely different parts of the tree. Then, $p$ can complete inserting $k_2$ without having any affect on $k_1$. Now, a process $q$ performs $\func{Find}(k_1)$ which returns \NULL, as the insertion of $k_1$ is not completed, follows by an $\func{Find}(k_2)$, which returns the leaf of $k_2$. The $\func{Insert}(k_1)$ operation will be completed later by any process accessing the flagged node. We get that $Op_2$ must be before $Op_1$ in the linearization, although $Op_1$ invoked first.

This kind of anomaly can be solved by having the \func{Find} routine helping in case of a flagged or marked node. In such case, once a process flagged a node, then any operation, including \func{Find}, will try and complete this operation, and thus either the operation is linearized at the time of the flagging, or there is no linearization at all. A more fine modification will be as follows. The routine \func{Find} calls \func{Search} and gets $\langle gp, p, l, pupdate, gpupdate \rangle$ in return. Then the process look to see if there is an unavoidable operation in progress that may affects its response. There are several cases to consider:
\begin{itemize}
	\item 
\end{itemize}


However, such a solution is not efficient for the \func{Find} routine, as it needs to complete other operations.

The helping mechanism, in which a process performing \func{Insert} or \func{Delete} first helps operation which marked of flagged the relevant nodes, guarantees the BST implementation is consistent even in the presence of crush-recovery. Any operation has a single linearization point, which is the point where the successful \CASB\ changes the tree data structure, that is, the \CASB\ at the \func{CAS-Child} routine. Therefore, even in the case of a crush, 



If a process $p$ crush in a middle of executing an operation $Op$, then either $Op$ did not flagged or marked any node, thus it did not took affect on the data-structure, nor it will in the future. However, if $Op$ already flagged or marked a node, then the helping mechanism guarantees that







\setcounter{linenum}{0}

\begin{figure}[bt]
	\footnotesize
	\begin{code}
		\firstline
		type Update \{ \hspace*{14mm} \com stored in one CAS word\nlc
		\n  $\{\clean,\deleteflag,\insertflag,\mk\}$ $state$ \nlc
		\Flag\ *{\it info}\nlc
		\p
		\}
		\nlc
		type Internal \{ \hspace*{13.25mm} \com subtype of Node\nlc
		\n $\Key \cup \{\infty_1,\infty_2\}$ $key$\nlc
		Update $update$\nlc
		Node *\lft, *$right$\nlc
		\p\}
		\nlc
		type Leaf \{ \hspace*{18.3mm} \com subtype of Node\nlc
		%\hspace*{4mm} \com could also store auxiliary information\nlc 
		\n $\Key \cup \{\infty_1,\infty_2\}$  $key$\nlc
		\p\}
		\nlc
		type \IFlag\ \{ \hspace*{17.7mm} \com subtype of \Flag\nlc
		\n Internal *$p$, *$newInternal$\nlc
		Leaf *$l$\nlc
		\textcolor{blue}{boolean $complete$} \nlc
		\p\}
		\nlc
		type \DFlag\ \{ \hspace*{16.4mm} \com subtype of \Flag\nlc
		\n Internal *$gp$, *$p$\nlc
		Leaf *$l$\nlc
		Update $pupdate$\nlc
		\textcolor{blue}{boolean $complete$} \nlc
		\p\}   
		\ul
		\com Initialization:\nlc
		shared Internal *$Root$ := pointer to new Internal node \ul
		\n with $key$ field $\infty_2$, $update$ field $\langle \clean,\NULL\rangle$, and\ul
		pointers to new Leaf nodes with keys $\infty_1$ and\ul 
		$\infty_2$, respectively, as \lft\ and $right$ fields.
		\p 
	\end{code}
	\caption{\label{code1}Type definitions and initialization.}
\end{figure}


\begin{figure*}
	\scriptsize
	\begin{code}
		\firstline
		\func{Search}($\Key\ k$) : $\langle \mbox{Internal*}, \mbox{Internal*}, \mbox{Leaf*}, \mbox{Update}, \mbox{Update}\rangle$  \{\ul
		\n \com Used by \func{Insert}, \func{Delete} and \func{Find} to traverse a branch of the BST; satisfies following {\it postconditions}:\ul
		\com \postnotnull\ $l$ points to a Leaf node and $p$ points to an Internal node\ul
		\com \postl\ Either $p\rightarrow \lft$ has contained $l$ (if $k<p\rightarrow key$) or $p\rightarrow right$ has contained $l$ (if $k\geq p\rightarrow key$)\ul
		\com \postpup\ $p\rightarrow update$ has contained $pupdate$\ul
		\com \postnonempty\ if $l\rightarrow key\neq \infty_1$, then the following three statements hold:\ul
		\com \hspace*{3mm}\postgpnotnull\ $gp$ points to an Internal node\ul
		\com \hspace*{3mm}\postp\ either $gp\rightarrow \lft$ has contained $p$ (if $k<gp\rightarrow key$) or $gp\rightarrow right$ has contained $p$ (if $k\geq gp\rightarrow key$)\ul
		\com \hspace*{3mm}\postgpup\ $gp\rightarrow update$ has contained $gpupdate$\nlc
		Internal *$gp$, *$p$\nlc
		Node *$l:= Root$  \label{restart-search}\nlc
		
		Update $gpupdate, pupdate$ \tabtabcom Each stores a copy of an $update$ field\bl\nlc
		%\com The initial values of $p,gp,pupdate$ and $gpupdate$ are unimportant
		%$p :=\NULL$ 
		%$pupdate := \langle \clean, \NULL\rangle$\nlc
		while $l$ points to an internal node \{ \nlc%\com Advance down the tree\nlc
		\n         $gp := p$ \tabtabcom Remember parent of $p$\nlc
		$p := l$ \tabtabcom Remember parent of $l$\nlc
		$gpupdate := pupdate$ \tabtabcom Remember $update$ field of $gp$\nlc
		$pupdate := p\rightarrow update$\label{store-pupdate}\tabtabcom Remember $update$ field of $p$\nlc  
		%           if $pupdate.state = \mk$ then \label{checkmark} \{\nlc
		%\n              \func{HelpMarked}$(pupdate.info)$ \label{call-hm3}\nlc
		%                goto line \lref{restart-search} \com Restart traversal\nlc
		%\p         \}\nlc        
		if $k < l\rightarrow key$ then $l:= p\rightarrow \lft$ else $l:=p \rightarrow right$ \label{read-child}\tabtabcom Move down to appropriate child\nlc
		%           if $pupdate \neq p\rightarrow update$ then goto line \lref{restart-search} \com Restart traversal \label{reread-pupdate}\nlc
		\p \} \nlc
		return $\langle gp, p, l, pupdate, gpupdate \rangle$ \nlc
		\p 
		\}\bl
		\nlc
		\func{Find}($\Key\ k$) : Leaf* \{ \nlc
		\n   Leaf *$l$\bl
		\nlc
		%     Update $pupdate,gpupdate$\ul \nlc
		$\langle -,-,l,-,-\rangle := \func{Search}(k)$\nlc
		if $l\rightarrow key = k$ then return $l$\nlc
		else return \NULL\nlc
		\p
		\}\bl
		\nlc
		\func{Insert}($\Key\ k$) : boolean \{ \nlc
		\n Internal *$p$, *$newInternal$\nlc 
		Leaf *$l$, *$newSibling$\nlc 
		Leaf *$new :=$ pointer to a new Leaf node whose $key$ field is $k$  \nlc
		Update $pupdate, result$\nlc
		\IFlag\ *$op$\bl\nlc%:=$ pointer to a new \IFlag\ \record\ \ul\nlc
		while \TRUE\ \{  \nlc
		\n $\langle -, p, l, pupdate, - \rangle := \func{Search}(k)$ \label{ins-search}\nlc
		if $l \rightarrow key = k$ then return \FALSE\ \tabtabcom Cannot insert duplicate key\label{insert-false}\nlc
		if $pupdate.state \neq \clean$ then \func{Help}$(pupdate)$ \tabtabcom Help the other operation \label{ins-help-unclean}\nlc
		else \{\nlc
		\n        $newSibling :=$ pointer to a new Leaf whose key is $l\rightarrow key$\nlc
		$newInternal :=$ pointer to a new Internal node with $key$ field $\max(k, l \rightarrow key)$,\label{create-internal}\ul      
		\n      $update$ field $\langle \clean, \NULL\rangle$, and with two
		child fields equal to $new$ and $newSibling$\ul 
		(the one with the smaller key is the left child)\nlc
		\p        $op :=$ pointer to a new \IFlag\ \record\  containing $\langle p, l, newInternal, \textcolor{blue}{\FALSE}\rangle$\label{new-IFlag}\nlc
		\textcolor{blue}{$Announce[p] := op$} \nlc
		$result := \CASB(p\rightarrow update, pupdate, \langle \insertflag, op\rangle)$ \tabtabcom {\bf iflag \CASB}\label{iflag-cas} \nlc
		if $result = pupdate$ \textcolor{blue}{or $result = \langle \insertflag,op \rangle$} then \{ \tabtabcom The iflag \CASB\ was successful\nlc
		\n            \func{HelpInsert}$(op)$ \tabtabcom Finish the insertion\label{finish-insert}\nlc
		return \TRUE\ \label{insert-true}\nlc
		\p        \}\nlc 
		else \func{Help}$(result)$ \tabcom The iflag \CASB\ failed; help the operation that caused failure\label{ins-help-after-failure}\nlc
		\p    \}\nlc
		\textcolor{blue}{if $op \rightarrow complete = \TRUE$ then} \nlc
		\n		\textcolor{blue}{return \TRUE} \nlc \p
		\p\}\nlc 
		\p
		\}\bl
		\nlc
		
		\func{HelpInsert}(\IFlag\ *$op$) \{\ul
		\n     \com {\it Precondition}:  $op$ points to an \IFlag\ \record\  (\ie, it is not $\NULL$)\nlc
		\func{CAS-Child}$(op\rightarrow p, op\rightarrow l, op\rightarrow newInternal)$ \tabtabcom {\bf ichild \CASB}\label{ichild-cas}\nlc
		$\CASB(op\rightarrow p\rightarrow update, \langle \insertflag, op \rangle, \langle \clean, op\rangle)$ \tabtabcom {\bf iunflag \CASB} \label{iunflag-cas}\nlc
		\textcolor{blue}{$op \rightarrow complete := \TRUE$} \tabtabcom {mark the operation as completed}\nlc 
		\p
		\}
	\end{code}
	\caption{\label{code2}Pseudocode for \func{Search}, \func{Find} and \func{Insert}.}
\end{figure*}

\begin{figure*}
	\scriptsize
	\begin{code}
		\firstline
		\func{Delete}($\Key\ k$) : boolean \{\nlc
		\n Internal *$gp$, *$p$\nlc
		Leaf *$l$\nlc
		Update $pupdate, gpupdate, result$\nlc
		\DFlag\ *$op$\bl\nlc
		while \TRUE\ \{ \nlc
		\n     $\langle gp, p, l, pupdate, gpupdate \rangle := \func{Search}(k)$\label{del-search}\nlc
		if $l\rightarrow key \neq k$ then return \FALSE\ \tabtabcom Key $k$ is not in the tree\label{delete-false}\nlc
		%       assert: $gp$ points to an internal node\label{assert-gpnotnull}\nlc
		if $gpupdate.state \neq \clean$ then \func{Help}$(gpupdate)$ \label{del-help-unclean-1}\nlc
		else if $pupdate.state \neq \clean$ then \func{Help}$(pupdate)$\label{del-help-unclean-2}\nlc
		else \{ \tabtabcom Try to flag $gp$\nlc
		\n          $op :=$ pointer to a new \DFlag\ \record\  containing $\langle gp, p, l, pupdate, \textcolor{blue}{\FALSE} \rangle$\label{new-DFlag}\nlc
		\textcolor{blue}{$Announce[p] := op$} \nlc
		$result := \CASB(gp\rightarrow update, gpupdate, \langle \deleteflag, op \rangle)$ \tabtabcom {\bf dflag \CASB}\label{dflag-cas}\nlc
		if $result = gpupdate$ \textcolor{blue}{ or $result = \langle \deleteflag, op \rangle$} then \{ \tabtabcom \CASB\ successful \nlc
		\n             if \func{HelpDelete}$(op)$ then return \TRUE \label{delete-true} \tabtabcom Either finish deletion or unflag\nlc
		\p          \}\nlc                 
		else \func{Help}$(result)$ \tabcom The dflag \CASB\ failed; help the operation that caused the failure \label{del-help-after-failure}\nlc%; help operation that is in the way 
		\p     \}\nlc
		\textcolor{blue}{if $op \rightarrow complete = \TRUE$ then} \nlc
		\n		\textcolor{blue}{return \TRUE} \nlc \p
		\p \}\nlc
		\p
		\}\bl
		\nlc
		
		\func{HelpDelete}(\DFlag\ *$op$) : boolean \{\ul
		\n   \com {\it Precondition}:  $op$ points to a \DFlag\ \record\  (\ie, it is not \NULL)\nlc%$op\rightarrow pupdate.state = \clean$\nlc
		Update $result$ \tabtabcom Stores result of mark \CASB\bl\nlc
		%\com First, try to mark the node that $op\rightarrow p$ points to\nlc
		%$pupdate = op\rightarrow pupdate$\nlc
		$result := \CASB(op\rightarrow p\rightarrow update, op\rightarrow pupdate, \langle \mk, op\rangle)$ \tabtabcom {\bf mark \CASB}\label{mark-cas}\nlc     
		if $result = op\rightarrow pupdate$ or $result = \langle \mk, op\rangle$ then \label{checkmark}\{\tabtabcom $op\rightarrow p$ is successfully marked\nlc
		\n          \func{HelpMarked}$(op)$ \label{call-hm1} \tabtabcom Complete the deletion\nlc
		return \TRUE\tabtabcom Tell \func{Delete} routine it is done\nlc
		\p       \}\nlc
		else \{\tabtabcom The mark \CASB\ failed \nlc
		\n              
		\func{Help}$(result)$\label{help-after-failed-mark}\tabtabcom Help operation that caused failure\nlc
		\CASB$(op\rightarrow gp\rightarrow update, \langle \deleteflag, op\rangle, \langle \clean, op\rangle)$ \tabtabcom {\bf backtrack \CASB}\label{backtrack-cas}\nlc
		
		return \FALSE\tabtabcom Tell \func{Delete} routine to try again\nlc
		\p       \}\nlc
		\p
		\}\bl
		\nlc
		
		\func{HelpMarked}(\DFlag\ *$op$) \{\ul
		\n   \com {\it Precondition}:  $op$ points to a \DFlag\ \record\  (\ie, it is not $\NULL$)\nlc
		Node *$other$\bl\ul
		\com Set $other$ to point to the sibling of the node to which  $op\rightarrow l$ points \nlc
		if $op\rightarrow p\rightarrow right = op\rightarrow l$ then $other := op\rightarrow p \rightarrow \lft$ else $other:=op\rightarrow p\rightarrow right$\label{read-other}\ul 
		\com Splice the node to which $op\rightarrow p$ points out of the tree, replacing it by $other$\nlc
		\func{CAS-Child}$(op\rightarrow gp, op\rightarrow p, other)$ \tabtabcom {\bf dchild \CASB}\label{dchild-cas}\nlc
		%     \com Unflag the node $op\rightarrow gp$ points to\nlc
		\CASB$(op\rightarrow gp\rightarrow update, \langle \deleteflag, op\rangle, \langle \clean, op\rangle)$ \tabtabcom {\bf dunflag \CASB}\label{dunflag-cas}\nlc
		\textcolor{blue}{$op \rightarrow complete := \TRUE$} \tabtabcom {mark the operation as completed} \nlc
		\p
		\}\bl\nlc
		
		\func{Help}(Update $u$) \{ \tabtabcom General-purpose helping routine\ul
		\n    \com {\it Precondition}:  $u$ has been stored in the $update$ field of some internal node\nlc
		if $u.state = \insertflag$ then \func{HelpInsert}$(u.\info)$\label{call-HelpInsert}\nlc
		else if $u.state = \mk$ then \func{HelpMarked}$(u.\info)$\label{call-hm2}\nlc
		else if $u.state = \deleteflag$ then \func{HelpDelete}$(u.\info)$\label{call-HelpDelete}\nlc
		\p
		\}\bl
		\nlc
		\func{CAS-Child}(Internal *$parent$, Node *$old$, Node *$new$) \{\label{CAS-Child}\ul
		\n  \com {\it Precondition}:  $parent$ points to an Internal node and $new$ points to a Node (\ie, neither is \NULL)\ul
		\com This routine tries to change one of the child fields of the node that $parent$ points to from $old$ to $new$.\nlc
		if $new \rightarrow key < parent\rightarrow key$ then\label{which-child}\nlc
		\n       \CASB$(parent\rightarrow \lft, old, new)$\label{child-cas-1}\nlc
		\p  else\nlc
		\n       \CASB$(parent\rightarrow right, old, new)$\label{child-cas-2}\nlc
		\p\p 
		\}
	\end{code}
	\caption{\label{code3}Pseudocode for \func{Delete} and some auxiliary routines.}
\end{figure*}