%% For double-blind review submission, w/o CCS and ACM Reference (max submission space)
\documentclass[sigplan,review,anonymous]{acmart}\settopmatter{printfolios=true,printccs=false,printacmref=false}
%% For double-blind review submission, w/ CCS and ACM Reference
%\documentclass[sigplan,review,anonymous]{acmart}\settopmatter{printfolios=true}
%% For single-blind review submission, w/o CCS and ACM Reference (max submission space)
%\documentclass[sigplan,review]{acmart}\settopmatter{printfolios=true,printccs=false,printacmref=false}
%% For single-blind review submission, w/ CCS and ACM Reference
%\documentclass[sigplan,review]{acmart}\settopmatter{printfolios=true}
%% For final camera-ready submission, w/ required CCS and ACM Reference
%\documentclass[sigplan]{acmart}\settopmatter{}


%% Conference information
%% Supplied to authors by publisher for camera-ready submission;
%% use defaults for review submission.
%\acmConference[PL'18]{ACM SIGPLAN Conference on Programming Languages}{January 01--03, 2018}{New York, NY, USA}
%\acmYear{2018}
%\acmISBN{} % \acmISBN{978-x-xxxx-xxxx-x/YY/MM}
%\acmDOI{} % \acmDOI{10.1145/nnnnnnn.nnnnnnn}
%\startPage{1}

%% Copyright information
%% Supplied to authors (based on authors' rights management selection;
%% see authors.acm.org) by publisher for camera-ready submission;
%% use 'none' for review submission.
\setcopyright{none}
%\setcopyright{acmcopyright}
%\setcopyright{acmlicensed}
%\setcopyright{rightsretained}
%\copyrightyear{2018}           %% If different from \acmYear

%% Bibliography style
\bibliographystyle{ACM-Reference-Format}
%% Citation style
%\citestyle{acmauthoryear}  %% For author/year citations
%\citestyle{acmnumeric}     %% For numeric citations
%\setcitestyle{nosort}      %% With 'acmnumeric', to disable automatic
                            %% sorting of references within a single citation;
                            %% e.g., \cite{Smith99,Carpenter05,Baker12}
                            %% rendered as [14,5,2] rather than [2,5,14].
%\setcitesyle{nocompress}   %% With 'acmnumeric', to disable automatic
                            %% compression of sequential references within a
                            %% single citation;
                            %% e.g., \cite{Baker12,Baker14,Baker16}
                            %% rendered as [2,3,4] rather than [2-4].


%%%%%%%%%%%%%%%%%%%%%%%%%%%%%%%%%%%%%%%%%%%%%%%%%%%%%%%%%%%%%%%%%%%%%%
%% Note: Authors migrating a paper from traditional SIGPLAN
%% proceedings format to PACMPL format must update the
%% '\documentclass' and topmatter commands above; see
%% 'acmart-pacmpl-template.tex'.
%%%%%%%%%%%%%%%%%%%%%%%%%%%%%%%%%%%%%%%%%%%%%%%%%%%%%%%%%%%%%%%%%%%%%%


%% Some recommended packages.
\usepackage{booktabs}   %% For formal tables:
                        %% http://ctan.org/pkg/booktabs
\usepackage{subcaption} %% For complex figures with subfigures/subcaptions
                        %% http://ctan.org/pkg/subcaption

%\usepackage{cite}
\usepackage{epic,eepic}
\usepackage{graphics}
\usepackage{graphicx}
%\usepackage{xspace}
\usepackage{fullpage}
\usepackage{xcolor}
\usepackage{smartref} % for referencing the line numbers
\usepackage{amsmath}
\usepackage[vlined,ruled,linesnumbered,noresetcount]{algorithm2e}
%\usepackage[noend]{algpseudocode}


% this command enables to remove a whole part of the text from the printout
% to use it just enter \remove{ before the text to be excluded and } after the text
\newcommand{\remove}[1]{}

\newcommand{\y}[1]{{\color{blue} #1}\normalcolor}
\newcommand{\yy}[1]{{\color{green} #1}\normalcolor}
\newcommand{\danny}[1]{{\color{orange} #1}\normalcolor}
\newcommand{\ohad}[1]{{\color{purple} #1}\normalcolor}

\newcommand{\here}[1]{{\bf [[[ #1 ]]]}}

\newtheorem{claim}{Claim}
\newtheorem{definition}{Definition}
\newtheorem{observation}{Observation}%[section]
\newcommand{\qedsymb}{\hfill{\rule{2mm}{2mm}}}
\newenvironment{proofsketch}{\begin{trivlist}
		\item[\hspace{\labelsep}{\bf\noindent Sketch of proof: }]
	}{\qedsymb\end{trivlist}}
\newenvironment{proofof}[1]{\begin{trivlist}
		\item[\hspace{\labelsep}{\bf\noindent Proof of #1: }]
	}{\qedsymb\end{trivlist}}

%\renewcommand{\topfraction}{.9}
%\renewcommand{\bottomfraction}{.9}
%\renewcommand{\textfraction}{.1}
%\renewcommand{\floatpagefraction}{.9}

\DontPrintSemicolon

%%%%%%%%%%%%%%%Linked-List commands%%%%%%%%%%%%%%%%%%%%%

\newcommand{\search}{\mbox{\sc Search}}
\newcommand{\recover}{\mbox{\sc Recover}}
\newcommand{\insertrecover}{\mbox{\sc Insert.Recover}}
\newcommand{\deleterecover}{\mbox{\sc Delete.Recover}}
\newcommand{\findrecover}{\mbox{\sc Find.Recover}}
\newcommand{\insertlst}{\mbox{\sc Insert}}
\newcommand{\delete}{\mbox{\sc Delete}}
\newcommand{\find}{\mbox{\sc Find}}
\newcommand{\getdata}{\mbox{get\_data}}


\newcommand{\init}{\mbox{$\bot$}}
\newcommand{\NULL}{\mbox{\sc Null}}
\newcommand{\fail}{\mbox{\sc Fail}}

\newcommand{\CAS}{\mbox{\textit{CAS}}}
\newcommand{\True}{\mbox{\textbf{true}}}
\newcommand{\False}{\mbox{\textbf{false}}}
\newcommand{\Int}{\mbox{int}}

\newcommand{\Info}{\mbox{Info}}


%%%%%%%%%%%%%%%Elimination-Stack commands%%%%%%%%%%%%%%%%%%%%%

\newcommand{\push}{\mbox{\sc Push}}
\newcommand{\trypush}{\mbox{\sc TryPush}}
\newcommand{\recoverPush}{\mbox{\sc Push-Roceover}}
\newcommand{\pop}{\mbox{\sc Pop}}
\newcommand{\trypop}{\mbox{\sc TryPop}}
\newcommand{\recoverPop}{\mbox{\sc Pop-Recover}}
\newcommand{\exchange}{\mbox{\sc Exchange}}
\newcommand{\recoverExchange}{\mbox{\sc Exchange-Recover}}
\newcommand{\visit}{\mbox{\sc Visit}}
\newcommand{\switchPair}{\mbox{\sc SwitchPair}}
%\newcommand{\search}{\mbox{\sc Search}}
%\newcommand{\recover}{\mbox{\sc Recover}}


%\newcommand{\init}{\mbox{$\bot$}}
%\newcommand{\NULL}{\mbox{\sc Null}}
%\newcommand{\fail}{\mbox{\sc Fail}}
\newcommand{\emptyst}{\mbox{\sc Empty}}
\newcommand{\waiting}{\mbox{\sc Waiting}}
\newcommand{\busy}{\mbox{\sc Busy}}
\newcommand{\timeout}{\mbox{\sc Timeout}}

%\newcommand{\Info}{\mbox{Info}}
\newcommand{\opInfo}{\mbox{OpInfo}}
\newcommand{\pushInfo}{\mbox{PushInfo}}
\newcommand{\popInfo}{\mbox{PopInfo}}
\newcommand{\csInfo}{\mbox{CSInfo}}
\newcommand{\exInfo}{\mbox{ExInfo}}


%%%%%%%%%%%%%%%BST commands%%%%%%%%%%%%%%%%%%%%%

\newenvironment{routine}[1][htb] {
	\renewcommand{\algorithmcfname}{Routine}% Update algorithm name
	\begin{algorithm}[#1]%
	}{\end{algorithm}}

\def\codeTabSpace{\hspace*{6mm}}
\newenvironment{code}%
{\begin{tabbing}%
		\codeTabSpace \= \hspace*{70mm} \= \hspace*{42mm} \= \kill%
	}%
	{\end{tabbing}%
}

\newcounter{linenum}
\addtoreflist{linenum}
\newcounter{ind}

\newcommand{\n}{\addtocounter{ind}{7}\hspace*{7mm}}
\newcommand{\p}{\addtocounter{ind}{-7}\hspace*{-7mm}}
\newcommand{\nlc}{\\\stepcounter{linenum}{\scriptsize \arabic{linenum}}\>\hspace*{\value{ind}mm}}
\newcommand{\ul}{\\\>\hspace*{\value{ind}mm}}
\newcommand{\bl}{\\[-1.5mm]\>\hspace*{\value{ind}mm}}
\newcommand{\firstline}{\stepcounter{linenum}{\scriptsize \arabic{linenum}}\>}
%\newcommand{\lref}[1]{\linenumref{#1}} % use this to refer to a line number
\newcommand{\lref}[1]{\ref{#1}} % use this to refer to a line number
% End of stuff for entering source code=====================================


\newcommand{\postnotnull}{(1)}
\newcommand{\postl}{(2)}
\newcommand{\postpup}{(3)}
\newcommand{\postnonempty}{(4)}
\newcommand{\postgpnotnull}{(4a)}
\newcommand{\postp}{(4b)}
\newcommand{\postgpup}{(4c)}

\newcommand{\tabtabcom}{\>\>\com}
\newcommand{\tabcom}{\>\com}


\newcommand{\Key}{Key}
\newcommand{\R}{{\bf R}}
%\newcommand{\NULL}{\mbox{$\bot$}}
\newcommand{\clean}{\mbox{\sc Clean}}
\newcommand{\mk}{\mbox{\sc Mark}}
\newcommand{\insertflag}{\mbox{I\Flag}}
\newcommand{\deleteflag}{\mbox{D\Flag}}
\newcommand{\record}{\mbox{record}}
\newcommand{\info}{\mbox{info}}
\newcommand{\FAIL}{\mbox{\sc Fail}}

\newcommand{\Flag}{Flag}
\newcommand{\DFlag}{DInfo}
\newcommand{\IFlag}{IInfo}

\newcommand{\ie}{{\it i.e.}}
\newcommand{\TRUE}{\mbox{\sc True}}
\newcommand{\FALSE}{\mbox{\sc False}}

\newcommand{\func}[1]{\mbox{\sc #1}}

\newcommand{\CASB}{\func{CAS}}

\newcommand{\comnospace}{\mbox{$\triangleright$}}
\newcommand{\com}{\mbox{\comnospace\ }}

\newcommand{\doublespace}{\addtolength{\baselineskip}{.8\baselineskip}}
\newcommand{\ds}{\addtolength{\baselineskip}{.9\baselineskip}}

%%%%%%%%%%%%%%%%%%%%% COMMENTS %%%%%%%%%%%%%%%%%%%%%%%%%%%%%%%%%%%%%%%%%
%
%%%% New state-of-the-art method for comment insertion
%%%% by multiple distributed authors.
%%%% These comments are present in the LATEX file and
%%%% do not appear in print.
%
%% define debug: use first def for debug, second def for final version
\newcommand{\debug}[1]{#1}
%\newcommand{\debug}[1]{}
%

%% \newcommand{\comment}[1]{\debug{\marginpar {\sloppy\tiny #1}}}
\newcommand{\lcomment}[1]{\debug{\comment{$\rightarrow$ #1}}}
\newcommand{\incomment}[1]{\debug{[[[#1]]]}}
%
\newcommand{\fM}[1]{\comment{#1 5-14 M.}}
\newcommand{\fD}[1]{\comment{#1 5-21 D.}}
\newcommand{\upcom}[1]{#1}
%
%%%%%%%%%%%%%%%% end of comments %%%%%%%%%%%%%%%%%%%%%%%%%%%%%%%%%



\newenvironment{remark}{\begin{trivlist}
		\item[\hspace{\labelsep}{\bf\noindent Remark. }]}{\end{trivlist}}
%
%

\newenvironment{centre}{\begin{center}}{\end{center}}
% sets up centre as alternate name for center.



% marginal comment
%\newcommand{\comment}[1]{\marginpar{\tiny #1}}
%\mnote{Example: only one algorithm is typed in so far.}
\newcommand{\mnote}[1]
{\marginpar%
	[{\tiny\begin{minipage}[t]{\marginparwidth}\raggedright#1%
	\end{minipage}}]%
	{\tiny\begin{minipage}[t]{\marginparwidth}\raggedright#1%
	\end{minipage}}%
}



%\newcommand{\here}[1]{[[[#1]]]\marginpar{***}}
\newcommand{\ignore}[1]{}
\newcommand{\floor}[1]{\left\lfloor #1 \right\rfloor}

\newcommand{\lft}{\mbox{\it left}}


%floating environment in two-column mode
\makeatletter
\newcommand{\removelatexerror}{\let\@latex@error\@gobble}
\makeatother






\begin{document}

%% Title information
\title{Recoverable Concurrent Data Structures: A Methodology Approach}         %% [Short Title] is optional;
                                        %% when present, will be used in
                                        %% header instead of Full Title.
%\titlenote{with title note}             %% \titlenote is optional;
                                        %% can be repeated if necessary;
                                        %% contents suppressed with 'anonymous'
%\subtitle{Subtitle}                     %% \subtitle is optional
%\subtitlenote{with subtitle note}       %% \subtitlenote is optional;
                                        %% can be repeated if necessary;
                                        %% contents suppressed with 'anonymous'


%% Author information
%% Contents and number of authors suppressed with 'anonymous'.
%% Each author should be introduced by \author, followed by
%% \authornote (optional), \orcid (optional), \affiliation, and
%% \email.
%% An author may have multiple affiliations and/or emails; repeat the
%% appropriate command.
%% Many elements are not rendered, but should be provided for metadata
%% extraction tools.

%% Author with single affiliation.
\author{First1 Last1}
\authornote{with author1 note}          %% \authornote is optional;
                                        %% can be repeated if necessary
\orcid{nnnn-nnnn-nnnn-nnnn}             %% \orcid is optional
\affiliation{
  \position{Position1}
  \department{Department1}              %% \department is recommended
  \institution{Institution1}            %% \institution is required
  \streetaddress{Street1 Address1}
  \city{City1}
  \state{State1}
  \postcode{Post-Code1}
  \country{Country1}                    %% \country is recommended
}
\email{first1.last1@inst1.edu}          %% \email is recommended

%% Author with two affiliations and emails.
\author{First2 Last2}
\authornote{with author2 note}          %% \authornote is optional;
                                        %% can be repeated if necessary
\orcid{nnnn-nnnn-nnnn-nnnn}             %% \orcid is optional
\affiliation{
  \position{Position2a}
  \department{Department2a}             %% \department is recommended
  \institution{Institution2a}           %% \institution is required
  \streetaddress{Street2a Address2a}
  \city{City2a}
  \state{State2a}
  \postcode{Post-Code2a}
  \country{Country2a}                   %% \country is recommended
}
\email{first2.last2@inst2a.com}         %% \email is recommended
\affiliation{
  \position{Position2b}
  \department{Department2b}             %% \department is recommended
  \institution{Institution2b}           %% \institution is required
  \streetaddress{Street3b Address2b}
  \city{City2b}
  \state{State2b}
  \postcode{Post-Code2b}
  \country{Country2b}                   %% \country is recommended
}
\email{first2.last2@inst2b.org}         %% \email is recommended


%% Abstract
%% Note: \begin{abstract}...\end{abstract} environment must come
%% before \maketitle command
\begin{abstract}
Recent developments foreshadow the emergence of new systems,
in which byte-addressable \emph{non-volatile main memory} (\emph{NVRAM}),
combining the performance benefits of conventional main memory
with the durability of secondary storage, co-exists with
(or eventually even replaces) traditional volatile memory.
Consequently, there is increased interest in \emph{recoverable}
concurrent objects: objects that are made robust to crash-failures
by allowing their operations to recover from such failures.
This paper presents a principled approach to deriving recoverable versions
of several widely-used concurrent data structures, 
in particular, a linked list and an elimination stack.
\end{abstract} 


%% 2012 ACM Computing Classification System (CSS) concepts
%% Generate at 'http://dl.acm.org/ccs/ccs.cfm'.
\begin{CCSXML}
<ccs2012>
<concept>
<concept_id>10011007.10011006.10011008</concept_id>
<concept_desc>Software and its engineering~General programming languages</concept_desc>
<concept_significance>500</concept_significance>
</concept>
<concept>
<concept_id>10003456.10003457.10003521.10003525</concept_id>
<concept_desc>Social and professional topics~History of programming languages</concept_desc>
<concept_significance>300</concept_significance>
</concept>
</ccs2012>
\end{CCSXML}

\ccsdesc[500]{Software and its engineering~General programming languages}
\ccsdesc[300]{Social and professional topics~History of programming languages}
%% End of generated code


%% Keywords
%% comma separated list
\keywords{keyword1, keyword2, keyword3}  %% \keywords are mandatory in final camera-ready submission


%% \maketitle
%% Note: \maketitle command must come after title commands, author
%% commands, abstract environment, Computing Classification System
%% environment and commands, and keywords command.
\maketitle

\section{Introduction}

Recent years has seen the emergence of systems in which
byte-addressable \emph{non-volatile main memory} (\emph{NVRAM}),
combining the performance benefits of conventional main memory
with the durability of secondary storage,
co-exists with (or eventually even replaces) traditional volatile memory.
This has lead to increased interest in the \emph{crash-recovery} model,
in which a failed process may be resurrected after it crashes.
Of particular interest is the design of \emph{recoverable concurrent
objects} (also called
\emph{persistent}~\cite{CoburnCAGGJW-Asplos2011,ChenQ-VLDB2015}
or \emph{durable}~\cite{VenkataramanTRC-FAST2011}):
% YF: persistent is misleading
% HA: But it's a term used in the literature; added citations.
objects that are made robust to crash-failures by allowing their operations
to recover after such failures.

It is challenging to design data structures that persist in the presence of
crashes and recoveries, and several concurrent implementations were proposed. %%% list them here
While many of these exploit specific aspects of an object to
optimize the implementation, it is important to develop generic
approaches for deriving recoverable implementations from their
non-recoverable counterparts.
Such an approach should preserve the structure and efficiency of
the implementation, as much as possible,
while avoiding the need to design new algorithms.

This paper presents a principled approach to deriving recoverable
objects and describe it in detail through three widely-used
concurrent data structures:
a linked list~\cite{DBLP:conf/wdag/Harris01},
binary search tree~\cite{}
and elimination stack~\cite{DBLP:journals/jpdc/HendlerSY10}.

Our results are presented in the context of an abstract individual-process
crash-recovery model for non-volatile memory~\cite{AttiyaBH-PODC2018},
in which processes communicate via non-volatile shared-memory variables.
%%% I have taken out the issue of volatile / non-volatile local variables
%%% and the exclusion of the program counter.
%%% It should be mentioned later in the introduction.
%Each process also has local variables stored in volatile processor registers
%[[YF: What about program counter? HA: in this paper, it can be volatile.]]
At any point, a process may incur a crash-failure,
causing all its local variables, [[[except for its program counter,???]]]
to be reset to arbitrary values.
Operation response values are returned via volatile processor registers,
which may become inaccessible to the calling process if it fails
just before persisting the response value.
[[[Each data structure operation???]]] has an associated \emph{recovery
function} that is
responsible for restoring it upon the recovery from a crash-failure.
The recovery function completes the current outstanding operation on
the data structure, if there was any, returning either its \emph{response}
or a \emph{fail} indication, if it was unsuccessful.
Both responses are consistent with the resulting state of the data structure,
to which the operation was applied (in the former case)
or not (in the latter case).

\subsection*{Related Work}

Several different correctness conditions and implementations have been
proposed for recoverable data
structures~\cite{CoburnCAGGJW-Asplos2011,ChenQ-VLDB2015,VenkataramanTRC-FAST2011,%
Aguilera2003StrictLA,GolabR16,DBLP:conf/opodis/BerryhillGT15,DBLP:conf/icdcs/GuerraouiL04,DBLP:conf/wdag/IzraelevitzMS16}.
However, this work concentrates on maintaining the consistency of
the concurrent object in the face of crash failures,
and does not consider \emph{detectability}, that is, the ability to
conclude upon recovery whether the crashed operation took effect.

There is a universal implementation~\cite{DBLP:journals/pacmpl/CohenFL17}
which is both durable [[undefined?]] and lock-free,
it uses only read, write and CAS.
This implementation applies at most one \emph{persistence[[?]] fence}
(flushing the contents of the memory) per operation, which is optimal,
but it keeps the entire history of the object in a designated shared queue;
it also keeps a per-process persistent log, such that these logs together
keep the entire history, and different logs may have a big overlap.
Furthermore, to determine the response of an operation,
it is necessary to read the entire history up to its linearization point.
This construction can easily be made detectable since an operation was
linearized if and only if it appears in the shared queue
(representing the object's history).
Specifically, after the system completes its recovery routine,
in the recovery from an $Update(op)$, 
the process can determine the response value from the shared queue;
this assumes that each $op$ is uniquely identified.
This work considers a \emph{system-wide} crash model, in which
all processes crash together and a single recovery function is
executed upon recovery,
in order to consistently reconstruct the queue data structure.
It ensures \emph{durable linearizability}~\cite{VenkataramanTRC-FAST2011}:
after a full system crash, the state of the object must reflect a consistent operation
sub-history that includes all operations completed by the time of the crash,
i.e., crashed operations may get lost.

\emph{Nesting-safe recoverable linearizability} (\emph{NRL})~\cite{AttiyaBH-PODC2018}
is a novel crash-recovery model together with a correctness condition.
It associates each recoverable operation with a recovery function, invoked 
after a crash in the operation to help a process to complete the operation, 
as well as restore the response value if needed.
They give recoverable implementations for read, write and CAS.
As suggested by its name, NRL support \emph{nesting},
so taking an algorithm that uses only read, write and CAS,
and replacing each primitive with its recoverable version yields an NRL implementation.
Some minor changes are still needed in order to use this transformation,
but they hold for all the implementations presented in that paper.
However, this transformation is quite costly, in terms of both time and space.

Indeed, implementations using only read, write and CAS can be made
recoverable and detectable~\cite{DBLP:journals/corr/abs-1806-04780},
by partitioning the code into \emph{capsules},
each containing a single CAS followed by some number of reads,
and replacing each CAS with its recoverable version.
This allows to recover from a crash inside the capsule.
\emph{Normalized} implementations~\cite{TimnatP-PPoPP2014} can be
further optimized so that an operation contains only two capsules.
However, not all implementation are given in a normalized form, and
converting an implementation into a normalized form may be costly by itself.
This general transformation has several drawbacks:
For example, replacing a CAS with its recoverable variant requires each
CAS to have distinct arguments, ensured by adding unique sequence numbers,
which are stored in the CAS location.
This means that CAS is applied to words with unbounded length,
even if the original implementation applied CAS over a finite domain.
Furthermore, although two capsules are used for each normalized operation,
these two capsules are repeatedly executed in attempt to complete the operation.
Thus, the implementation is lock-free.

In many cases we can avoid it,
without the extra capsule complexity for a failed attempt.
Moreover, we would like the transformation to change as little
as necessary from the original code,
even at the price of having a costly recover function.
Assuming crashes are rare,
this may yield a more efficient implementation in practice. 

\section{Overview of Our Approach}

Our methodology separately considers the operations provided by
a linearizable implementation $A$ of a data structure $Q$,
according to their properties.
In this initial description, we assume each operation becomes
visible and is linearized through at most a single \CAS;
this simplifies recovery by avoiding scenario of an
operation changing the data structure but is yet to complete
and be linearized.
% we didn't discuss helping yet!
% while some other helping process may complete and linearize it.
(Later in this section, discuss how to remove this assumption,
in the context a binary search tree.)

Consider an implementation of an operation $Op$ that does not change
the data structure, as is often the case with a search operation.
In this case, the operation remains intact, and the recovery
function simply restarts it from scratch.
(Later, in the context of the linked list, we explain how this
approach is extended to the case where stopping $Op$ at any point
and restarting it does not violate linearizability.)
% say this later, not here:
%i.e., it is indistinguishable to all other processes from a run in which it is executed only once.

For operations that do change the data structure, assume first that
there is a way to uniquely identify each instance of $Op$.
In this case, $Op$ can be made recoverable by adding an indication
once it is visible.
The recovery function then checks for this indication:
if found, the operation has completed, and its response is returned;
otherwise, the operation is restarted.
For example, assume process $p$ tries to add a new unique node $nd$
to the data structure using \CAS\
(other processes may add different nodes with the same data).
Once $nd$ is added, it can be found in the data structure,
indicating that the operation has completed.
However, this indication is lost when $nd$ is removed.
This problem can be overcome by flagging a special field in $nd$ before
removing it.
This way, we have two indicators,
one is set once $nd$ is added and the other when $nd$ is deleted.

A further complication is posed when the implementation employs a helping
mechanism, namely, several processes attempt to perform an operation,
%%% what do mean in the next sentence? each attempt is not uniquely identified?
which is not uniquely identified.
When $p$ crashes during an execution of $Op$ and recovers, it might be
able to check whether $Op$ is complete, but it is unable to
know whether it is the one to perform it, or some other process.
As a result, $p$ may not be able to recover the right response value.
% but why it is a problem? if the operation is uniquely identified, then who cares who performed it?
This is resolved by adding an \emph{owner} field (for each operation type),
used for agreeing which process performed the operation;
after a process $p$ completes an operation, \CAS\ its id to \emph{owner},
to declare itself as the winner.
If $p$ crashes during an execution of $Op$ and recovers,
$p$ first checks whether the operation is visible, and if so,
$p$ tries to set $owner$ to $p$ with \CAS\ and then responds according to
the id stored in $owner$.
%For example, trying to delete a node $nd$ from the data-structure is such an operation.
%An indication for it being visible is that $nd$ is no longer part of the data-structure.
%However, several process may try and delete $nd$ concurrently.
(This is demonstrated in the \emph{delete} of our Linked-List implementation.)

In all cases, it is necessary to persist the response value before returning it.
In some cases, $p$ may also need to store some recovery data;
for example, the node to be added or removed.
This information is stored in a designate memory location, for each process;
after the process stores this information and before the process starts
performing the operation,
a checkpoint is set to signify that the data is relevant for the current operation.

We next further explain our approach with a relatively simple example,
of a linked-list set, and then explain some extensions with a
\emph{binary search tree}.
Later in the paper (Section~\ref{section:elimination-stack}),
we present a comprehensive example
of taking a sophisticated concurrent data structure,
an \emph{elimination stack}, and making it recoverable.

\subsection*{Linked List}

Algorithms~\ref{alg:linked-list} and~\ref{alg:linked-list2} show
Harris' classic lock-free \emph{linked list}
set~\cite{DBLP:conf/wdag/Harris01}.\footnote{
    The \delete\ pseudo-code is slightly optimized so that if the key is found and
    is later logically deleted, then \delete\ returns \True\ if the logical deletion
    was performed by the current process, and \False\ otherwise
    (lines \ref{delete-while-loop}-\ref{delete:logical-delete}, \ref{delete:return-res}).}
It supports \find, \insertlst\ and \delete\;
the latter two use the \search\ helper function in order to find the node with
the smallest key greater than or equal to the input key (denoted \emph{curr})
and its predecessor in the list (denoted \emph{pred}).
\delete\ first logically deletes a node by marking it
as deleted and then physically removes it from the list.
\ohad{[add here description how this is done, or reference to details in section?!][add a bit more detail]}
%While traversing the list, \search\ attempts to physically remove any marked node
%it encounters from the list.



\begin{figure}[!t]
	\removelatexerror

\begin{algorithm}[H]

	\footnotesize

	\begin{flushleft}	
	\textbf{Type} Node \{MarkableNodeRef $*next$, int \emph{key}\}\\
	\textbf{Shared variables:} Node $*head$ \\	
	\end{flushleft}	

%%%%%%%%%%%%%%%%%%%%%% Linked List - Search %%%%%%%%%%%%%%%%%%%%%%%%

	\begin{procedure}[H]
  		\caption{() $\langle$Node $*$, Node $*$$\rangle$ \search\ (T $key$)}

		Node $*pred, *curr, *succ$ \;
		retry: \While{\True} { \label{find-outer-loop}
			$pred := head$ \;
			$curr := pred.next$ \label{search-read-cur}\;	
			\While{\True}{
				$succ := curr.next$ \;
				\tcp{if $curr$ was logically deleted}
				\uIf {$marked(succ)$} {
					\uIf (\tcp*[f]{Help physical delete}) {$pred.next.\CAS\ $ ($<0,curr>$,$<0,succ>$) = \False}{
					\textbf{go to} retry \tcp*{Help failed}
					}
					$curr := succ$ \tcp*{Help succeeded}
				} \uElse {
					\uIf (\tcp*[f]{First unmarked node with key $\geq key$}) {$curr.key \geq key$} {
					\KwRet $\langle pred,curr \rangle$
					}
					$pred := curr$ \tcp*{Advance $pred$}
					$curr := succ$ \tcp*{Advance $curr$}
				}
			}
		}
	\end{procedure}

%%%%%%%%%%%%%%%%%%%%%% Linked List - Find %%%%%%%%%%%%%%%%%%%%%%%%

	\begin{procedure}[H]
  		\caption{() boolean \find\ (T $key$)}

		Node $*curr := head$ \label{linked-list-find-start} \;
		\tcc{Search for first node with key $\geq$ $key$}
		\While {$curr.key < key$}{
			$curr = curr.next$ \label{find-read-cur}\;	
		}
		\KwRet ($curr.key = key\ \&\&\ \neg marked(curr.next)$) \label{linked-list-find-end}
	\end{procedure}

\caption{Linked list: \search\ and \find} \label{alg:linked-list}
\end{algorithm}

\end{figure}




\begin{figure}[!t]
\removelatexerror
	
\begin{algorithm}[H]
		
	\footnotesize
	\begin{flushleft}	
	\end{flushleft}

%%%%%%%%%%%%%%%%%%%%%% Linked List - Insert %%%%%%%%%%%%%%%%%%%%%%%%

	\begin{procedure}[H]
  		\caption{() boolean \insertlst\ (T $key$)}
  		%\textbf{Private variables:}
  		Node $*pred, *curr$ \label{ll:insert-entry} \;
		Node $newnd$ := \textbf{new} Node ($key$) \;
		\While{\True}{ \label{ll:insert-while}
			\tcp{Search the right location to insert}
			$\langle pred, curr \rangle := \search(key)$ \label{ll:insert-search} \;
			\uIf (\tcp*[f]{$key$ in the list}) {$curr.key = key$ \label{ll:insert-if-in-list}} {
				\KwRet \False \label{ll:insert-return-false}
			} \uElse {
				$newnd.next$ := $<0,curr>$ \;
				\tcp{Try to add $newnd$}
				\uIf {$pred.next.\CAS$ ($<0,curr>$, $<0,newnd>$)} {\KwRet \True} \label{ll:insert-CAS}
			}		
		}
	\end{procedure}


%%%%%%%%%%%%%%%%%%%%%% Linked List - Delete %%%%%%%%%%%%%%%%%%%%%%%%

	\begin{procedure}[H]
		\caption{() boolean \delete\ (T $key$)}

		Node $*pred, *curr, *succ$ $\quad$  boolean $res := \False$ \label{ll:delete-entry} \;
		$\langle pred, curr \rangle := \search (key)$ \tcp*{Search for $key$} \label{ll:delete-search}
		\uIf {$curr.key \neq key$ \label{ll:delete-if-in-list}} {
			\KwRet \False \tcp*[f]{$key$ not in the list} \label{ll:delete-return-false}
		}
		\uElse {
			\While (\tcp*[f]{Repeatedly attempt logical delete}) {$\lnot marked(curr.next)$  \label{ll:delete-while}}{
				$succ := curr.next$ \label{ll:delete-set-succ-to-next}\;
				$res := curr.next.\CAS$ ($<0,succ>$, $<1,succ>$) \label{ll:delete-logical-delete}\;
			}
			\tcp{Physical deletion attempt}
			$succ := curr.next$ \label{ll:delete-read-succ-after-CAS}\;
			$pred.next.\CAS$ ($<0,curr>$, $<0,succ>$) \label{ll:delete-physical-delete}
			\KwRet $res$ \label{ll:delete-return-res}
			}

	\end{procedure}

\caption{Linked list: \insertlst\ and \delete} \label{alg:linked-list2}
\end{algorithm}

\end{figure}



To find a key $k$, a process simply looks for an unmarked node with key $k$
(lines~\ref{linked-list-find-start}-\ref{linked-list-find-end}).
To insert a key $k$, process $p$ calls \search\ in order to find the position
in the list where $k$ should be added (line~\ref{ll:insert-search})
If $k$ is already in the list, \insertlst\ returns \False\
(lines \ref{ll:insert-if-in-list}-\ref{ll:insert-return-false}),
otherwise it tries to set \emph{pred}.\emph{next} to point to a new node
containing $k$ using \CAS\ (line~\ref{ll:insert-CAS});
this fails if \emph{pred} has been logically deleted in the interim.
To delete a key $k$, $p$ calls \search\, returning \False\ if $k$ not in the set
(lines \ref{ll:delete-if-in-list}-\ref{ll:delete-return-false}).
Otherwise, $p$ repeatedly tries to logically delete it by marking the \emph{next}
field of its node using \CAS, until the node is marked
(lines \ref{ll:delete-while}-\ref{ll:delete-logical-delete}).
After the node is marked, $p$  tries to physically remove it
(lines \ref{ll:delete-set-succ-to-next}-\ref{ll:delete-physical-delete}).

The operations are linearized as follows:
A \emph{Find} is linearized with the last read of \emph{curr}
(line~\ref{find-read-cur}), \find\ returns \True\ if node \emph{curr} is unmarked
and \False\ otherwise.
An \emph{Insert} is linearized either when the \search\ called by \insertlst\
reads (line~\ref{search-read-cur}) an unmarked node with key $k$,
when \insertlst\ returns \False,
or with a successful \CAS\ inserting $k$ to the list (line~\ref{ll:insert-CAS}),
when \insertlst\ returns \True.
A \emph{delete} is linearized either when the \search\ instance it calls
reads in line~\ref{search-read-cur} an unmarked list node with a key greater
than $k$ (in which case \delete\ returns \False),
upon a successful logical deletion by the current operation
(in line~\ref{ll:delete-logical-delete}, \delete\ returns \True),
or upon a logical deletion of the node by another concurrently executing
process (in line~\ref{delete:logical-delete}, \delete\ returns \False).

Thus, successful \insertlst\ and \delete\ are linearized when
they change the data-structure in a manner visible to other processes,
e.g., after the successful \CAS\ in line~\ref{ll:insert-CAS} of \insertlst,
or after a successful logical deletion in line~\ref{ll:delete-logical-delete} of \delete.
The instructions in lines~\ref{ll:insert-CAS} and~\ref{ll:delete-logical-delete}
are the {\em realization} \CAS\ for \insertlst\ and \delete, respectively.

Consider now the situation when some process $p$ recovers from a crash-failure.
By the way linearization points are assigned to operations and by the definition
of realization \CAS\ for \insertlst\ and \delete,
the algorithm satisfies \textit{strict linearizability} \cite{Aguilera2003StrictLA}:
a crashed operation can be either removed from the history or linearized
between its invocation and crash.
If $p$ failed during an operation $Op$,
either the realization \CAS\ for $Op$ has already been executed
in which case the effect of $Op$ becomes visible
and $Op$ is linearized in that \CAS,
or $p$ crashes at some earlier point of $Op$'s exeution
in which case none of the steps $Op$ performed before crashing
will ever become visible and $Op$ is not linearized.

However, the response of the failed operation may be lost.
For example,
in a scenario in which process $p$ performs $\delete(k)$
and crashes immediately after applying a \CAS\ to mark the node containing
key $k$, $p$ has no way of knowing when it recovers whether its failed operation
had any effect on the set. Specifically, $p$ cannot know whether its \delete\
has executed its realization \CAS, and if so what response it should return.

We now explain our approach to make this algorithm recoverable.
(See detailed description and code in Section~\ref{section:linked-list}.)
Our approach leaves \find{} as is, since it is a read-only operation.
\search{} is not read-only, but it can be stopped and restarted at any point,
without violating the linearizability of the resulted history,
and hence need not be changed.

\insertlst{} is uniquely identify by the node $nd$ it tries to add,
as different instances creates different nodes.
Moreover,
once the operation is visible and linearized in a successful realization \CAS,
$nd$ can be found in the list, indicating the operation is visible.
Finally, $nd$ is marked before it is physically removing from the list,
indicating that $nd$ was in the list.
This means there are two indications, one of which holding if
a realization \CAS\ was successfully performed before the crash,
and none of which holds, otherwise.

Our approach makes these operations recoverable,
by testing these indications upon recovery.
For \delete, note that $nd$ logical deletion (in a successful realization \CAS)
is indicated in $nd$ itself, which is marked, and will stay so forever.
Therefore, if $p$ crashes during a delete of $nd$, then $p$ can conclude,
upon recovery, whether $nd$ was deleted by checking if it is marked.

However, $p$ can not tell whether it is the process to delete $nd$.
We add a \emph{deleter} field to each node, so that once the node is deleted,
the first process to \CAS{} its id to \emph{deleter} is the one to perform the \delete.

\remove{
It is easy to make an operation crash-persistent by using crash-persistent CAS,
since an operation is linearized at its first successful CAS that
does not help other operations.
However, the operation is not necessarily detectable:
if there is a crash right after this CAS, then, upon recovery,
the crashed process needs to know whether the CAS was successful or not,
in order to conclude the right response value.
Moreover, two or more processes can try to perform the same operation,
for example, helping to delete a node; upon recovery the process not only
has to know whether the node was deleted,
but also to determine it has deleted the node or some process.

We address these problems by exploiting two properties of the implementation.

First, data is added to the linked-list nodes, ensuring they are unique,
even if they hold the same key.
A node \emph{nd} inserted to the list with a CAS will stay accessible until it is deleted
and the recovery function of the Insert operation, can check with the insert was
successful by testing the former two conditions. % did I miss something? which two conditions?

For \delete, we notice a logically deleted node (whose marked bit is set)
remains so forever, and therefore if process $p$ crashes while deleting \emph{nd},
it can check the marked bit upon recovery to detect whether \emph{nd} is logically deleted.
However, this information is insufficient, since several processes may try
to delete \emph{nd} concurrently, and exactly one should succeed.
For that, each node is equipped with an extra \emph{deleter} field,
initially $\bot$, used by the deleting processes to agree which process
succeeded to delete \emph{nd}.
The first process that writes its id to \emph{deleter} (with a CAS)
a \emph{nd} is logically deleted, is the one to succeeds in deleting the node,
while all others fail and return false.
Notice that crashed operations are also trying to write to \emph{deleter}
and that this filed is written to exactly once.
A recovering process can check if it succeeded in the delete by checking
if \emph{deleter} holds its id.

As we discuss later, the same technique can be applied to any implementation
with the following features:
An operation that takes effect using a single CAS,
and needs to guarantee exactly one process is the one to perform it,
can use a \emph{new} field (as in delete) to determine which process executed it;
this relies on the ability to determine, upon recover, whether the CAS was successful.
Additionally, an \emph{insert} to a data structure in which a new,
unique object is used whenever a new key is added (in a node),
can be recovered by looking for the new object,
or for an indication it has been removed from the data structure.
One such implementation is the \emph{queue}~\cite{MichaelS-PODC1996}.
% but we do not discuss this implementation but two others, more complicated

}


\subsection*{Binary Search Tree}

In the lock-free \emph{binary search tree} (BSF)
of~\cite{DBLP:conf/podc/EllenFRB10},
processes help each other to carry their operations.
The implementation, described in detail in Section~\ref{section:BSF}.
each operation has an associated info record,
initialized in a single CAS by the calling process,
and using by helping processes to track their progress.

When a process $p$ needs to modify a node $nd$ it marks $nd$ with
a pointer to its info record; this marking remains until $p$'s operation
completes (either by $p$ or by a helping process).
This implies that upon recovery after a crash,
$p$ can check to see if the node is still marked with its own operation,
and if so, try to complete the operation, starting from right after the marking CAS;
otherwise, $p$ restarts the operation.

As is often the case when helping is used,
several processes may try to perform the same operation (concurrently),
leading to redundant efforts that yield only a single completed operation.
Even more problematic in the context of the crash-recovery model,
and because helping is anonymous,
repeated attempts of $p$ to perform an operation (due to crashes)
may be indistinguishable from an execution where other processes are helping $p$
complete its operation.

The above recovery scheme may still allow a scenario in which $p$ crashes
right after completing its operation and before unmarking the node,
causing $p$ to restart the operation upon recovery and applying it twice.
To avoid this scenario, a process updates a \emph{done} field in
the info record, which signifies the operation is done,
before unmarking the node in the cleaning phase.
Thus, if the node is not marked with the operation upon recovery,
then the new field allows the process to conclude whether
the operation has been completed or failed.
If a process crashes after updating the done field and before cleaning,
then upon recovery, unless another process performed the cleaning,
it will observe that the node is marked and complete the cleaning phase.

%%% this may belong in the elimination stack section
\subsection*{Elimination Stack}

After presenting these specific data structures, linked list and BST,
we can explain the two principles underlying our general approach,
in the context of a more complex data structure---an
\emph{elimination stack}~\cite{DBLP:journals/jpdc/HendlerSY10}.

An elimination stack combines of two components:
a central stack and an elimination array.
%%% Explain the implementation more carefully.
The central stack is implemented in a way that resembles Harris'
linked list, described above:
to push or pop the process tries to atomically swing the head pointer using CAS.
Therefore, the same approach taken in the Linked-List can be used for this case also.
The Pop operation will use a new $popby$ field, similar to the deleter field,
in order to determine which process is the one to pop the node.

However, unlike the linked list,
where a node is first marked before removing it from the list,
in the stack a Pop operation is done using a single CAS without marking.
Hence, if a process $p$ crashes right after successfully pushing a new node
and the node is later removed from the [[stack??]], $p$ can not tell
between the resulting situation and one in which the Push has failed.

This problem is resolved by having a process mark a node [[as part of the list??]]
before trying to pop it; marking does not mean that the node has been removed,
as the Pop operation may fail.
%%% the following seems too technical and an optimization.
For efficiency, the $popby$ field is used also for this marking.
A process first writes NULL to it, and in case of a successful Pop try to write its id.
This is all done using CAS, so to avoid overwriting.

In the elimination array each entry holds an \emph{exchanger} object that matches
two processes---one doing a push and another doing a pop---to exchange their values.
We change the implementation such that a process first creates an info
record containing its values, and then processes exchange info records
instead of values:
The process doing a push writes its info record by a CAS on an exchanger,
while the process doing a pop takes this it by replacing it, with a CAS,
with its own info record.

To allow recovery, the info record includes s a state, identifying whether
$p$ is the first or second process to write to an exchanger,
i.e., whether it was performing a push or a pop;
in the latter case, the record also contains a reference to the info record $p$
is trying to collide with.
If $p$ crashes and the info record indicates it is part of an ongoing
exchange, then $p$ tries to complete the exchange;
otherwise, either its exchange attempt failed,
or that it succeeded and completed by some other process.
The latter case is indicated in the info record, which also contains the
response value.

The info records also support a simple and efficient helping mechanism.
After a process writes its info record second to the exchanger,
any other process complete the exchange by reading this record and
updating both records with the right response values.
That is, other processes do not wait for the two colliding processes
and complete the exchange on their behalf.
Then, they can reset the exchanger, making it ready for reuse.


%\input{Definitions&Model-PODC18-relaxed}

\section{Model and Definitions}
\label{section: Model}

%\subsection{Standard Shared-Memory Model}

We consider a system where $N$ asynchronous \textit{processes} $p_1, \ldots, p_{N}$
communicate by applying operations to \emph{concurrent objects}. The system provides
\textit{base objects} that support atomic read, write, and read-modify-write \danny{primitive operations}.
%\y{We call these operations, {\em primitive operations}.}
%No bound is assumed on the size of a shared variable (i.e., the number of distinct values it can take).
Base objects can be used for {\em implementing} more complex concurrent objects
%(e.g. counters, queues and stacks,
by defining \y{algorithms, for every process,} that implement
the operations of the implemented object using \danny{primitive operations}.
\y{These more complex objects (together with base objects) }
may be used in turn, similarly, for implementing even more complex objects,
and so on.

The state of each process consists of {\em non-volatile} shared-memory variables,
as well as \emph{local} variables stored in volatile processor registers.
At any point during the execution of an operation,
a process can incur  a \emph{crash-failure}
(or simply a \emph{crash}) that resets all its local variables to arbitrary values,
but preserves the values of all its non-volatile variables.
A process $p$ \emph{invokes an operation} $Op$ on an object by performing
an \emph{invocation step}. Upon \emph{Op}'s completion, a \emph{response step}
is executed, in which \y{$Op$'s response is stored to a local variable of $p$}.
The response value is lost if $p$ incurs a crash before {\em persisting} it
\y{(i.e. before writing it to a non-volatile variable)}.


Operation $Op$ is \emph{pending} if it was invoked but was not yet completed.
For simplicity, we assume that, at any point in time, each process has at most
a single pending operation \danny{on any single object}\footnote{This assumption
can be removed, but this would require substantial changes to the notions
of sequential executions and linearizability which we chose to avoid in this work.}.
%Therefore, a process may have more than one operations pending at each point in time
%(albeit those operations are each on a different object)..
%
\y{An operation $Op$ is called {\em recoverable} if there is a \emph{recovery function},
denoted $Op.\texttt{Recover}$, associated with it which
is responsible for completing $Op$ upon recovery from a crash.
If all operations of an implementation of an object $O$ are recoverable,
then the implementation is called {\em recoverable}.}
\danny{The execution of operations (and recoverable operations in particular) may be nested, that is, an operation $Op$ can invoke another operation $Op_1$.}
%, that is, an operation $Op_1$ can invoke another
%operation $Op_2$. %Thus, at any point in the execution, \emph{each process may have multiple pending operations}. For simplicity, we assume that, at all times, each process has at most a single pending operation on any one object.\footnote{This assumption can be removed, but this would require substantial changes to the notions of %sequential executions and linearizability which we chose to avoid in this work.}
%The execution of operations (and recoverable operations
%in particular) may be nested, that is, an operation $Op_1$ can invoke another
%operation $Op_2$. %Thus, at any point in the execution, \emph{each process may have multiple pending operations}.
\y{For example, during the execution of a recoverable operation $Op$ on a simulated object $O$ by process $p$,
$p$ may invoke a recoverable operation $Op_1$ on another object $O_1$.} \danny{Consequently, multiple nested invocations of recoverable operations on different objects by process $p$ (in our example, $Op$ and $Op_1$) can be pending at any point in time.}
\y{Following a crash of process $p$, the system} \danny{may} \y{eventually resurrect process $p$
by invoking the recovery function of the (inner-most) operation
that $p$ was executing at the time of the failure. }
The invocation of this recovery function comprises a \emph{recovery step for p}.

%such that if a process crash while executing an operation on the recoverable object, upon recovering the %appropriate \texttt{Recover} function is triggered (by the system), and we require the process to complete its %pending operation before invoking the next one. As we explain later, the completion requirement, although %seems too restrictive, does not rule out an option for the \texttt{Recover} function to abort the pending %operation, as in such case the process can reissue it. Nevertheless, this restriction simplifies the %definition and proofs.

More formally, a \textit{history} $H$ is a sequence of \emph{steps}. There are four types of steps in a history:
\begin{enumerate}
\item an \emph{invocation step}, denoted $(INV, p, O, Op)$,
represents the invocation of operation $Op$ on object $O$ by process $p$;
\item an operation $Op$ can be completed either \emph{directly} or when,
following one or more crashes, \y{the execution of the last instance of
$Op.\texttt{Recover}$ invoked by the system for $p$ is completed.
In either case, a \emph{response step} $s$, denoted $(RES, p, O, Op, ret)$,
represents the completion of operation $Op$ invoked on object $O$ by process $p$.
When $s$ takes effect, the response $ret$ is written to a local variable of $p$.
If $s'$ is the invocation step of $Op$ by $p$,
we say that $s$ {\em matches} $s'$;}
\item a \emph{crash step} $s$ {\em for process $p$}, denoted $(CRASH, p)$,
represents the crash of process $p$. We call the inner-most recoverable operation $Op$ invoked by $p$
that was pending when the crash occurred, the \emph{crashed operation} of $s$;
\danny{$p$ may crash also when executing $Op.\texttt{Recover}$, in which case another $(CRASH, p)$ step is appended to $H$ and $Op$ is the crashed operation of $s$ also in this case.}
\item \danny{a \emph{recovery step $s$ for process p}, denoted $(REC, p)$, is the only step of $p$ that is allowed to follow a $(CRASH, p)$ step $s'$. It represents the resurrection of $p$ by the system, in which it invokes $Op.\texttt{Recover}$,\footnote{A history does not contain invocation/response steps for recovery functions.} where $Op$ is the crashed operation of $s'$.
We say that \emph{s is the recovery step that matches} $s'$.}
\end{enumerate}

When a recovery function $Op.\texttt{Recover}$ is invoked by the system
\y{to recover from a crash, }
we assume it receives the same arguments as those with which $Op$ was invoked when the crash occurred. We also assume the existence of a per-process non-volatile variable $\text{CP}_p$
\danny{that may be used by recoverable operations and recovery code for managing check-points in their execution flow.} The system stores to $\text{CP}_p$ the address
of the first instruction of a recoverable operation $Op$ \danny{when $Op$ is invoked},
and stores to it the address of the next instruction of $Op$ to be executed
when a function call invoked by $Op$ \danny{returns}.
$\text{CP}_p$ can be read
and written by recoverable operations (and their recovery functions)\footnote{This is a relaxation of the model of \cite{ABH-PODC2018}, which assumed that recovery code
has access to the address of $Op$'s instruction that $p$ was about to execute when it crashed. %This assumption implied that the program counter must be persistent.
}.
\y{In what follows, we consider only histories that arise from recoverable implementations.}
%
%

\remove{%%%%%%
\y{Consider a scenario in which $p$ invokes an operation $Op$ on object $O$.
Assume that, while executing $Op$, $p$ invokes
an operation $Op'$ on some object other than $O$.
Assume also that $p$ incurs a crash step $s$
immediately after $Op'$ completes
(regardless of whether $Op'$ completed directly
or through the completion of  $Op'.\texttt{Recover}$).
Let $r$ be the response step of $Op'$.
%an event represented by a response step $r$.
Then, the response value \y{for $Op'$ may be lost, because when $Op'$ completes, it is only guaranteed that its response is stored in a local variable and was not necessarily persisted.}
Moreover, $Op'.\texttt{Recover}$
will not be invoked by the system, since the crashed operation of $s$
is no longer $Op'$ but the operation $Op$ that invoked $Op'$.
However,
$Op$ may not be able to proceed correctly without knowing the response of $Op'$.}
Hence, \y{it is sometimes desirable} to guarantee
that a recoverable operation \y{completes} only after its response value has been persisted.
\danny{Definition~\ref{def:strict-recoverable-op} formalizes this notion.}

\begin{definition}
\label{def:strict-recoverable-op}
\here{YOULA: Please see below two definitions about strictly recoverable operations.
Which one is the one you had in mind?}
\begin{enumerate}
\item
\y{
 Consider a history $H$ and an operation $Op$ for which a response step $(RES,p,O,Op,ret)$
exists in $H$. We say that $Op$ is {\em strictly recoverable},
if by the point in time that the response step occurs,
the response value $ret$ is stored in a designated persistent variable accessed only by process $p$.}

\item
\y{
An operation $Op$ is \textit{strictly recoverable}, if
in every execution $\alpha$ the following holds:
for every instance of $Op$  which incurs
a response step $(RES,p,O,Op,ret)$ in $\alpha$,
%, i.e. $Op$ is completed in $\alpha$
%(either directly or by the completion of $Op.\texttt{Recover}$),
by the point in time that the response step occurs,
the response value $ret$ is stored in a designated persistent variable accessed only by process $p$. }
\end{enumerate}
\here{YOULA: If we go with the second version, we have to define executions (or legal histories
and instead of talking about all executions above, we have to talk about all legal histories). Also, in both versions
it is not clear what "a persistent variable accessed only by process $p$" means (see also relevant comment
earlier).}
\end{definition}
}%%%%%%%%
%We emphasize that the requirement of Definition \ref{def:strict-recoverable-op} does not imply any changes in object semantics, as we do not require that the response value is persisted when $Op$ is linearized.
%In Section \ref{section: Recoverable Base Objects} we present algorithms for several recoverable primitives, as well as a recoverable counter that uses a non-strict recoverable read operation for efficiency reasons.

For a history $H$, we let $H | p$ denote the subhistory of $H$
consisting of all the steps by process $p$.
We let $H | O$ denote the subhistory of $H$
consisting of all \y{the invocation and response steps on object $O$ in $H$},
as well as \danny{any crash step in $H$, by any process $p$, whose crashed operation is an operation on $O$ and the corresponding recover step by $p$ (if it appears in $H$)}
$H$ is \emph{crash-free} if it contains no crash steps (hence also no recovery steps).
\y{Let $H|{<}p,O{>} = (H | O) | p$.}
% denote the subhistory consisting of all the steps
%on $O$ by $p$. %$H$ is \emph{crash-free} if it contains no crash/recovery steps.
A crash-free subhistory $H | O$ is well-formed,
if for every process $p$, $H|{<}p,O{>}$ is a sequence of \y{alternating} matching invocation
and response steps, starting with an invocation step.


%For a history $H$, we let $H | p$ denote the subhistory of $H$ consisting of all the steps by process $p$ in %$H$, and let $H | O$ denote the subhistory consisting of all the %invoke and response
%steps on object $O$ in $H$. These subhistories consist of all the invoke and response steps on object $O$ in %$H$, as well as any crash step in $H$, by any process $p$, whose crashed operation is an operation on $O$ and %the corresponding recovery step by $p$ (if it appears in $H$).%
%and $H|{<}p,O{>}$ denote the subhistory consisting of all the steps on $O$ by $p$. $H$ is \emph{crash-free} if it contains no crash/recovery steps. A crash-free subhistory $H | O$ is well-formed, if for all processes $p$, $H|{<}p,O{>}$ is a sequence of alternative matching invocation and response steps, starting with an invocation step.

%as well as any crash step in $H$, by any process $p$, whose crashed operation is an operation on $O$ and the corresponding recovery step by $p$ (if it appears in $H$).

%A response step is \textit{matching} with respect to an invocation step $s$ by a process $p$ on object $O$ in %a history $H$ if it is the first response step by $p$ on $O$ that follows $s$ in $H$, and it occurs before %$p$'s next invocation (if any) in $H$.

%For history $H$ and process $p$, an \textit{operation} by $p$ in $H$ comprises an invocation step and its matching response, if it exists. An operation is \textit{complete} if it has a matching response step, and \textit{pending} otherwise.
Given two operations $op_1$ and $op_2$ in a history $H$,
we say that $op_1$ \textit{happens before} $op_2$,
denoted by $op_1 <_H op_2$, if \y{$op_1$ has a response step in $H$ that} precedes the invocation step of $op_2$ in $H$.
If neither $op_1 <_H op_2$ nor $op_2 <_H op_1$ holds,
then we say that $op_1$ and $op_2$ are \textit{concurrent} in $H$.
$H | O$ is a \emph{sequential object history}, if it is an alternating \y{sequence}
of invocations and \y{their} matching responses starting with an invocation
\y{(the sequence may end with a pending invocation)}.
The \textit{sequential specification} of an object $O$ is the set of
all possible (legal) sequential histories over $O$.
%$H$ is a \emph{sequential history} if $H | O$ is a sequential object history for all objects $O$.


%As already mentioned, our model allows histories in which a process may have multiple pending operations on different objects, since it explicitly considers implemented operations that invoke other implemented operations. We call such a history a \emph{nested history}. A nested history may not be sequential even if it is a single-process history. We next define notions that allow us to argue about the correctness of nested histories.
A crash-free history $H$ is \emph{well-formed} if: 1) for {\em every} object $O$, $H | O$ is well-formed,
and 2) for {\em every} process $p$, \y{and for every two pairs $\langle i_1,r_1 \rangle$
and $\langle i_2,r_2 \rangle$ of matching invocation/response steps in $H | p$
such that $i_1$ precedes $i_2$ and $i_2$ precedes $r_1$,
then it holds that $r_2$ precedes $r_1$. }
The second requirement guarantees that \y{if process $p$, while executing an operation $Op_1$,}
invokes an operation $Op_2$,
$Op_2$'s response must precede $Op_1$'s response.
\y{In what follows, we consider only well-formed histories.}

%
%\begin{definition}[Nested well-formedness]
%A crash-free history $H$ is \textit{well-formed} if the following holds.
%\begin{enumerate}
%\item For all $O$, $H | O$ is well-formed.
%\item For all $p$, let $i_1,r_1$ and $i_2,r_2$ denote two matching invocation/response steps in $H | p$. If %$i_1 <_H i_2 <_H r_1$ holds, then $r_2 <_H r_1$ holds as well.
%\end{enumerate}%
%
%\end{definition}
\y{Two histories $H$ and $H'$ are \emph{equivalent},
if $H|p=H'|p$ for all processes $p$.}
A history $H$ is \textit{object-sequential}, if $H | O$ is sequential for all objects $O$ that appear in $H$.
%Two histories $H$ and $H'$ are \textit{object-equivalent} if for every process $p$ and object $O$, $H|{<}p,O{>} = H'|{<}p,O{>}$ holds. A history $H$ is \textit{object-sequential} if $H | O$ is sequential for all objects $O$ that appear in $H$. An object-sequential history H is \emph{legal}, if for every object $O$ that appears in $H$, $H|O$ belongs to the sequential specification of $O$.
Given a {\em finite} history $H$, a \textit{completion} of $H$ is a history $H'$ constructed from $H$ by selecting separately,
for each object $O$ that appears in $H$, a subset of the operations pending on $O$ in $H$
and appending matching responses to all these operations,
and then removing all remaining pending operations on $O$ (if any).

\begin{definition} [Linearizability \cite{herlihyWingLinearizability}, rephrased]
\label{Definition: Linearizability}
A finite crash-free history $H$ is \emph{linearizable} \y{if there exists a completion $H'$ of $H$}
and an \danny{object-sequential} history $S$ such that \danny{the following requirements hold}:
%\begin{enumerate}
%	\item [L1.] 
(i) $H'$ is equivalent to $S$,
%    \item [L2.] 
(ii) \danny{$S | O$ is legal for all objects $O$,} and
%	\item [L3.] 
(iii) \y{$<_{H'} \subseteq <_S$ (i.e., if $op_1 <_{H'} op_2$, then $op_1 <_S op_2$)}.
%\end{enumerate}
\end{definition}

Thus, a finite history is linearizable, if we can linearize the subhistory
of each object that appears in it.
Next, we define a more general notion of well-formedness that applies also
to histories that contain crash/recovery steps. For a history $H$,
we let $N(H)$ denote the history obtained from $H$ by removing all crash and recovery steps.

\begin{definition}% [Recoverable Well-Formedness]
\label{def:recoverable-well-formedness}
A history $H$ is \textit{recoverable well-formed} if (i) $N(H)$ is well-formed,
and (ii)
%\begin{enumerate}
%\item
every crash step in $H | p$ is either $p$'s last step in $H$ or is followed in $H | p$ by a matching recovery step of $p$.
%\item 
%\end{enumerate}
\end{definition}

%We can now define the notion of nesting-safe recoverable linearizability.

\begin{definition}% [Nesting-safe Recoverable Linearizability (NRL)]
\label{Definition:NRL}
A finite history $H$ satisfies \emph{nesting-safe recoverable linearizability} (NRL)
if it is recoverable well-formed and $N(H)$ is a linearizable history.
\y{An object implementation satisfies NRL if every history it produces satisfies NRL.}
\end{definition}


%----------------------------------------------------------------e
\remove{%%%%%%%%%%%
is immediately followed by a matching response, or by a crash step, and every response step in $H|p$ is a matching response for a preceded invocation. Informally, $H$ is well-formed if $H|p$ is a sequential history of operations, except for the ones that may not have a response step due to crash steps. The \textit{sequential specification} of an object $O$ is the set of possible sequential histories over $O$. \emph{A sequential history H is legal} if for every object $O$ that appears in $H$, $H|O$ belongs to the sequential specification of $O$.

A concurrent object is called \textit{recoverable} if any of its operations is equipped with a \texttt{Recover} function, such that if a process crash while executing an operation on the recoverable object, upon recovering the appropriate \texttt{Recover} function is triggered (by the system), and we require the process to complete its pending operation before invoking the next one. As we explain later, the completion requirement, although seems too restrictive, does not rule out an option for the \texttt{Recover} function to abort the pending operation, as in such case the process can reissue it. Nevertheless, this restriction simplifies the definition and proofs.

Recoverable object by its own does not consider the response value of the operation. For example, a primitive CAS is a recoverable object (with an empty \texttt{Recover} function), although a process crashing after executing CAS have no access to the response value upon recovery, as it was lost. For this reason we extend the definition of recoverable object, such that in addition to a \texttt{Recover} function it also needs to satisfy the following: every operation returns (i.e., there is a response step in the history) only after the response value is persistent. In a more formal way, a process $p$ have a designated variable $Res_p$ in the non-volatile memory such that at the time of $(RES,p,X,ret)$ step, the value $ret$ is written in $Res_p$.
Notice that the object's semantic does not change, that is, we do not require the response value to be persistent at the linearization point.

In order to formally capture this behaviour we introduce a recovery step, denoted $(RECOVER,p)$, represents the recovery of a process $p$, and the invocation of the \texttt{Recovery} function matching the pending operation, if there is one.
A history $H$ is called \textit{recoverable well-formed} if every crash step by a process $p$ which is not the last step of $p$, is followed by a recovery step of $p$ (and no other steps of $p$ are allowed in between), and vice verse. In addition, every response step follows either the matching invocation step or a recovery step. We care only about such histories, as we assume the system always triggers the proper \texttt{Recovery} function whenever a process is waking-up after a crash.


A recoverable object is in some sense "fail-resistant". If a process crash after completing an operation, then upon recovery the process have an access to the response value residing in the non-volatile memory. On the other hand, if a process crash before the response value is persistent, i.e., before the operation was completed, then upon recovery the \texttt{Recovery} function will complete the operation, together with making the response value persistent. To our knowledge, this is the first definition to consider the affect of a crash on the crashing process, and not only on the object.

One can think of different ways to persist the response value. For example, an operation to a recoverable object gets a location in the shared memory as an extra operand, and the response value is persistent in this location at the response step. This can be implemented easily by replacing $Res_p$ with the supplied location. We do not role out such solutions. However, for ease of presentation the definition uses a simple version.

Notice that any object can be implemented in a recoverable manner, as long as there is no restriction on the correctness condition it requires to satisfy. The \texttt{Recover} function does not allow a process $p$ to invoke a new operation before completing the pending operation, hence in every history $H$ a process have at most a single pending operation which is his last invoked operation. Therefore a natural requirement for such an object is linearizability. Since every operation of a process needs to be complete (except for maybe the last one), and we use the original definition of linearizability, this implies that locality holds under this definition.



The formal definition allows the use of recoverable objects only, as we require the history to be linearizable under the original Herlihy and Wing's definition. However, such objects can be used in a more general context. One may use recoverable objects only in critical parts of the program, as such an object guarantee the ability to recover and complete the operation in case of a crash. In the rest of the program, assuming data lost is less critical, a simpler objects can be used, e.g., implementations that only guarantee recoverable linearizability.
In such a way, the programmer have the flexibility to protect certain parts of the program in the cost of time and space complexity by using the more powerful recoverable objects, while the rest of the program is more efficient but not completely robust to crashes.


Recoverable object by its own does not consider the response value of the operation. For example, a primitive CAS is a recoverable object (with an empty \texttt{Recover} function), although a process crashing after executing CAS have no access to the response value upon recovery, as it was lost. For this reason we extend the definition of recoverable object, such that in addition to a \texttt{Recover} function it also needs to satisfy the following: every operation returns (i.e., there is a response step in the history) only after the response value is persistent. In a more formal way, a process $p$ have a designated variable $Res_p$ in the non-volatile memory such that at the time of $(RES,p,X,ret)$ step, the value $ret$ is written in $Res_p$.
Notice that the object's semantic does not change, that is, we do not require the response value to be persistent at the linearization point.
}%%%%%%%%% END REMOVE



\section{Linked-List Based Set}
\label{section-linked-list}

In this section, we present a recoverable version of the linked-list set algorithm of Harris \cite{DBLP:conf/wdag/Harris01}.\footnote{Some implementation details 
follow the algorithm's presentation in~\cite{DBLP:books/daglib/0020056}.}
 
The linked-list set supports the \find, \insertlst\ and \delete\ operations. 
The algorithm maintains a linked list of nodes sorted in increasing order of keys. 
\y{The list} always \y{contains} a \emph{head} and \emph{tail} sentinel nodes, 
containing keys $-\infty, \infty$, respectively.
The pseudo-code for the algorithm is presented 
in Algorithms \ref{alg:linked-list-part1}-\ref{alg:linked-list-part2}.
Pseudo-code in blue font was added for recoverability. 
the \emph{next} field of \y{each node 
%is a MarkableNodeRef type, consisting of a reference 
consists of a reference} to the next node and a \emph{marked} bit that is set when the node is logically deleted. 
Both components can be manipulated atomically, either together or individually, 
using a singe-word \CAS\ operation\footnote{In the Java implementation of the algorithm 
presented in \cite{DBLP:books/daglib/0020056}, \emph{next} fields are represented 
by AtomicMarkableReference \y{objects. In the implementation of an AtomicMarkableReference object}
the marked bit occupies the least significant bit of the reference.}. 
The \emph{marked} predicate can be applied 
to \y{the next field of a node to determine whether or not the node is marked}.

%\subsection{Detailed Description of the Algorithm}

We now describe an extension of the Harris algorithm 
that allows a process recovering from a crash-failure 
to determine whether \ohad{its failed operation completed, and return the correct response.}
%\y{the realization \CAS\ of its failed operation occured before the failure and return the correct response.}
\y{The recovery function of the \find\ operation simply re-invokes \find\ 
(hence its pseudo-code is not shown).}
To support the recovery of \insertlst\ and \delete\ operations, 
a (persistent) shared-memory array \emph{RD} was added, 
where \emph{RD}[\emph{p}] stores recovery data for process $p$. 
More specifically, variable \emph{RD}[\emph{p}] contains a pointer 
to an \Info\ structure storing recovery data for the process' current recoverable operation. 
Each \Info\ structure stores two fields - a reference \emph{nd} to a node 
and a \emph{result} field used for persisting the response value for the operation before returing.

We now describe the additions to the \insertlst\ operation that were introduced 
for supporting recoverability. First, \insertlst\ installs a fresh \Info\ structure \y{into $RD$}
in line \ref{insert-install-IS}, whose \emph{nd} field points 
to a newly allocated node structure (the \emph{deleter} field is only 
used by \delete\ operations and is described later). 
It then updates $p$'s check-point variable (in line \ref{insert-set-checkpoint}) 
by invoking the curPC macro, returning the current value of the program counter 
\y{(for simplicity, we assume that it is \ref{insert-set-checkpoint}, in this case)}. 
\y{By executing this line, $p$ {\em persistently reports}
that the info structure for its current operation has been installed.}
Finally, once \insertlst\ determines its response, 
it persists it just before returning (in lines \ref{insert-persist-false-response} 
and \ref{insert-persist-true-response}).

We now describe the \insertrecover\ function. Let $W$ denote the instance 
of \insertlst\ from whose failure  \insertrecover\ attempts to recover. 
\insertrecover\ starts by reading $p$'s check-point variable 
in line \ref{insert-recover-read-CP} in order to check 
whether $p$'s current info structure was installed by $W$. 
If the failure occurred before $W$ \y{persistently reported that} 
it installed its info structure 
(in line \ref{insert-set-checkpoint}), then $W$ is re-executed from scratch. 
\y{Otherwise, the following actions take place.}
If a response was already written to $W$'s \Info\ structure, 
then \insertrecover\ simply returns this response 
(lines \ref{insert-recover-response-persisted}-\ref{insert-recover-return-persisted-response}). 
We are left with the case that a response was not yet written by $W$ to its \Info\ structure, 
\y{so either $W$ did not execute a successful realization \CAS\ (in line \ref{insert-CAS}) 
or its realization \CAS\ succeeded 
%(hence $W$ was linearized) 
but $W$ failed before writing 
the response in line \ref{insert-persist-true-response}. }
In order to determine which of these two scenarios occurred, 
\insertrecover\ searches the list for \emph{key} (in line \ref{insert-recover-search}). 
%As we prove, 
\y{If either} the key is found in the list inside the node allocated 
by $W$ or if that node was marked for deletion (line \ref{insert-if-was-linearized}), 
\y{then $W$ had executed its realization \CAS\ before the failure occured
(i.e. $W$ succeeded in inserting its node in the linked list),} 
and so the recovery function persists the response 
and returns \True (lines \ref{insert-linearized-persist-true}-\ref{insert-linearized-return-true}). 
\y{Notice that indeed if the node has been successfully linked in the list,  
the only way for the \search\ not to find it is if it has, in the meantime, been
deleted. However, in that case, the node will have, first, been marked for deletion, and
therefore the condition of the if statement of line~\ref{insert-if-was-linearized} will be evaluated to true.}
Otherwise, recovery proceeds from line \ref{insert-entry}
in order to re-attempt insertion.\footnote{\textcolor{red}{BY ``recovery proceeds from line X'' we do not mean 
to say that there is a jump to line X, since in order to do so execution context 
before line X must be saved and then restored. Instead, we mean that the pseudo-code performed 
by the recovery function from this point on is the same as that of the recoverable \y{operation}
starting from line X. One efficient way of implementing this is to have 
both the \y{recoverable operation} and the recovery function invoke a parameterless macro call 
that embeds this pseudo-code during compilation pre-processing.}}

Next, we describe how recoverability \y{is ensured for} \delete\ operations. 
Similarly to the recoverable \insertlst\ operation, 
a recoverable \delete\ first installs a fresh \Info\ structure \y{into $RD$}
in line \ref{delete-entry}. It then updates $p$'s check-point variable 
(in line \ref{delete-set-checkpoint}) \y{to {\em persistently report} 
that its info structure has been installed.} 
If \emph{key} is not found 
(line \ref{search-in-delete}), the response \y{(which is in this case \False) is persisted 
and then it is returned} (lines \ref{delete-write-false}-\ref{delete-return-false}). 
If \emph{key} \emph{is} found (in node \emph{curr}), 
a reference to \emph{curr} is persisted to the \emph{nd} field of $p$'s \Info\ structure.
Following this, $p$ proceeds as \y{in the original algorithm} by repeatedly trying 
\y{to mark \emph{curr} using \CAS\ in lines \ref{delete-while-loop}-\ref{delete:logical-delete}
(i.e. it repeatedly executes \CAS\ until it logically deletes the node).} 
Once it is marked, $p$  tries to physically remove \emph{curr} 
in lines \ref{delete-read-succ-after-CAS}-\ref{delete-physical-delete}. 

The key technical difficulty for supporting recoverability of \delete\ operations 
is the following. If $p$ fails immediately before or after the \CAS\ 
of line \ref{delete:logical-delete}, recovers and then finds that \emph{curr}
was logically deleted, how can it know whether it was the (single) process 
that succeeded \y{in deleting \emph{curr}}? Our solution to this problem is 
to ``attribute'' a node's deletion to a single process using a new node-field 
called \emph{deleter}, initialized to $\bot$. After $p$ finds that \emph{curr} 
is logically deleted (marked) in line \ref{delete-while-loop}, 
\emph{regardless of whether it was marked by $p$ or by another process}, 
$p$ tries to establish itself as the deleter of \emph{curr} by atomically 
changing \emph{curr.deleter} from $\bot$ to its ID using \CAS\ 
(line \ref{delete-CAS-deleter}).
%This way, if $p$ crashes during a \delete\ operation, it can use the \emph{deleter} field in order to determine its response value.
Finally, $p$ persists and returns the result of this \CAS\ 
\y{as the response of the \delete\ operation}
in lines \ref{delete-write-after-CAS}-\ref{delete:return-res}.

We now describe the \deleterecover\ function. Let $D$ denote the instance of \delete\ 
from whose failure  \deleterecover\ attempts to recover. 
\deleterecover\ starts by reading $p$'s check-point variable 
in line \ref{delete-recover-read-CP}. If the failure occurred 
before $D$ \y{persistently reported that} it installed its info structure (in line \ref{delete-set-checkpoint}), 
then $D$ is re-executed from scratch. 
\y{Otherwise, the following actions take place.}
If a response was already written 
to $D$'s \Info\ structure, then \delete\ simply returns this response 
(lines \ref{delete-recover-response-persisted}-\ref{delete-recover-return-persisted-response}).
Otherwise, if the \emph{nd} field of $p$'s \Info\ structure was previously set 
and node $nd$ is logically deleted (line \ref{delete-recover-if-should-perform-deleter-CAS}), 
$p$ attempts to establish itself as the deleter of $nd$ using \CAS,
\ohad{and then persists and returns the result according to the id written in \emph{curr.deleter}} 
%and then persists and returns the result of that \CAS\ 
(lines \ref{delete-recover-deleter-CAS}-\ref{delete-recover-return-res})
\y{as the response of its \delete\ operation.} 
Finally, if the condition of line \ref{delete-recover-if-should-perform-deleter-CAS} 
does not hold, $p$ re-attempts the deletion (line \ref{delete-recover-reattempt-deletion}).



\subsection{Correctness Argument}

In the following, we give a high-level argument for the correctness of the algorithm.
We say that a node is in the linked-list if it is reachable
by following $next$ pointers starting from $head$.
We say that a node is in the implemented set if it is in the linked-list and is not marked.
%For simplicity of presentation, we sometimes use $curr$ (or $newnd$) 
%to refer to the node pointed to by it.} 

The proof relies on the following observations.
As long as a node $nd$ is in the linked-list there is exactly one node in the list pointing to it.
In addition, $nd$ can be marked exactly once, and it stays so forever, as any \CAS\ is executed with an unmarked node as its first argument.
For the same reason, $nd$ can be physically deleted exactly once, since no node in the list points to it once it is deleted, and we never add a marked node back to the list.

\find\ operation is implemented in a read-only manner, and thus it can be re-executed without effecting any other concurrent operation.
In addition, \search\ routine, even though not read-only, simply traverses the list 
while trying to physically delete any marked node it encounters. As any marked node can be physically deleted exactly once, re-executing \search\ can not cause a deletion of an unmarked node, or a deletion of the same node more then once. This implies re-executing \search\ can only help physical deletion of more nodes, and does not effect the list in any other way.

\insertlst\ and \delete\ first install info structure and set a check-point. Clearly, a crash before the check-point implies the operation did not effect the list or any other operation, and in such case the \recover\ function simply re-execute the operation. This argument holds for any number of crashes, as long as the check-point is yet to be set.
%We are left with the case of a crash after the check-point. We prove it for each operation.

Assume process $p$ performs an $\insertlst(key)$ operation. If $p$ does not crash, then it repeatedly search for the right location for the new node $newnd$, 
and tries to insert it.
In case $p$ updates $RD[p].result$ to \False\ in line \ref{insert-persist-false-response}, then the preceding \search\ in line \ref{insert-search} finds a node $curr$ in the set with data $key$ (when \search\ read $curr$ it is not marked). Therefore, there is a point along the \insertlst\ operation where the set contains $key$, and the operation is linearized at this point. If $p$ crash after updating $result$ then eventually, in order to complete \insertrecover\ $p$ must read $RD[p].result$ and return \False.

In case $p$ performs its realization \CAS\ in line \ref{insert-CAS}, then $newnd$ is in the set. The only way to physically remove it from the list is by first marking it. Therefore, after the realization \CAS, either $newnd$ is in the list, or it is marked, and one of these conditions must hold. As a result, in any crash after the realization \CAS\ the \insertrecover\ function returns \True, either in line \ref{insert-recover-return-persisted-response} (because $result$ has already been updated), or in line \ref{insert-if-was-linearized} which evaluates to \True. In particular, each \insertlst\ operation can perform at most a single realization \CAS, as any crash after it results a response of \True, without performing any more \CAS\ instances.

We are left with the case where $p$ crash before performing its realization \CAS\ or updating $result$. In such case, $newnd$ was not added to the list yet, and in particular no process can mark it. As a result, \insertlst\ did not effect any other operation, and indeed \insertrecover\ re-executes it.

Assume process $p$ performs $\delete(key)$ operation. If $p$ updates $result$ to \False\ in line \ref{delete-write-false}, then the preceding \search\ found two adjustments nodes $pred$ and $curr$ in the list, where $pred.key < key < curr.key$. Since we keep the list sorter in an increasing manner, it follows the list does not contains $key$ at the point when $pred$ points to $curr$. In particular, there is point along the interval of \delete\ where $key$ is not in the set, and the operation is linearized at this point. Any crash after line \ref{delete-write-false} results the \deleterecover\ function must read $result$ and return \False.

Assume now $p$ does not write in line \ref{delete-write-false}. A node $curr$ is written to $RD[p].nd$ in line \ref{delete-update-nd} only if \search\ observes $curr$ is in the list and not marked. Clearly, a crash before updating $RD[p].nd$ implies the operation did not mark any node, nor effected any other operation, and \deleterecover\ simply re-execute \delete. On the other hand, once $p$ updates $RD[p].nd$, it keeps trying to mark $nd$.
If $p$ crash and recovers, and finds $nd$ is not marked (line \ref{delete-recover-if-should-perform-deleter-CAS}), then in particular $p$'s operation did not mark $nd$, nor effected the list or any other operation. In such case, \deleterecover\ re-executes \delete. This argument holds for any number of crash and recover, as long as $p$ observe $nd$ is not marked.

If $p$ observes $nd$ is marked, either in the \delete\ or in \deleterecover, we conclude the marking was done after the read of $nd$ in \search. Moreover, once $nd$ is marked, in order to complete the \delete\ operation, $p$ must performs \CAS, trying to write its name to $nd.deleter$, either in \delete, or in \deleterecover. Since $deleter$ is initialised to \init, only the first such \CAS\ succeeds.
Let $q$ be the first process to perform such \CAS, if exists. Then, it set $nd.deleter$ to $q$, and this does not change. Once $deleter$ is set, if $p=q$, then it can only return \True, even in case it crash, as line \ref{delete-recover-read-deleter} always evaluates to \True. Otherwise, $p \neq q$, and by similar argument it can only return \False.
As a result, any process trying to delete $nd$ and observe it is marked (at any point), must have the marking step in its \delete\ operation interval. In addition, exactly one such process, denoted $q$, returns \True, while any other returns \False. We linearize the \delete\ operation of $q$ at the time of the marking, and any other \delete\ returning \False\ is linearized right after it (in an arbitrary order).







\remove{
We will argue each $\insertlst(key)$ can add at most a single node to the list. Moreover, it returns \False\ only if there is a point in time along its interval where there is a node in the set with data $key$.
Similarly, $\delete(key)$ can mark at most a single node with data $key$. In case it returns \False\ there is a point in time along its interval where there is no node in the set with data $key$. In addition, we show no two process can return \True\ when trying to delete the same node.



We will argue that


The proof relies on the following arguments. For a node $nd$ that was added to the list, at each point of the execution, if $nd$ has not been physically deleted then it is reachable. Moreover, there is a single node in the list pointing to $nd$. 

\find\ operation is implemented in a read-only manner, and thus does not effect any other concurrent operation. As a result, re-executing the operation in case of a crash does not effect any other process. 

The proof is based on the following argument. Each node can be added (marked) exactly once, and this is attribute to a single \insertlst\ (\delete) operation. Moreover, if \insertlst\ (\delete) with input $key$ returns \False, there is a point along the interval execution where there is (there is no) node in the set with $key$.

Each \insertlst(key) operation can add at most a single node, and each \delete(key) operation can logically delete at most a single node. Moreover, in case no node has been added (logically deleted), there is a point it time along the operation interval where key is in the set (not in the set). In addition, once a node is marked it is considered as not in the list, and no future \find\ operation 

We will show that each \find(key)\ operation either adds a new node exactly once and return \True, or returns \False, in which case there is a point along its interval where $key$ is the the set. 

\find\ operation is implemented in a read-only manner, and thus does not effect any other concurrent operation. As a result, re-executing the operation in case of a crash does not 
satisfies NRL, and the linearization point is set in the interval of the last complete execution of the operation.
The \search\ procedure simply \y{traverses} the list, 
while trying to physically delete marked nodes.
Once $curr.next$ is marked, \y{just one} process can perform the physical \y{deletion}. 
This follows from the fact that, at any point in time, there is a single node 
in the list which points to $curr$. Once curr is physically deleted,
no node in the list points to curr, and thus any \CAS\ operation with curr 
as the first parameter will fail. This observation relays on the fact 
that any new allocated node has a different address than $curr$. 
As a result, repeating the attempt to physically delete a node effect the list at most once.

\here{YOULA: It is unclear what we mean when we say "the list consistency".
We usually define the consistency of a data structure based on linearizability,
but linearizability is defined for a history. So, we probably need to fix a configuration,
define the history up to that configuration, define what we mean when we say "data structure"
for this configuration, and then define what we mean when we say "consistent data
strucutre in that configuration. Even then, it is not clear what we mean when we say 
that the data structure is consistent during the execution of a search 
because that the way I can think for this to make sense is that it is consistent 
in all configurations that the search is active.  Based on what I wrote above, 
I removed the reference to the consistency of the list.}

Assume a process $p$ performs an $\insertlst(key)$ operation.
First, $p$ writes to $RD[p]$ and set a check-point, reporting it is about to perform an \insertlst. A crash before setting the check-point implies the operation did not effect the linked-list yet, and \insertrecover\ re-execute the operation.
Once $p$ set the check-point it repeatedly search for the right location for the new node, 
and tries to insert it by performing a \CAS\, unless $key$ is already in the list (As in the original algorithm).
As long as $p$ does not performs its realization \CAS\ $newnd$ is not in the list, and therefore the operation is not visible to any other process. Hence, repeating the operation in such case does not violate linearizability. In addition, once $p$ updates $RD[p].result$ with a response we conclude the operation already had a linearization point (as in the original algorithm), and indeed, in this case \insertrecover\ returns the response stored in $RD[p].result$.

We are left with 
If $p$ does not crash, then, as in the original algorithm, 
it repeatedly tries to find the right location for the new node, 
and insert it by performing a \CAS\.
%changing $pred.next$ to point to $newnd$. 
In addition, a crash after updating $RD[p].result$ 
is after the operation has performed its realization \CAS,
%had its linearization point, 
and \insertrecover\ procedure will return 
the right response. \y{It remains to consider a crash that occurs before the} update to $RD[p].result$. 
There are two scenarios to consider.

Assume $p$ \y{crashes} without performing a successful \CAS\ in line \ref{insert-CAS}. 
Notice that $p$ is the only process to have a reference to $newnd$, 
and it is yet to update any node with this reference, 
and thus no linked-list node points to $newnd$. 
As a result, the operation did not \y{affect} any other process, 
nor it will be in the future. 
Hence, considering the operation as not having a linearization point 
does not violate the list consistency. Indeed, since no node points to $newnd$, 
upon recovery $p$ will see that $newnd$ is not in the list and also not marked, 
and thus will return FAIL. Notice this argument holds whether key is already in the tree, 
or not, as the operation in both cases did not affect the system.
\here{YOULA: I do not agree with the argument above. I think that most of the 
statements in this paragraph are wrong. What does it mean "it will return FAIL"?
According to the pseudocode, this is never the case (all operations return either true or false).
If the key is not in the list, the operation is repeated, no? So, eventually it will add 
the new key in the list and it will affect other operations.
Also, notice that in either case, if the operation is completed, it is linearized. 
So, I suggest to avoid talking about linearization in the discussion above.}

Assume now $p$ \y{crashes after successfully performing its realization} \CAS\ in line \ref{insert-CAS}.
\y{Right after the \CAS, $newnd$ is in the implemented set, as $pred.next$ points to it
and $newnd$ is initialized to be unmarked.}
Also notice \y{that the successful execution of the realization \CAS\ for $p$'s \insertlst\
does not cause the deletion of} any other node: $pred.next$ pointed to $curr$, 
and after the \CAS\ it points to $newnd$ which points to $curr$. 
\y{
We argue that if a node is in the implemented set at some point in time 
and it is not at some later point in time, then at all times after
the point that it is not in the set anymore, the node is marked.
The above argument relies on the fact that a marked node can not be unmarked, 
and that an Insert and Delete can not mistakenly remove nodes from the list. 
Therefore, if $newnode$ is no longer in the list, it must be marked.
As a result, when $p$ executes the Recover procedure, either it will see $newnd$ in the list, 
or it will find it marked. In either case, the condition of the if statement of line~\ref{???}
\here{YOULA: Ohad can you please fill in the reference here?} 
will be evaluated to true and the response of the operation will be true.}
%that it is no longer part of the list, and it must be some other process deleted it, 
%and hence $newnd.next$ is marked. In any case, $p$ will return true as required.

Assume a process $p$ performs a $delete(key)$ operation. First, $p$ writes to RD[p], 
\y{reporting that} it is about to perform a Delete. As before, a crash after writing to RD[p].result 
will return the right response. Also, a \y{crash} before updating any of RD[p].result 
or RD[p].nd implies $p$ is yet to try and mark any node, 
and thus the operation did not affect the system so far, 
nor it will be in the future. Therefore, we can consider the operation 
as not having a linearization point (even in case key is not the list), 
and indeed, the Recover procedure returns FAIL in such case.
\here{YOULA: I do not agree with the argument above for the same reasons 
as those I mentioned previously for \insertlst. }

Assume thus $p$ \y{wrote to $RD[p].nd$ before it crashes}. 
It follows that $p$ completed \y{\search\ } 
and found a node $curr$ storing key. \y{\search\ } guarantees  that there is a point 
in time \y{(during its execution)} where $curr$ is in the list and $curr.next$ is not marked. 
\y{If $p$ recovers and observes} that $curr$ is unmarked, then in returns FAIL. 
\here{YOULA: I do not understand. It never returns FAIL. Has this been left from some previous version?}
Since a marked node can not be unmarked, as there is no \CAS\ changing a marked node, 
it follows that $p$ did not marked $curr$. \here{Therefore, the operation did not affect any process, 
nor it will be, and we consider it as having no linearization point. YOULA: I do not follow:
depending on whether $p$ will manage to record itself as the deleter, it will 
return true or false (in either case it will be linearized).}
Otherwise, \y{the recovery function observes} $curr$ as marked, and we can conclude 
the marking point of $curr$ is along the delete operation. We now prove 
we can linearize the operation, according to its response.
\here{YOULA: I do not follow the last sentence.}

Let $q$ be the process to mark $curr$. Since once $curr.next$ is marked \y{it will never change}, 
the reference of $curr.next$ is fixed to $succ$ \y{(i.e. to the value that variable $succ$ in the
\delete\ instance of $q$ had at the point that $q$ performed the marking of $curr$). }
This also implies $q$ is unique and well defined, and any future \CAS\ on $curr.next$ will fail. 
As a result, any process leaving the while loop in line \ref{delete-while-loop} reads the same value 
in line \ref{delete-read-succ-after-CAS}, which is this $succ$. The attempt to physically delete $curr$ 
in line \ref{delete-physical-delete} will succeed only if $pred.next$ points to $curr$, 
and as we said, \y{$curr.next$} points to $succ$, and any other attempt will fail. 
Thus, if this attempt succeeds, it deletes only $curr$, and can not delete additional nodes.

In line \ref{delete-CAS-deleter} process $p$ tries to write its id to $curr.deleter$. 
\y{As it is initialized to $\bot$}, only the first process to perform this \CAS\ will succeed. 
Also, \y{every} $p$ must go through line \ref{delete-CAS-deleter} in order to complete its operation, 
\here{as the recovery function \y{redirects} the process to this line. YOULA: Not accurate: 
the recovery function contains a similar CAS in line 106. It "redirects" the process (if I understand
correctly what redirects mean) only if line 110 is executed.}
\y{Therefore, if there exists a process 
that has completed its delete operation on $curr$, 
there must be a successful \CAS\ to $curr.deleter$.} Let $q'$ be the first process to perform this \CAS. 
As proved above, $q'$ tries to delete $curr$, and the point in time 
where $curr$ is marked must be contained in its operation interval. 
Moreover, $q'$ is the only process to write to $curr.deleter$, 
and the first one to do so, thus $q'$ is the only process to obtain true when testing 
($curr.deleter = q'$) in line \ref{delete-write-after-CAS} (and thus to also return true), 
while any other process will \y{return} false. We linearize the operation of $q'$ at the point 
of the marking, and any other attempt to delete $curr$ is linearized after it (in an arbitrary order).

A corollary of the analysis is that processes trying to delete the same node $curr$, 
all keep trying to mark $curr$ (and therefore it is like if they help each other). 
However, the process that successfully performs the marking 
is not necessarily the one to return true. Also, in the original algorithm, 
if a process fails to mark a node, it starts the delete operation from the beginning. 
In our implementation, the process can keep trying to mark the node without the need 
to perform a \y{\search\ } again after each failed \CAS. We guarantee that once $curr$ is marked, 
exactly one process will return true, while the rest can consider $curr$ as being deleted 
(in the course of their delete execution), and thus there is a point along their execution 
at which $key$ is not in the tree, and they can return false.


\here{YOULA: I think it would be nice if we add 1-2 paragraphs to explain why nothing goes wrong 
in case a failure occurs while $p$ executes the recovery function of the \insertlst\ or
 the \delete.}


}






\begin{figure}[!t]
	\removelatexerror

\begin{algorithm}[H]

	\footnotesize
	
	\begin{flushleft}	
		\textbf{Type} Node \{MarkableNodeRef $*next$, int \emph{key}, \textcolor{blue} {int \emph{deleter}}\} \\
		\textcolor{blue} {\textbf{Type} \Info\ \{Node $*nd$, boolean $result$\}}\\
	
		\textbf{Shared variables:} Node $*head$, \textcolor{blue} {$\Info *$ RD[N]} \\	
	\end{flushleft}


%%%%%%%%%%%%%% Recoverable Linked List - Insert %%%%%%%%%%%%%%%%%%%%%%%%%

	\begin{procedure}[H]
		\caption{() boolean \insertlst\ (T $key$)}
		
			Node $*pred, *curr$ \label{insert-entry}\;
			Node $newnd$ := \textbf{new} Node ($key$)\;
			\textcolor{blue}{newnd.deleter := \init} \;
			\textcolor{blue} {$RD[pid]$ := \textbf{new} \Info\ ($newnd$, \init) \tcp*{Install Info structure for this operation} \label{insert-install-IS}}
	        \textcolor{blue} {$\text{CP}_p$ := curPC() \tcp*{Set a check-point indicating \Info\ structure was installed}\label{insert-set-checkpoint}}
			\While{\True}{ \label{insert-while}
				\tcp{Search for right location to insert}
				$\langle pred, curr \rangle := \search(key)$ \label{insert-search} \;
				\uIf (\tcp*[f]{$key$ in the list}) {$curr.key = key$ \label{insert-if-in-list}} {
					\textcolor{blue} {$RD[pid].result := \False$ \tcp*{Persist response} \label{insert-persist-false-response}}
					\KwRet \False \label{insert-return-false}
				} \uElse {
					$newnd.next$ := $<0,curr>$\;
					\tcp{Try to add $newnd$}
					\uIf {$pred.next.\CAS$ ($<0,curr>$, $<0,newnd>$) \label{insert-CAS}} {
						\textcolor{blue} {$RD[pid].result := \True$ \tcp*{Persist response} \label{insert-persist-true-response}}
						\KwRet \True
					}
				}		
			}
	\end{procedure}


%%%%%%%%%%%%% Recoverable Linked List - Insert Recover %%%%%%%%%%%%%%%%%%%%%

	\textcolor{blue}{
	\begin{procedure}[H]
		\caption{() boolean \insertrecover\ (T $key$)}
		%
		Node $*nd := RD[pid].nd$\;
		\tcp{Failed before installing info structure}
		\uIf {$\text{CP}_p$ $<$ \ref{insert-set-checkpoint} \label{insert-recover-read-CP}}
		{Proceed from line \ref{insert-entry} \tcp*{re-execute}}
		\tcp{If operation response was persisted}
		\uElseIf {$RD[pid].result \neq \init$ \label{insert-recover-response-persisted}}
		{\KwRet $RD[pid].result$ \tcp*[f]{Return response \label{insert-recover-return-persisted-response}}}
		$\langle pred, curr \rangle := \search (key)$ \tcp*{Search for $nd$ in the list} \label{insert-recover-search}
		\tcp{If $nd$ in list or is marked (hence was)}
		\uIf {$curr = nd \mid \mid marked(nd.next)$ \label{insert-if-was-linearized}} {
			$RD[pid].result := \True$ \tcp*{Persist response} \label{insert-linearized-persist-true}
			\KwRet \True \label{insert-linearized-return-true}\;
		}
		\uElse
		{
			Proceed from line \ref{insert-entry} \tcp*[f]{Re-attempt insertion}
		}
	\end{procedure}
	}

\caption{Recoverable linked list: \insertlst} \label{alg:linked-list-part1}
\end{algorithm}

\end{figure}




%%%%%%%%%%%%%%Recoverable Linked List - Delete%%%%%%%%%%%%%%%%%%%%%%%%%%%%

\begin{figure}[!t]
	\removelatexerror

\begin{algorithm}[H]

\footnotesize

\begin{flushleft}	
\end{flushleft}


%%%%%%%%%%%%% Recoverable Linked List - Delete %%%%%%%%%%%%%%%%%%%%%%%%%

\begin{procedure}[H]
	\caption{() boolean \delete\ (T $key$)}

		 Node $*pred, *curr, *succ$ $\quad$ boolean $res$ := \False \label{delete-entry}\;
		\textcolor{blue} {$RD[pid]$ := \textbf{new} \Info\ (\init, \init) \tcp*{Install info structure for this operation}}
        \textcolor{blue} {$\text{CP}_p$ := curPC() \tcp*{Set a check-point indicating \Info\ structure was installed}\label{delete-set-checkpoint}}
		%\While{\True}{ \label{recover-delete}
			\tcp{Search for $key$ in the list}
			$\langle pred, curr \rangle := \search (key)$ \label{search-in-delete} \;
			\uIf (\tcp*[f]{$key$ not in the list}) {$curr.key \neq key$ \label{delete-key-in-list}} {
				\textcolor{blue} {$RD[pid].result := \False$ \tcp*{Persist response}} \label{delete-write-false}
				\KwRet \False \label{delete-return-false}
			}
			\uElse {
				\textcolor{blue}{$RD[pid].nd := curr$ \tcp*{Persist reference to node containing key}} \label{delete-update-nd}
				\While (\tcp*[f]{Repeatedly attempt logical delete}) {$\lnot marked(curr.next)$  \label{delete-while-loop}}{
					$succ := curr.next$ \label{delete:set-succ-to-next}\;
					$res := curr.next.\CAS$ ($<0,succ>$, $<1,succ>$) \label{delete:logical-delete}\;
				}
				\tcp{Physical deletion attempt}
				$succ := curr.next$ \label{delete-read-succ-after-CAS}\;
				$pred.next.\CAS$ ($<0,curr>$, $<0,succ>$) \label{delete-physical-delete} \;
				\textcolor{blue}{\tcp{Try establishing yourself as deleter}}
				\textcolor{blue}{$res := curr.deleter.\CAS (\init, pid)$}  \label{delete-CAS-deleter}
				\textcolor{blue}{$RD[pid].result := res$ \tcp*{Persist response}} \label{delete-write-after-CAS}
				\KwRet $res$ \label{delete:return-res}
				}

\end{procedure}


%%%%%%%%%%%%% Recoverable Linked List - Delete Recover %%%%%%%%%%%%%%%%%%%%%

\textcolor{blue}{
\begin{procedure}[H]
	\caption{() boolean \deleterecover\ (T $key$)}
	%
	Node $*nd := RD[pid].nd$ , $\quad$ boolean $res$:=\False\;
	\tcp{Failed before installing info structure}
    \uIf {$\text{CP}_p$ $<$ \ref{delete-set-checkpoint} \label{delete-recover-read-CP}}
        {Proceed from line \ref{delete-entry} \tcp*{re-execute}}
    \tcp{If operation response was persisted}
 	\uElseIf {$RD[pid].result \neq \init$ \label{delete-recover-response-persisted}}
		{\KwRet $RD[pid].result$ \tcp*[f]{Return response \label{delete-recover-return-persisted-response}}}
		\tcp{If $nd$ was logically deleted}
		\uIf {$nd \neq \init$ \&\& $marked(nd.next)$ \label{delete-recover-if-should-perform-deleter-CAS}} {
			\tcp{Try establishing yourself as deleter}
			$nd.deleter.\CAS\ (\init, pid)$ \label{delete-recover-deleter-CAS} \;
			$res := (nd.deleter == pid)$ \label{delete-recover-read-deleter} \;
			$RD[pid].result := res$ \tcp*{Persist response}
			\KwRet res \label{delete-recover-return-res}\;
		}
        \uElse {
		Proceed from line \ref{delete-entry} \label{delete-recover-reattempt-deletion} \tcp*[f]{Re-attempt deletion}
        }
\end{procedure}
}

\caption{Recoverable linked list: \delete} \label{alg:linked-list-part2}
\end{algorithm}

\end{figure}




\section{Robust BST}
\label{section:BST}

The original BST algorithm does not support the crash-recovery model. It is clear from the code a process does not persist the operation's response in the non-volatile memory, and thus, once a process crash the response is lost. For example, assume a process $q$ apply $\func{Insert} (k)$, performs a successful \CASB\ in line \lref{iflag-cas} and fails after completing the \func{HelpInsert} routine. In this case, the \func{Insert} operation took effect, that is, the new key appears as a leaf in the tree, and any $\func{Find} (k)$ operation will return it. However, even though the operation must be linearized before the crash, upon recovery process $q$ is unaware of it. Moreover, looking for the new leaf in the tree may be futile, as it might be $k$ has been removed from the tree after the crash.

Furthermore, if no recover routine is supplied, it may result an execution which is not well-formed. Consider for example the following scenario. A process $q$ invoke an $Op_1 = \func{Insert}(k_1)$ operation. $q$ performs a successful \CASB\ in line \lref{iflag-cas} followed by a crush. After recovering, $q$ invoke an $Op_2 = \func{Insert}(k_2)$ operation. Assume $k_1$ and $k_2$ belongs to a different parts of the tree (do not share parent or grandparent). Then, $q$ can complete the insertion of $k_2$ without having any affect on $k_1$. Now, a process $q'$ performs $\func{Find}(k_1)$ which returns \NULL, as the insertion of $k_1$ is not completed, followed by $\func{Find}(k_2)$, which returns the leaf of $k_2$. The $\func{Insert}(k_1)$ operation will be completed later by any \func{Insert} or \func{Delete} operation which needs to make changes to the flagged node. We get that $Op_2$ must be before $Op_1$ in the linearization, although $Op_1$ invoked first.

The kind of anomaly described above can be addressed by having the first \CASB\ of a successful attempt for \func{Insert} or \func{Delete} as the linearization point, as in the Linked-List. For that, the \func{Find} routine should take into consideration future unavoidable changes, for example, a node flagged with \insertflag\ ensures an insertion of some key. A simple solution is to change the \func{Find} routine such that it also helps other operations, as described in figure \ref{robust find - solution 1}. The \func{Find} routine will search for key $k$ in the tree. If the \func{Search} routine returns a grandparent or a parent that is flagged, then it might be that an insert or delete of $k$ is currently in progress, thus we first help the operation to complete, and then search for $k$ again. Otherwise, if $gpupdate$ or $pupdate$ has been changed since the last read, it means some change already took affect, and there is a need to search for $k$ again. If none of the above holds, there is a point in time where $gp$ points to $p$ which points to $l$, and there is no attempt to change this part of the tree. As a result, if $k$ is in the tree at this point, it must be in $l$, and the find can return safely.

The approach described above is not efficient in terms of time. We would like a solution which maintain the desirable behaviour of the original \func{Find} routine, where a single \func{Search} is needed. A more refined solution is given in figure \ref{robust find - solution 2}. The intuition for it is drown from the Linked-List algorithm.
In the Linked-List algorithm it was enough to consider a marked node as if it has been deleted, without the need to complete the deletion. Nonetheless, the complex BST implementation is more challenging, as the \func{Delete} routine needs to successfully capture two nodes using \CASB\ in order to complete the deletion. Therefore, if a process $p$ executes $\func{Find}(k)$ procedure, and observes a node flagged with $\deleteflag$ attempting to delete the key $k$, it can not know whether in the future this delete attempt will succeed or fail, and thus does not know whether to consider the key $k$ as part of the tree or not. To overcome this problem, in such case the process will first try and validate the delete operation by marking the relevant node. According to whether the marking attempt was successful, the process can conclude if the delete operation is successful or not.
In order to easily implement the modified \func{Find} routine there is a need to conclude from \IFlag\ what is the new leaf (leaf $new$ in the \func{Insert} routine). For simplicity of presentation, we do not add this field, and abstractly refer to it in the code.

The correctness of the two suggested solutions relies on the following argument.
Once a process flags a node during operation $Op$ with input key $k$ (either \func{Insert} or \func{Delete}), then if this attempt to complete the operation eventually succeed (i.e., the marking is also successful in the case of \func{Delete}), then any \func{Find}(k) operation invoked from this point consider $Op$ as if it is completed.

The suggested modification, although being simple and local, only guarantee the implementation satisfy R-linearzability. However, the problem of response being lost in case of a crash is not addressed. Roughly speaking, the critical points in the code for recovery are the \CASB\ primitives, as a crash right after applying \CASB\ operation results the lost of the response, and in order to complete the operation the process needs to know the result of the \CASB. In addition, because of the helping mechanism, a suspended \func{Delete} operation which flagged a node and yet to mark one, may be completed by other process in the future, and may not. Upon recovery, the process needs to distinguish between the two cases, in order to obtain the right response.

To address this issue, we expend the helping mechanism so that it also update the info structure in case of a success. This is done by adding a boolean field, done, to the Flag structure. This way, if a process crash along an operation $Op$, upon recovery it can check to see if the operation was already completed. A crucial point is to update the $done$ bit before performing the unflagging. Therefore, if a node is no longer flagged we can be sure done was already updated.
If we switch the order, then it might be an operation and unflagging were completed, but the done bit is yet to be updated. Therefore, other processes can change the BST structure. However, if the process crash and recover at this point, the done bit is off, and the BST structure has been changed, so it will be harder for the process to conclude whether the operation took affect.

Before a process $q$ attempt to perform an operation, as it creates the \Flag\ structure $op$ describing the operation and its affect on the data structure, the process stores $op$ in a designated location in the shared memory (for simplicity, we use an array). As a result, upon recovery $q$ has an access to this information. Now, $q$ can check to see if the operation is still in progress, i.e., if the relevant node (parent or grandparent) is still flagged. If so, it first tries to complete the operation. Otherwise, it implies either the operation was completed, and therefore done bit is updated, or that the attempt was unsuccessful and there wa no write to the done bit. Hence, the done bit can distinguish between the two scenarios.
Notice that there is a scenario in which process $q$ recovers and observes an operation $Op$ as it in a progress, but just before it retries it, some other process complete the operation. We need to prove that even in such case, the operation will affect the data structure exactly once, and the right response is returned.

The given implementation does not recover the \func{Find} routine, since this routine does not make any changes to the BST, hence it is always safe to consider it as having no linearization point and reissue it. Also, for ease of presentation, we only write to $Announce[id]$ once we are about to capture a node using a CAS. However, writing to $Announce[id]$ at the beginning of the routine may be helpful in case of a crash early in the routine, so that the process will be able to use the data stored in $Announce[id]$ in such case also. The same is true with response value, $Announce[id]\rightarrow done$ is updated only if the routine made changes to the BST.

\subsection*{Correctness Arguments}

In the following section we give a proof sketch for the algorithm correctness. We assume for simplicity nodes and \Flag\ records are always allocated new memory locations, although it is enough to require no location is reallocated as long as there is a chain of pointers leading to it. The proof relies on the correctness of the original algorithm, which can be found on [....].

The proof relies on several key arguments given below.

\newcommand{\argAnonymous}{arg1}
[\textbf{Arg1}] The original algorithm is anonymous and uniform, i.e., any number of processes can use the BST, and there is no need to know the number of processes in the system in order to use the BST. Notice that all helping routines in the given implementation are completely anonymous, and an execution of such a routine by either the process which invoked $op$ or any other helping process executes the exact same code. This observation allows the use of the following argument. If a process crash while executing some helping routine, we can consider it as an helping process which stop taking steps (more formally, there is an equivalent execution in which there is such a process, and it is indistinguishable to all process in the system). Since such process can not cause a wrong behaviour of the algorithm, so does the crash.
A corollary of this argument is that repeating an helping routine multiple times by the same process can not violate the BST specification, as there is an equivalent executions in which multiple processes executes the different helping routines.

\newcommand{\argSearch}{arg2}
[\textbf{Arg2}] It is easy to verify the post-conditions of the \func{Search} routine still holds, as they follow directly from the routine's code, and does not rely on the structure or correctness of the BST. Also, the \func{Search} routine does not make any changes to the BST, but rather simply traverse it. Therefore \func{Find} routine, which only uses \func{Search}, does not affect any process, and in case of a failure along \func{Find} execution, reissuing it satisfies NRL.

\newcommand{\argNodeRef}{arg3}
[\textbf{Arg3}] If an internal node $nd_1$ stops pointing to a node $nd_2$ at some point of the execution, it can not point to $nd_2$ again. This attributes to the fact an \func{Insert} presents a node with two new children. Therefore, if $nd_2$ is a leaf, it can either be delete, or replaced by a new copy of an \func{Insert} operation. Otherwise, $nd_2$ is an internal node, and as such, the pointer to it by $nd_1$ can not be replaced by an \func{Insert} operation (which only allows to replacement a leaf), and therefore it can only be removed from the tree.

\newcommand{\argNodeUpdate}{arg4}
[\textbf{Arg4}] The field update of a node $nd$ can have any value only once along an execution. Any attempt to perform an operation creates a new record in the memory. If $nd\rightarrow update$ is marked, it can not be unmarked or changed. Otherwise, any attempt to flag it uses a new created record $op$. If the attempt succeed, then eventually it will be unflagged while still referring to $op$. In order to replace the value again, there must be an operation reading $nd\rightarrow update$ after it was unflagged (as any operation first help a flagged node). This operation must create a new record, and thus we can use the same argument again. As a corollary, if a process successfully flag or mark a node, there was no change to the node since the last time it read the update field of the node.

\paragraph{Proof Sketch}
Assume a process $q$ performs an operation $Op$ (either \func{Insert} or \func{Delete}). If $q$ does not crash, the algorithm is identical to the original algorithm, except for the additional write to $Announce[q]$ and $op\rightarrow done$, and thus the correctness of the original algorithm can be applied. Otherwise, $q$ crash at some point, and upon recovery it reads $op$ from $Announce[q]$. This record represent the last attempt of $q$ to complete $Op$. We split the proof based on the type of operation.

$Op = \func{Insert}$. Consider the read of $op\rightarrow p\rightarrow update$ upon recovery, and denote this value by $pupdate$. If $pupdate = \langle \insertflag, op \rangle$, this implies the iflag \CASB\ in line \lref{iflag-cas} was successful and the operation is yet to complete. It might be that \func{Insert} already took affect, that is, the new key is part of the tree, but the unflagging is yet to happen. In such case, $q$ calls \func{HelpInsert}$(op)$ in order to try and complete the operation. Considering \argAnonymous, this call can not violate the BST correctness, even if it not the first time $q$ executes it. Moreover, during \func{HelpInsert} there is a write to $op\rightarrow done$, and thus after completing the routine $q$ returns \TRUE, as required.

Else $pupdate \neq \langle \insertflag, op \rangle$. There are two scenarios to consider. Either the iflag \CASB\ of $q$ in line \lref{iflag-cas} was successful or not. If it was successful, then $p\rightarrow update = \langle \insertflag, op \rangle$ at this point. The only way to change it is to first unflag $p$. To do so, a process needs to complete an \func{HelpInsert}$(op)$ routine, and in particular must write to $op\rightarrow done$. In such case, the \func{Insert} operation was completed, and $q$ returns \TRUE. Otherwise, the \CASB\ was not successful, either because it failed, or the crash was before the \CASB. In both cases, the \func{Insert} operation will not be completed, as $op$ is not stored in $p\rightarrow update$, and thus no process has an access to it. Consequently, no process can update $op\rightarrow done$, and $q$ returns \FAIL.

$Op = \func{Delete}$. Consider the read of $op\rightarrow gp\rightarrow update$ upon recovery, and denote this value by $gpupdate$. If $gpupdate = \langle \deleteflag, op \rangle$, this implies the dflag \CASB\ of $q$ in line \lref{dflag-cas} was successful, and the operation is yet to complete. As in the \func{Insert}, it might be the operation already changed the tree. After reading $gpupdate$ $q$ invokes \func{HelpDelete}$(op)$ routine. Again, following \argAnonymous, executing this multiple times by $q$ can not violate the BST correctness.
The first process to try and mark $op\rightarrow p\rightarrow update$ during an \func{HelpDelete}$(op)$ routine is the one to determine the outcome of it. If it is successful, then $p$ is marked, and the $update$ field can not be changed. That is, any \func{HelpDelete}$(op)$ execution will obtain true in line \lref{checkmark}, and will call \func{HelpMarked}$(op)$ routine. Otherwise, the \CASB\ fails, and so $p\rightarrow update$ is no longer equal to $op\rightarrow pupdate$. By \argNodeUpdate\ it will never get this value again, and thus any marking \CASB\ during a \func{HelpDelete}$(op)$ execution will fail, and there is no call to \func{HelpMarked}$(op)$. In the first case, any \func{HelpDelete}$(op)$ routine must first complete a \func{HelpMarked}$(op)$, and thus must write to $op\rightarrow done$, while in the later case, there is no write to $op\rightarrow done$, as no \func{HelpMarked}$(op)$ is ever invoked. Therefore, in both cases, when $q$ completes \func{HelpMarked}$(op)$ it reads $op\rightarrow done$ and returns the right response.


Otherwise $gpupdate \neq \langle \deleteflag, op \rangle$, and there are two scenarios to consider. If the dflag \CASB\ of $q$ in line \lref{dflag-cas} never took affect, because it either failed, or the crash preceded it, then $op$ is never written to $gp\rightarrow update$, or to any update field. Thus, no process is aware of it, and $op\rightarrow done$ remains \FALSE, resulting $q$ returning \FAIL\ as required. Else, the \CASB\ was successful, and $gp\rightarrow update$ was flagged. The only way to change it is to first unflag it, and this in turn can be done only during an \func{HelpDelete}$(op)$ routine. In this case, it can be unflagged in either the \func{HelpMarked} routine in line \lref{dunflag-cas}, or in line \lref{backtrack-cas} of the \func{HelpDelete} routine. As mention before, the first \CASB\ in line \lref{mark-cas} of an \func{HelpDelete}$(op)$ execution determines the outcome for all \func{HelpDelete}$(op)$. If it is successful, $p\rightarrow update$ is forever marked, and all \func{HelpDelete}$(op)$ must invoke \func{HelpMarked}$(op)$. Therefore, the only option to unflag $gp\rightarrow update$ is at the end of \func{HelpMarked}$(op)$ routine, and this done only after setting $op\rightarrow done$. In such case, the \func{Delete} operation took affect, and $q$ will return \TRUE. On the other hand, if the \CASB\ was not successful, then any \func{HelpDelete}$(op)$ will fail to mark $p\rightarrow update$, and hence no \func{HelpMarked}$(op)$ is ever invoked. As a result, there is no write to $op\rightarrow done$. In such case, the \func{Delete} operation did not took affect, nor will be, and indeed $q$ will return \FAIL.





\remove{
%%%%%%%REMOVE%%%%%%%%%%%%%%
Assume process $q$ performs an \func{Insert}$(k)$ operation. As argued in \argSearch, a crash before writing to $Announce[q]$ implies no changes has been made to the BST. Assume thus $q$ executes line \lref{insert-write-announce}, i.e., $q$ stored in $Announce[q]$ a pointer to an \IFlag\ record $op$ containing all the data of the current attempt to complete the \func{Insert} routine. Since $q$ is in the middle of a while loop, it is enough to prove that if $q$ crash before the next time it writes to $Announce[q]$, if there is such write, upon recovery it will either complete its operation with the right response, or will continue to the next write to $Announce[q]$ without having any affect on the BST. Hence, the same argument can be applied once $q$ writes to $Announce[q]$ again. Notice that if $q$ does not crash, the algorithm is identical to the original algorithm, except for the additional write to $op\rightarrow done$, and thus the original proof can


Assume $q$ performs a successful CAS in line \lref{iflag-cas}. Then, a reference to the \IFlag\ $op$ is stored in $p\rightarrow update$, which is also flagged. Following \argNodeUpdate, $p$ was not changed since the \func{Search} routine read it, and it still points to $l$.
Starting from this point, no changes can be made to $p$, except for the change point to by $op$, as the node is flagged. Now, only the first process

Relying on the correctness of the original algorithm, no matter how many times \func{HelpInsert}(op) will be executed, the change will occur only once. This follows from the fact that many process can observe op, and will try to complete it in the future. The core for this argument is that a node never point twice to the same node.
}







\section{Elimination Stack}
\label{section:elimination-stack}

For simplicity, we assume a value \init, which is different from \NULL\ and any other value the stack can store. Since \NULL\ is used as a legit return value, representing the value of \pop\ operation (when exchanging values using the elimination array), \NULL\ can not be used to represent an initialization value, different then any stack value. The same holds for a Node, since a \NULL\ node represent an empty stack, the value \init\ is used to distinguish between initialization value and empty stack.

For simplicity, we split the \recover\ routine into sub-routines, based on which operation (\push, \pop, \exchange) is pending, or needs to be recover. This can be concluded easily by the type of record stored in $Announce[pid]$ (\exInfo\ or \opInfo), thus there is no need to explicitly know where exactly in the code the crash took place. Also, the \recover\ routine returns \fail\ in case the last pending operation did not took affect (no linearization point), nor it will take in any future run. In such case, the user has the option to either re-invoke the operation, or to skip it, depends on the needs and circumstances of the specific use of the data structure.

The given implementation ignores the log of failures and successes of the exchange routine when recovering. That is, in case of a crash during an \exchange, a process is able to recover the \exchange\ routine, however, the log of successes and failures is not update, since it might be the process already updated it. In addition, in case of a \fail\ response, we do not know whether the time limit (timeout) was reached, or that the process simply crashed earlier in the routine without completing it. The given implementation can be expanded to also consider the log. Nonetheless, for ease of presentation we do not handle the log in case of a crash. Assuming crash events are rare, the log still gives a roughly good approximation to the number of failures and successes, thus our approach might be useful in practice.

\subsection{A Lock-Free Exchanger}
An exchanger object supports the \exchange\ procedure, which allows exactly two processes to exchange values.
If process A calls the \exchange\ with argument $a$, and B calls the \exchange\ of the same object with argument $b$, then A's call will return value $b$ and vice versa.

On the original algorithm [cite the book?!], processes race to win the exchanger using a \CAS\ primitive. A process accessing the exchanger first reads its content, and act according to the state of it. The first process observe an \emptyst\ state, and tries to atomically writes its value and change the state to \waiting. In such case, it spins and wait for the second process to arrive. The second, observing the state is now \waiting, tries to write its value and change the state to \busy. This way, it informs the first one a successful collision took place. Once the first process notice the collision, it reads the other process value and release the exchanger by setting it back to \emptyst.
In order to avoid an unbounded waiting, if a second process does not show up, the call eventually timeout, and the process release the exchanger and return.

Assume a process $p$ successfuly capture the exchanger by setting its status to \waiting, followed by a crash. Now, some other process $q$ complete the exchange by setting the exchanger to \busy. Upon recovery, $p$ can conclude some exchange was completed, but it can not tell whether its value is part of the exchange, and thus it can not complete the operation. Moreover, $p$ and $q$ must agree, otherwise $q$ will return $p$'s value, and thus the operation of $p$ must be linearized together with $q$ operation.

In order to avoid the above problem, we take an approach resembling the BST implementation. Instead of writing a value to the exchanger, processes will use an info record, containing the relevant information for the exchange. This way, processes use the exchanger in order to exchange info records (more precisely, pointers to such records), and not values. To overcome the problematic scenario described earlier, if a process $q$ observe the exchanger state is \waiting\ with some record $yourop$, it first update its own record $myop$ it is about to try and collide with $yourop$, and only then performs the \CAS. This way, if the collision is successful, the record $myop$ which now stored in the exchanger implies which two records collide. Also, the fact that different processes uses different records guarantee that at most one record can collide with $yourop$.

Using records instead of values, when using wisely, allows us to farther improve the algorithm. First, there is no need to store the exchanger's state in it (by using 2 bits of it to mark the state), but we can rather have this info in the record. Second, if there is a \busy\ record in the exchanger, it contains the info of the two colliding records. Therefore, a third process, trying to also use the exchanger, can help the processes to complete the collision, and then can try and set the exchanger back to \emptyst, so it can use it again. In the original implementation, a process observaing a \busy\ exchanger, have to wait for the first process to read the value and release the exchanger. Therefore, if the first process crash after the collision, the exchanger will be hold by it forever. The helping mechanism avoids this scenario, making the exchange routine non-blocking.

Notice that no exchange record with \emptyst\ state is ever created, except for the $default$ record. Therefore, reading \emptyst\ state is equivalent to the exchanger storing a pointer to $default$. A process $p$ creates a new record $myop$ when accessing the exchanger, with a unique address. As long as $p$ fails to perform a successful \CAS, and thus fails to store $myop$ in $slot$, it is allowed to try again. However, once a process performs a successful \CAS\ and stores $myop$ in $slot$, the only other \CAS\ it is allowed to do are in order to try and store $defualt$ in $slot$. Thus, $myop$ can be written exactly once to $slot$. It follows that a collision can occur between two processes exactly - once a \waiting\ record stored in $slot$, only a single \CAS\ can replace it with a \busy\ record. As the two records can not be written again to $slot$, no other process can collide with any of the records.

The \recoverExchange\ routine relies on the following argument. If a process $p$ successfully wrote $op_p$ to $slot$ using the \CAS\ in line~\ref{exchage-waiting-cas}, the only way to overwrite it by a different process $q$, is by a \CAS\ in line~\ref{exchange-busy-cas} with a record $op_q$ such that its state is \busy, and $op_q.partner = op_p$. In addition, the only way to overwrite $op_q$ is by a \CAS\ replacing it with $default$, and this is done only after \switchPair$(op_p, op_q)$ is completed, and thus both $result$ fields are updated.

The correctness of the \recoverExchange\ routine is based on the above argument. There are few scenarios to consider.
If $p$ crash after a successful \CAS\ in line~\ref{exchage-waiting-cas}, then $op_p$ state is \waiting. Therefore, when reading $slot$ in the \recoverExchange\ one of the following must hold. If $slot$ contains $op_p$, then no process collide with $p$, and $p$ continue to run as if the time limit has been reached. Otherwise, there was a collision. From the above argument, it must be that either $op_q$ that collide with $op_p$ is stored in $slot$, in this case $op_q.partner = op_p$, and $p$ will try to complete the collision and release $slot$, or that $op_q$ has been overwritten, and in this case the $result$ field of $op_p$ is updated. In both cases, $p$ returns $op_p.result$.
If $p$ crash after a successful \CAS\ in line~\ref{exchange-busy-cas}, then $op_p$ state is \busy. It follows from the argument that the only way to overwrite $op_p$ is only after completing the collision by \switchPair. Thus, either upon recovery $p$ reads $op_p$ from $slot$, and in this case it tries to complete the the operation, or that $op_p.result$ was already updated. In both cases, $p$ returns it.
If non of the above holds, then $op_p$ was not involved in any collision, because either no successful \CAS\ was done by $p$, or $p$ reached the time limit while no process show up, and was able to set $slot$ back to $defualt$. In any case, after the crash of $p$, $op_p$ will never be written again to $slot$, nor any other $op_q$ such that $op_q.partner = op_p$, as any such $op_q$ tries to perform \CAS$(op_p,op_q)$ that will fail. Also, as no process can collide with $op_p$, no \switchPair\ with $op_p$ as parameter is ever invoked, and in particular $op_p.result = \init$ for the rest of the execution. This in turn implies that upon recovery $p$ will return \fail, as required.


\subsection{Lock-Free Stack}

The stack implementation is due to [....]. The \trypush\ routine tries to atomically have a new node pointing to the old top, and then updating the top to be the new node. The \trypop\ routine tries to atomically read the top of the stack, and change the top to the next node of it. The two routines uses \CAS\ in order to gurantee no change for the top was made between the read and write.
\push\ (resp. \pop) routine is alternating between a \trypush\ (\trypop) routine, which access the central stack, and the \exchange, trying to collide with an opposite operation.

In order to make the implementation recoverable, we need a way to infer whether a \pop\ or \push\ already took affect, in case of a crash. Moreover, in case of a \pop, we also need to infer which process is the one to pop the node. For that, we use an approach similar to the Linked-List implementation. Each node contains a new field $popby$ which is used to identify a \push\ of the node completed, as well as a \pop\ of the node was completed, and who is the process to pop it.
Consider the following scenario. Assume a process $p$ performs a \push\ operation with node $nd$, and using a \CAS\ succeed to update the stack top to point to $nd$, followed by a crash. Now, process $q$ performing a \pop\ operation performs a \CAS\ causing the removal of $nd$ from the stack (by changing top to the next node). In this case, once $p$ recovers, $nd$ is no longer part of the stack, and it is also not marked as deleted. This is indistinguishable from a configuration in which the \push\ of $nd$ was yet to take affect (a crash before \CAS), and thus $p$ can not know what the right response is.

One way to solve this issue is by first marking a node for removal, and only then remove it. This way, if a node is no longer part of the stack it must be marked, and thus we can conclude it was in the stack, and the \push\ routine was successful. However, such an implementation, in addition for the need of to system to support a markable reference, also requires process to help each other. If a node is marked for delete, then a process trying to perform a different operation first needs to complete the deletion, before applying its own operation, otherwise the physical delete of the node may not take place, leaving the node forever in the stack. As the original algorithm avoids any marking, and simply tries to swing the $Top$ pointer, we would like to maintain this property.

A field $popby$ is initialised to $\init$ when a node $nd$ is created. Once the node is successfully insert to the stack by a \push\ operation, the inserting process tries to mark it by changing $popby$ to \NULL\ using a \CAS. Before a process tries to remove the node from the stack during a \pop\ routine, it first mark it as part of the stack by doing the same thing, helping the inserting process conclude the node is in the stack. This replace the logic delete of the node, as we only need to know the node was part of the stack if it is removed. After a successful \CAS\ to remove $nd$ from the stack, another \CAS\ is used in order to try and set $popby$ to the identifier of the process who performed the \CAS. The use of \CAS\ to change $popby$ from \init to \NULL, and from \NULL\ to an identifier guarantee that only the first process to perform each of these \CAS\ will succeed. Note that before writing an identifier to $popby$ a process must try and set it to \NULL, and thus it can not store two different identifiers along in any execution.

The correctness proof follows the same guidelines as of the proof for the Linked-List. If a \push\ operation did not introduce a new node $nd$ into the stack, then no process but $p$ is aware of $nd$. Thus, upon recovery the \search\ routine will not find $nd$ in the stack, nor its $popby$ field has been changed, and the \recoverPush\ returns \fail. Otherwise, $nd$ was successfully inserted to the stack. As discessed above, the only way to delete $nd$ from the stack is by first changing its $popby$ field to \NULL. Thus, upon recovery $p$ will either find $nd$ in the stack, using the \search\ routine, or that $popby$ is different then $\init$ in case it was deleted, and in both cases it returns \True.
For the \pop\ routine, if $p$ tries to remove a node $nd$ from the top of the stack and crash, then upon recovery it first check if $nd$ is still in the stack using the \search\ routine. If it is so, then clearly $nd$ was yet to delete, and it returns \fail. Otherwise, $nd$ was deleted, either by $p$ or by some other process. Only the first process of which to performs a \CAS, writing its identifier to $popby$ will return the value stored in $nd$, while the others return either \init\ (in the \trypop\ routine) or \fail\ (in the \recoverPop\ routine).

Notice that both \recoverPush\ and \recoverPop\ are wait-free. Due to the structure of stack, no $next$ pointer of any node in the stack is ever changed. Therefore, once a process reads $Top$ at the beginning of its \recover\ routine, the chain of pointers from this $Top$ to the last node in the stack is fixed for the rest of the execution, and thus traversing it using the \search\ routine is wait-free.





%% Bibliography
\bibliography{references}


%% Appendix
%\clearpage
%\appendix
%\section{Appendix}
\onecolumn


%\setcounter{linenum}{0}


\begin{figure}[H]
	\footnotesize
	\begin{code}
		\firstline
		type Update \{ \hspace*{14mm} \com stored in one CAS word\nlc
		\n  $\{\clean,\deleteflag,\insertflag,\mk\}$ $state$ \nlc
		\Flag\ *{\it info}\nlc
		\p
		\}
		\nlc
		type Internal \{ \hspace*{13.25mm} \com subtype of Node\nlc
		\n $\Key \cup \{\infty_1,\infty_2\}$ $key$\nlc
		Update $update$\nlc
		Node *\lft, *$right$\nlc
		\p\}
		\nlc
		type Leaf \{ \hspace*{18.3mm} \com subtype of Node\nlc
		%\hspace*{4mm} \com could also store auxiliary information\nlc 
		\n $\Key \cup \{\infty_1,\infty_2\}$  $key$\nlc
		\p\}
		\nlc
		type \IFlag\ \{ \hspace*{17.7mm} \com subtype of \Flag\nlc
		\n Internal *$p$, *$newInternal$\nlc
		Leaf *$l$\nlc
		\textcolor{blue}{boolean $done$} \nlc
		\p\}
		\nlc
		type \DFlag\ \{ \hspace*{16.4mm} \com subtype of \Flag\nlc
		\n Internal *$gp$, *$p$\nlc
		Leaf *$l$\nlc
		Update $pupdate$\nlc
		\textcolor{blue}{boolean $done$} \nlc
		\p\}   
		\ul
		\com Initialization:\nlc
		shared Internal *$Root$ := pointer to new Internal node \ul
		\n with $key$ field $\infty_2$, $update$ field $\langle \clean,\NULL\rangle$, and\ul
		pointers to new Leaf nodes with keys $\infty_1$ and\ul 
		$\infty_2$, respectively, as \lft\ and $right$ fields.
		\p 
	\end{code}
	\caption{\label{code1} BST type definitions and initialization.}
\end{figure}



\begin{figure}[H]
	\footnotesize
	\color{blue} {
	\begin{code}
		
		\firstline
		\func{Insert-Recover}() \{ \nlc \n
		\IFlag\ $*op = RD[pid]$ \bl \nlc
		
		if $\text{CP}_p$ $<$ \lref{BSTinsert-set-checkpoint} or $op = \init$ then \label{BSTinsert-recover-read-CP} \nlc \n
			Proceed from line \lref{BSTInsert-entry} \nlc \p

		$test := op\rightarrow p\rightarrow update$ \nlc
		if $test = \langle \insertflag,op \rangle$ then \func{HelpInsert}$(op)$ \tabtabcom Finish the insertion \nlc
		if $op\rightarrow done = \TRUE$ then return \TRUE\ \nlc
		else Proceed from line \lref{BSTInsert-entry} \nlc
		\p \} \bl \nlc
		
		
		\func{Delete-Recover}() \{ \nlc \n
		\DFlag\ $*op = RD[pid]$ \bl \nlc
		
		if $\text{CP}_p$ $<$ \lref{BSTdelete-set-checkpoint} or $op = \init$ then \label{BSTdelete-recover-read-CP} \nlc \n
		Proceed from line \lref{BSTDelete-entry} \nlc \p
		
		$test := op\rightarrow gp\rightarrow update$ \nlc
		if $test = \langle \deleteflag, op \rangle$ then \func{HelpDelete}$(op)$ \tabtabcom Either finish deletion or unflag \nlc
		if $op\rightarrow done = \TRUE$ then return \TRUE\ \nlc
		else Proceed from line \lref{BSTDelete-entry} \nlc
		\p \}
	
	\end{code}
}
	\caption{\func{Recover} routines}
\end{figure}



\begin{figure}[H]
	\scriptsize
	\begin{code}
		\firstline
		\func{Search}($\Key\ k$) : $\langle \mbox{Internal*}, \mbox{Internal*}, \mbox{Leaf*}, \mbox{Update}, \mbox{Update}\rangle$  \{\ul
		\n \com Used by \func{Insert}, \func{Delete} and \func{Find} to traverse a branch of the BST; satisfies following {\it postconditions}:\ul
		\com \postnotnull\ $l$ points to a Leaf node and $p$ points to an Internal node\ul
		\com \postl\ Either $p\rightarrow \lft$ has contained $l$ (if $k<p\rightarrow key$) or $p\rightarrow right$ has contained $l$ (if $k\geq p\rightarrow key$)\ul
		\com \postpup\ $p\rightarrow update$ has contained $pupdate$\ul
		\com \postnonempty\ if $l\rightarrow key\neq \infty_1$, then the following three statements hold:\ul
		\com \hspace*{3mm}\postgpnotnull\ $gp$ points to an Internal node\ul
		\com \hspace*{3mm}\postp\ either $gp\rightarrow \lft$ has contained $p$ (if $k<gp\rightarrow key$) or $gp\rightarrow right$ has contained $p$ (if $k\geq gp\rightarrow key$)\ul
		\com \hspace*{3mm}\postgpup\ $gp\rightarrow update$ has contained $gpupdate$\nlc
		Internal *$gp$, *$p$\nlc
		Node *$l:= Root$  \label{restart-search}\nlc
		
		Update $gpupdate, pupdate$ \tabtabcom Each stores a copy of an $update$ field\bl\nlc
		%\com The initial values of $p,gp,pupdate$ and $gpupdate$ are unimportant
		%$p :=\NULL$ 
		%$pupdate := \langle \clean, \NULL\rangle$\nlc
		while $l$ points to an internal node \{ \nlc%\com Advance down the tree\nlc
		\n         $gp := p$ \tabtabcom Remember parent of $p$\nlc
		$p := l$ \tabtabcom Remember parent of $l$\nlc
		$gpupdate := pupdate$ \tabtabcom Remember $update$ field of $gp$\nlc
		$pupdate := p\rightarrow update$\label{store-pupdate}\tabtabcom Remember $update$ field of $p$\nlc  
		%           if $pupdate.state = \mk$ then \label{checkmark} \{\nlc
		%\n              \func{HelpMarked}$(pupdate.info)$ \label{call-hm3}\nlc
		%                goto line \lref{restart-search} \com Restart traversal\nlc
		%\p         \}\nlc        
		if $k < l\rightarrow key$ then $l:= p\rightarrow \lft$ else $l:=p \rightarrow right$ \label{read-child}\tabtabcom Move down to appropriate child\nlc
		%           if $pupdate \neq p\rightarrow update$ then goto line \lref{restart-search} \com Restart traversal \label{reread-pupdate}\nlc
		\p \} \nlc
		return $\langle gp, p, l, pupdate, gpupdate \rangle$ \nlc
		\p 
		\}\bl
		\nlc
		\func{Find}($\Key\ k$) : Leaf* \{ \nlc
		\n   Leaf *$l$\bl
		\nlc
		%     Update $pupdate,gpupdate$\ul \nlc
		$\langle -,-,l,-,-\rangle := \func{Search}(k)$\nlc
		if $l\rightarrow key = k$ then return $l$\nlc
		else return \NULL\nlc
		\p
		\}\bl
		\nlc
		\func{Insert}($\Key\ k$) : boolean \{ \nlc
		\n Internal *$p$, *$newInternal$ \label{BSTInsert-entry} \nlc 
		Leaf *$l$, *$newSibling$\nlc 
		Leaf *$new :=$ pointer to a new Leaf node whose $key$ field is $k$  \nlc
		Update $pupdate, result$\nlc
		\IFlag\ *$op$\bl\nlc%:=$ pointer to a new \IFlag\ \record\ \ul\nlc
		
		\textcolor{blue} {$RD[pid]$ := \init} \nlc
		\textcolor{blue} {$\text{CP}_p$ := curPC()}  \tabtabcom Set a check-point indicating \IFlag\ structure was installed \label{BSTinsert-set-checkpoint} \nlc
		while \TRUE\ \{  \nlc
		\n $\langle -, p, l, pupdate, - \rangle := \func{Search}(k)$ \label{ins-search}\nlc
		if $l \rightarrow key = k$ then return \FALSE\ \tabtabcom Cannot insert duplicate key\label{insert-false}\nlc
		if $pupdate.state \neq \clean$ then \func{Help}$(pupdate)$ \tabtabcom Help the other operation \label{ins-help-unclean}\nlc
		else \{\nlc
		\n        $newSibling :=$ pointer to a new Leaf whose key is $l\rightarrow key$\nlc
		$newInternal :=$ pointer to a new Internal node with $key$ field $\max(k, l \rightarrow key)$,\label{create-internal}\ul      
		\n      $update$ field $\langle \clean, \NULL\rangle$, and with two
		child fields equal to $new$ and $newSibling$\ul 
		(the one with the smaller key is the left child)\nlc
		\p        $op :=$ pointer to a new \IFlag\ \record\  containing $\langle p, l, newInternal, \textcolor{blue}{\FALSE}\rangle$\label{new-IFlag}\nlc
		\textcolor{blue}{$RD[id] := op$} \label{insert-write-announce} \nlc
		$result := \CASB(p\rightarrow update, pupdate, \langle \insertflag, op\rangle)$ \tabtabcom {\bf iflag \CASB}\label{iflag-cas} \nlc
		if $result = pupdate$ then \{ \tabtabcom The iflag \CASB\ was successful\nlc
		\n            \func{HelpInsert}$(op)$ \tabtabcom Finish the insertion\label{finish-insert}\nlc
		return \TRUE\ \label{insert-true}\nlc
		\p        \}\nlc 
		else \func{Help}$(result)$ \tabcom The iflag \CASB\ failed; help the operation that caused failure\label{ins-help-after-failure}\nlc
		\p    \}\nlc
		\p\}\nlc 
		\p
		\}\bl
		\nlc
		
		\func{HelpInsert}(\IFlag\ *$op$) \{\ul
		\n     \com {\it Precondition}:  $op$ points to an \IFlag\ \record\  (\ie, it is not $\NULL$)\nlc
		%	\textcolor{blue}{$op \rightarrow done := \insertflag$} \tabtabcom {announce the iflag \CASB\ was successful}\nlc 
		\func{CAS-Child}$(op\rightarrow p, op\rightarrow l, op\rightarrow newInternal)$ \tabtabcom {\bf ichild \CASB}\label{ichild-cas}\nlc
		\textcolor{blue}{$op \rightarrow done := \TRUE$} \tabtabcom {announce the operation completed}\nlc 
		$\CASB(op\rightarrow p\rightarrow update, \langle \insertflag, op \rangle, \langle \clean, op\rangle)$ \tabtabcom {\bf iunflag \CASB} \label{iunflag-cas}\nlc
		\p
		\}
	\end{code}
	\caption{\label{code2}Pseudocode for \func{Search}, \func{Find} and \func{Insert}.}
\end{figure}

\begin{figure}[H]
	\scriptsize
	\begin{code}
		\firstline
		\func{Delete}($\Key\ k$) : boolean \{\nlc
		\n Internal *$gp$, *$p$ \label{BSTDelete-entry} \nlc
		Leaf *$l$\nlc
		Update $pupdate, gpupdate, result$\nlc
		\DFlag\ *$op$\bl\nlc
		\textcolor{blue} {$RD[pid]$ := \init} \nlc
		\textcolor{blue} {$\text{CP}_p$ := curPC()}  \tabtabcom Set a check-point indicating \IFlag\ structure was installed \label{BSTdelete-set-checkpoint} \nlc
		
		while \TRUE\ \{ \nlc
		\n     $\langle gp, p, l, pupdate, gpupdate \rangle := \func{Search}(k)$\label{del-search}\nlc
		if $l\rightarrow key \neq k$ then return \FALSE\ \tabtabcom Key $k$ is not in the tree\label{delete-false}\nlc
		%       assert: $gp$ points to an internal node\label{assert-gpnotnull}\nlc
		if $gpupdate.state \neq \clean$ then \func{Help}$(gpupdate)$ \label{del-help-unclean-1}\nlc
		else if $pupdate.state \neq \clean$ then \func{Help}$(pupdate)$\label{del-help-unclean-2}\nlc
		else \{ \tabtabcom Try to flag $gp$\nlc
		\n          $op :=$ pointer to a new \DFlag\ \record\  containing $\langle gp, p, l, pupdate, \textcolor{blue}{\FALSE} \rangle$\label{new-DFlag}\nlc
		\textcolor{blue}{$RD[id] := op$} \nlc
		$result := \CASB(gp\rightarrow update, gpupdate, \langle \deleteflag, op \rangle)$ \tabtabcom {\bf dflag \CASB}\label{dflag-cas}\nlc
		if $result = gpupdate$ then \{ \tabtabcom \CASB\ successful \nlc
		\n             if \func{HelpDelete}$(op)$ then return \TRUE \label{delete-true} \tabtabcom Either finish deletion or unflag\nlc
		\p          \}\nlc                 
		else \func{Help}$(result)$ \tabcom The dflag \CASB\ failed; help the operation that caused the failure \label{del-help-after-failure}\nlc%; help operation that is in the way 
		\p     \}\nlc
		\p \}\nlc
		\p
		\}\bl
		\nlc
		
		\func{HelpDelete}(\DFlag\ *$op$) : boolean \{\ul
		\n   \com {\it Precondition}:  $op$ points to a \DFlag\ \record\  (\ie, it is not \NULL)\nlc%$op\rightarrow pupdate.state = \clean$\nlc
		Update $result$ \tabtabcom Stores result of mark \CASB\bl\nlc
		%\com First, try to mark the node that $op\rightarrow p$ points to\nlc
		%$pupdate = op\rightarrow pupdate$\nlc
		$result := \CASB(op\rightarrow p\rightarrow update, op\rightarrow pupdate, \langle \mk, op\rangle)$ \tabtabcom {\bf mark \CASB}\label{mark-cas}\nlc     
		if $result = op\rightarrow pupdate$ or $result = \langle \mk, op\rangle$ then \label{checkmark}\{\tabtabcom $op\rightarrow p$ is successfully marked\nlc
		\n          \func{HelpMarked}$(op)$ \label{call-hm1} \tabtabcom Complete the deletion\nlc
		return \TRUE\tabtabcom Tell \func{Delete} routine it is done\nlc
		\p       \}\nlc
		else \{\tabtabcom The mark \CASB\ failed \nlc
		\n              
		\func{Help}$(result)$\label{help-after-failed-mark}\tabtabcom Help operation that caused failure\nlc
		\CASB$(op\rightarrow gp\rightarrow update, \langle \deleteflag, op\rangle, \langle \clean, op\rangle)$ \tabtabcom {\bf backtrack \CASB}\label{backtrack-cas}\nlc
		
		return \FALSE\tabtabcom Tell \func{Delete} routine to try again\nlc
		\p       \}\nlc
		\p
		\}\bl
		\nlc
		
		\func{HelpMarked}(\DFlag\ *$op$) \{\ul
		\n   \com {\it Precondition}:  $op$ points to a \DFlag\ \record\  (\ie, it is not $\NULL$)\nlc
		Node *$other$\bl\ul
		\com Set $other$ to point to the sibling of the node to which  $op\rightarrow l$ points \nlc
		if $op\rightarrow p\rightarrow right = op\rightarrow l$ then $other := op\rightarrow p \rightarrow \lft$ else $other:=op\rightarrow p\rightarrow right$\label{read-other}\ul 
		\com Splice the node to which $op\rightarrow p$ points out of the tree, replacing it by $other$\nlc
		\func{CAS-Child}$(op\rightarrow gp, op\rightarrow p, other)$ \tabtabcom {\bf dchild \CASB}\label{dchild-cas}\nlc
		%     \com Unflag the node $op\rightarrow gp$ points to\nlc
		\textcolor{blue}{$op \rightarrow done := \TRUE$} \tabtabcom {announce the operation completed} \nlc
		\CASB$(op\rightarrow gp\rightarrow update, \langle \deleteflag, op\rangle, \langle \clean, op\rangle)$ \tabtabcom {\bf dunflag \CASB}\label{dunflag-cas}\nlc
		\p
		\}\bl\nlc
		
		\func{Help}(Update $u$) \{ \tabtabcom General-purpose helping routine\ul
		\n    \com {\it Precondition}:  $u$ has been stored in the $update$ field of some internal node\nlc
		if $u.state = \insertflag$ then \func{HelpInsert}$(u.\info)$\label{call-HelpInsert}\nlc
		else if $u.state = \mk$ then \func{HelpMarked}$(u.\info)$\label{call-hm2}\nlc
		else if $u.state = \deleteflag$ then \func{HelpDelete}$(u.\info)$\label{call-HelpDelete}\nlc
		\p
		\}\bl
		\nlc
		\func{CAS-Child}(Internal *$parent$, Node *$old$, Node *$new$) \{\label{CAS-Child}\ul
		\n  \com {\it Precondition}:  $parent$ points to an Internal node and $new$ points to a Node (\ie, neither is \NULL)\ul
		\com This routine tries to change one of the child fields of the node that $parent$ points to from $old$ to $new$.\nlc
		if $new \rightarrow key < parent\rightarrow key$ then\label{which-child}\nlc
		\n       \CASB$(parent\rightarrow \lft, old, new)$\label{child-cas-1}\nlc
		\p  else\nlc
		\n       \CASB$(parent\rightarrow right, old, new)$\label{child-cas-2}\nlc
		\p\p 
		\}
	\end{code}
	\caption{\label{code3}Pseudocode for \func{Delete} and some auxiliary routines.}
\end{figure}





\remove{

\begin{figure}[H]
	\footnotesize
	
	\begin{code}
		\func{Find}($\Key\ k$) : Leaf* \{ \nlc
		\n Internal *$gp$, *$p$\nlc
		Leaf *$l$\nlc
		Update $pupdate, gpupdate$\bl
		\nlc
		
		%     Update $pupdate,gpupdate$\ul \nlc
		while (\TRUE) \{ \nlc \n
		$\langle gp, p, l, pupdate, gpupdate \rangle := \func{Search}(k)$\nlc
		if $gpupdate.state \neq \clean$ then $\func{Help}(gpupdate)$ \nlc
		else if $pupdate.state \neq \clean$ then $\func{Help}(pupdate)$ \nlc
		else if $gp \rightarrow update = gpupdate$ and $p \rightarrow update = pupdate$ then \{ \nlc \n
		if $l \rightarrow key = k$ then return $l$ \nlc
		else return \NULL \nlc
		\p \} \nlc
		\p \} \nlc
		\p \}
	\end{code}
	
	\caption{Solution 1: R-linearizable \func{Find} routine}
	\label{robust find - solution 1}
\end{figure}



\begin{figure}[H]
	\footnotesize
	
	\begin{code}
		\func{Find}($\Key\ k$) : Leaf* \{ \nlc
		\n Internal *$gp$, *$p$\nlc
		Leaf *$l$\nlc
		Update $pupdate, gpupdate$\bl
		\nlc
		
		
		%     Update $pupdate,gpupdate$\ul \nlc
		$\langle gp, p, l, pupdate, gpupdate \rangle := \func{Search}(k)$\nlc
		if $l\rightarrow key \neq k$ then \{ \nlc
		\n if ($pupdate.state = \insertflag$ and $pupdate.info$ attempt to add key $k$) then \nlc
		\n return leaf with key $k$ from $pupdate.info$ \nlc \p
		else return \NULL \nlc \p
		\} \nlc
		if ($pupdate.state = \mk$ and $pupdate.info \rightarrow l \rightarrow key = k$) then return \NULL \nlc
		if ($gpupdate.state = \deleteflag$ and $gpupdate.info \rightarrow l \rightarrow key = k$) then \{ \nlc
		\n $op := gpupdate.info$ \nlc
		$result := \CASB (op \rightarrow p \rightarrow update, op \rightarrow pupdate, \langle \mk, op \rangle)$ \tabtabcom {\bf mark \CASB} \label{robust-find2-mark-cas} \nlc
		if ($result = op\rightarrow pupdate$ or $result = \langle \mk, op\rangle$) then return \NULL \label{robust-find2-checkmark} \tabtabcom $op\rightarrow p$ is successfully marked \nlc \p
		\} \nlc
		return $l$ \nlc \p
		\}
	\end{code}
	
	\caption{Solution 2: R-linearizable \func{Find} routine}
	\label{robust find - solution 2}
\end{figure}

}

\clearpage


\begin{figure*}[]
	\small
\begin{flushleft}
	Type Node \{ \\
	\hspace*{6mm} T $value$ \\
	\hspace*{6mm} int $popby$ \\
	\hspace*{6mm} Node $*next$ \\
	\} \\
	
	\remove{
	Type \pushInfo\ \{ \hspace*{20.0mm} $\triangleright$ subtype of \Info \\
	\hspace*{6mm} Node $*pushnd$ \\
	\} \\
	
	Type \popInfo\ \{ \hspace*{22.0mm} $\triangleright$ subtype of \Info \\
	\hspace*{6mm} Node $*popnd$ \\
	\} \\
	}

	Type \csInfo\ \{ \hspace*{22.0mm} $\triangleright$ subtype of \Info \\
	\hspace*{6mm} Node $*nd$ \\
	\hspace*{6mm} T $result$ \\
	\} \\
	
	Type \exInfo\ \{ \hspace*{24.0mm} $\triangleright$ subtype of \Info \\
	\hspace*{6mm} \{\emptyst, \waiting, \busy\} $state$ \\
	\hspace*{6mm} T $value, result$ \\
	\hspace*{6mm} \exInfo\ $*partner, *slot$ \\
	\}
	
\end{flushleft}	
	\caption{Elimination-Stack type definition}
\end{figure*}



%%%%%%%%%%%%%%%%%%Elimination array%%%%%%%%%%%%%%%%%
\begin{figure*}[!t]
	\removelatexerror
	
	\begin{algorithm}[H]
		
	\small
	\begin{flushleft}	
	\end{flushleft}
	
	\exInfo\ $default$ - global static \exInfo\ object with state = \emptyst
	
	\begin{procedure}[H]
		\caption{() T \exchange\ (\exInfo\ $*slot$, T $myitem$, long $timeout$)}
		
		long $timeBound$ := getNanos() + $timeout$ \;
		\exInfo\ $myop$ := \textbf{new} \exInfo (\waiting, $myitem$, \init, \init, $slot$)\;
		$RD[pid]$ := $myop$ \tcp*{update Info structure}
		
		\While{\True} {
			\uIf{getNanos() $> timeBound$} {
				%$myop.result := \timeout$ \tcp*{time limit reached}
				\KwRet \timeout 
			}
			%$myop.state$ := \waiting \tcp*{set to default state}
			$yourop := slot$ \;
			%		$\langle$youritem,state$\rangle$ := slot\;
			\Switch{$yourop.state$} {
				\uCase{\emptyst} {
					$myop.state$ := \waiting \tcp*{attempt to replace $default$}
					$myop.partner := \init$ \;
					\If (\tcp*[f]{try to collide}) {$slot.\CAS (yourop, myop)$ \label{exchage-waiting-cas}} {
						\While{getNanos() $< timeBound$}{
							$yourop := slot$ \;
							\uIf(\tcp*[f]{a collision was done}) {$yourop \neq myop$} {
								\uIf (\tcp*[f]{$yourop$ collide with $myop$}) {$youop.parnter = myop$} {
									\switchPair $(myop, yourop)$ \;
									$slot.\CAS (yourop, default)$ \tcp*{release $slot$}
								}
								\KwRet $myop.result$ \;
							}
						}
						\tcp{time limit reached and no process collide with me}
						\uIf (\tcp*[f]{try to release $slot$}) {$slot.\CAS(myop, default)$}{
							%$myop.result := \timeout$ \;
							\KwRet \timeout\;
						} \uElse (\tcp*[f]{some process show up}) {
							$yourop := slot$ \;
							\uIf{$yourop.partner = myop$} {
								\switchPair $(myop, yourop)$ \tcp*{complete the collision}
								$slot.\CAS (yourop, default)$ \tcp*{release $slot$}
							}
							\KwRet $myop.result$ \;
						}
					}
					break \;
				}
				\uCase (\tcp*[f]{some process is waiting in $slot$}){\waiting} {
					$myop.partner := yourop$ \tcp*{attempt to replace $yourop$}
					$myop.state := \busy$ \;
					\uIf (\tcp*[f]{try to collide}){$slot.\CAS (yourop, myop)$ \label{exchange-busy-cas}} {
						\switchPair $(myop, yourop)$ \tcp*{complete the collision}
						$slot.\CAS (myop, default)$ \tcp*{release $slot$}
						\KwRet $myop.result$ \;
					}
					break \;
				}
				\uCase (\tcp*[f]{a collision in progress}) {\busy} {
					\switchPair $(yourop, yourop.parnter)$ \tcp*{help to complete the collision}
					$slot.\CAS (yourop, default)$ \tcp*{release $slot$}
					break \;
				}
			}	
		} 
	\end{procedure}

	\caption{Recoverable Elimination-Stack: \exchange\ routine.}
	\label{alg:eliminition-stack-exchange}
	\end{algorithm}

\end{figure*}




\begin{figure*}[!t]
	\removelatexerror
	
	\begin{algorithm}[H]
		
	\small
	\begin{flushleft}	
	\end{flushleft}
	
	\begin{procedure}[H]
		\caption{() void \switchPair (\exInfo\ $first$, \exInfo\ $second$)}
		
		\tcc{exchange the valus of the two operations}
		$first.result := second.value$ \;
		$second.result := first.value$ \;
		
	\end{procedure}
	
	\begin{procedure}[H]
		\caption{() T \visit\ (T $value$, int $range$, long $duration$)}
		
		\tcc{invoke \exchange\ on a random entery in the collision array}
		int $cell$ := randomNumber$(range)$ \;
		\KwRet \exchange$(exchanger[cell], value, duration)$
	\end{procedure}
	
	\begin{procedure}[H]
		\caption{() T \recoverExchange\ (\exInfo\ $*myop$)}
		
		%\exInfo\ $*myop := RD[pid]$ \tcp*{read your last operation record}
		\exInfo\ $*slot := myop.slot$ \tcp*{slot to recover}
		
		\uIf (\tcp*[f]{If operation response was persisted}) {$myop.result \neq \init$}
		{\KwRet $myop.result$}
		
		\uIf {$myop.state = \waiting$} {
			\tcc{crash while trying to exchange $defualt$, or waiting for a collision}
			$yourop := slot$ \;
			\uIf (\tcp*[f]{still waiting for a collision}) {$yourop = myop$} {
				\uIf (\tcp*[f]{try to release $slot$}){$\neg slot.\CAS(myop, default)$} {
					$yourop := slot$ \tcp*{some process show up; complete collision}
					\uIf{$yourop.partner = myop$} {
						\switchPair$(myop, yourop)$ \tcp*{complete the collision}
						$slot.\CAS (yourop, default)$ \tcp*{release $slot$}
					}
					%\KwRet $myop.result$ \;
					%\textbf{go to} \ref{ExRecover-successful} \;
				}	
			}
			\uElseIf (\tcp*[f]{$yourop$ collide with $myop$}) {$yourop.partner = myop$} {
				\switchPair$(myop, yourop)$ \tcp*{complete the collision}
				$slot.\CAS (yourop, default)$ \tcp*{release $slot$}
				%\KwRet $myop.result$ \;
			}
			%\uIf {$myop.result \neq \init$} {\KwRet $myop.result$ \;}
			%\KwRet \fail	
		}
		\uIf {$myop.state = \busy$} {
			\tcc{crash while trying to collide with $myop.partner$}
			$yourop := slot$ \;
			\uIf (\tcp*[f]{collide was successful and in progress}) {$yourop = myop$ \label{exchange-rec-busy-if}} {
				\switchPair$(myop, myop.partner)$ \tcp*{complete the collision}
				$slot.\CAS (myop, default)$ \tcp*{release $slot$}
				%\KwRet $myop.result$ \;
				%\textbf{go to} \ref{ExRecover-successful} \;
			}	
		}
		\KwRet $myop.result$
		
		\remove{
			returnCode: \tcp*{successful exchange} \label{ExRecover-successful}
			\uIf {exactly one of \{myop.value, myop.result\} is \NULL}
			{KwRet myop.result \tcp*[f]{Exchange of \push and \pop}}
			\uElse {\KwRet \fail \tcp*[f]{Exchange of two operations of the same type}}
		}
	\end{procedure}
	
	\caption{Recoverable Elimination-Stack: Elimination Array routines.}
	\label{alg:eliminition-stack-exchanger}
	\end{algorithm}

\end{figure*}



%%%%%%%%%%%%%%%%%%PUSH%%%%%%%%%%%%%%%%%%%%%
\begin{figure*}[!t]
	\removelatexerror
	
	\begin{algorithm}[H]
		
	\small
	\begin{flushleft}	
	\end{flushleft}
	
	\begin{procedure}[H]
		\caption{() boolean \trypush\ (Node $*nd$)}
		
		\tcc{attempt to perform \push\ to the central stack}
		Node $*oldtop := Top$\;
		$nd.next := oldtop$\;
		$RD[pid] := data$ \tcp*{update Info structure for this operation}
		\uIf (\tcp*[f]{try to declare $nd$ as the new $Head$}) {$Top.\CAS(oldtop, nd)$} {
			$nd.popby.\CAS(\init, \NULL)$ \tcp*{announce $nd$ is in the stack}
			$data.result$ := \True \tcp*{Persist response}
			\KwRet \True
		} 
		\KwRet \False
	\end{procedure}
	
	\begin{procedure}[H]
		\caption{() boolean \push\ (T $myitem$)}
		
		Node $*nd$ = \textbf{new} Node $(myitem, \init, \init)$ \label{push-entry} \;
		%$nd.popby := \init$ \;
		\csInfo\ $*data$ := \textbf{new} \csInfo\ ($nd, \init$) \;
		%\pushInfo\ $data$ := new \pushInfo\ ($nd$) \;
		$RD[pid]$ := $data$ \tcp*{Install Info structure for this operation}
		$\text{CP}_p$ := curPC() \tcp*{Set a check-point indicating \Info\ structure was installed} \label{push-set-checkpoint}
		\While{\True}{
			%$RD[pid] := data$ \tcp*{declare - trying to push node $nd$}
			\uIf (\tcp*[f]{if central stack \push\ is successful}){\trypush$(nd)$} {\KwRet \True}
			$range$ := CalculateRange() \tcp*{get parameters for collision array}
			$duration$ := CalculateDuration()\;
			$othervalue$ := \visit$(myitem, range, duration)$ \tcp*{try to collide}
			\uIf (\tcp*[f]{successfuly collide with \pop\ operation}) {$othervalue = \NULL$}{
				RecordSuccess ()\;
				\KwRet \True\;
			} \uElseIf (\tcp*[f]{failed to collide}) {$othervalue = \timeout$} {RecordFailure ()}
		}
	\end{procedure}
	
	
	\begin{procedure}[H]
		\caption{() boolean \recoverPush\ $()$}
		
		\info\ $data := RD[pid]$ \tcp*{read recovery data}
		T $result$ := \init \;
		
		\uIf {$\text{CP}_p$ $<$ \ref{push-set-checkpoint} \label{push-recover-read-CP}}
		{Proceed from line \ref{push-entry} \tcp*{Failed before installing info structure, re-execute}}
		
		\uIf (\tcp*[f]{crash while accessing collision array}) {$data$ of type \exInfo} {
			\uIf (\tcp*[f]{successful collision}) {\recoverExchange(data) = \NULL}
				{$result := \True$}
		}
		\uElse (\tcp*[f]{crash while accessing central stack}) {
			Node $*nd$ := $RD[pid].nd$ \;
			\uIf (\tcp*[f]{If operation response was persisted}) {$RD[pid].result \neq \init$}
			{$result := RD[pid].result$}
			\uElseIf (\tcp*[f]{$nd$ in the stack, or was announced as such}) {$\search(nd) \mid \mid  nd.popby \neq \init$} {
				$nd.popby.\CAS(\init, \NULL)$ \tcp*{announce $nd$ is in the stack}
				$result := \True$
			}
		}
		\uIf (\tcp*[f]{operation was completed}) {$result \neq \init$} {
			$RD[pid].result := result$ \tcc*{persist response and return}
			\KwRet $result$
		}
		\uElse (\tcp*[f]{operation was not completed})
		{Proceed from line \ref{push-entry} \tcp*[f]{re-execute}}

	\end{procedure}
	
	\begin{procedure}[H]
		\caption{() boolean \search\ (Node $*nd$)}
		
		\tcc{search for node $nd$ in the stack}
		
		Node $*iter := Top$ \;
		\While{$iter \neq \init$}{
			\uIf {$iter = nd$} {\KwRet \True}
			$iter := iter.next$	
		}
		\KwRet \False
	\end{procedure}
	
	\caption{Recoverable Elimination-Stack: \push\ routines.}
	\label{alg:eliminition-stack-push}
	\end{algorithm}

\end{figure*}




%%%%%%%%%%%%%%%%%%POP%%%%%%%%%%%%%%%%%%%%%%%%%
\begin{figure*}[!t]
	\removelatexerror
	
	\begin{algorithm}[H]
		
	\small
	\begin{flushleft}	
	\end{flushleft}
		
	\begin{procedure}[H]
		\caption{() T \trypop $()$}
		
		Node $*oldtop := Top$ \;
		Node $*newtop$ \;
		$data.nd$ := $oldtop$\;
		$RD[pid]$ := $data$ \tcp*{update Info structure for this operation}
		\uIf (\tcp*[f]{stack is empty}) {$oldtop = \init$} {
			$data.result$ := \emptyst \tcp*{Persist response}
			\KwRet \emptyst}
		$newtop := oldtop.next$ \;
		$oldtop.popby.\CAS(\init,\NULL)$ \tcp*{announce $oldtop$ is in the stack}
		\uIf (\tcp*[f]{try to pop $oldtop$ by changing $Top$ to $newtop$}) {$Top.\CAS(oldtop, newtop)$}{
			\uIf (\tcp*[f]{try to announce yourself as winner}) {$newtop.popby.\CAS(\NULL, pid)$} {
				$data.result$ := $oldtop.value$ \tcp*{Persist response}
				\KwRet $oldtop.value$}
		}
		\uElse {\KwRet \init}
	\end{procedure}
	
	\begin{procedure}[H]
		\caption{() T \pop\ $()$}
		
		Node $*result$ \label{pop-entry} \;
		\csInfo\ $*data$ := \textbf{new} \csInfo\ ($Top, \init$) \;
		$RD[pid]$ := $data$ \tcp*{Install Info structure for this operation}
		$\text{CP}_p$ := curPC() \tcp*{Set a check-point indicating \Info\ structure was installed} \label{pop-set-checkpoint}
%		\popInfo\ $data$ := \textbf{new} \popInfo\ ($Top$) \;
		
		\While{\True}{
		%	$RD[pid] := data$ \tcp*{declare - trying to perform \pop}
			$result$ := \trypop $()$ \tcp*{attempt to pop from central stack}
			%\uIf {$result = \emptyst$} {\KwRet \emptyst}
			\uIf (\tcp*[f]{if central stach \pop\ is successful}) {$result \neq \init$} {\KwRet $result$}
			$range$ := CalculateRange() \tcp*{get parameters for collision array}
			$duration$ := CalculateDuration()\;
			$othervalue$ := \visit $(\NULL, range, duration)$ \tcp*{try to collide}
			\uIf (\tcp*[f]{failed to collide}) {$othervalue = \timeout$}{RecordFailure ()}
			\uElseIf (\tcp*[f]{successfuly collide with \push\ operation}) {$othervalue \neq \NULL$} {
				RecordSuccess ()\;
				\KwRet $othervalue$
			}
		}
		
	\end{procedure}
	
	\begin{procedure}[H]
		\caption{() T \recoverPop $()$}
		
		\info\ $data := RD[pid]$ \tcp*{read recovery data}
		T $result$ := \init \;
		
		\uIf {$\text{CP}_p$ $<$ \ref{pop-set-checkpoint} \label{pop-recover-read-CP}}
		{Proceed from line \ref{pop-entry} \tcp*{Failed before installing info structure, re-execute}}
		
		\uIf (\tcp*[f]{crash while accessing collision array}) {$data$ of type \exInfo} {
			$temp$ := \recoverExchange(data) \;
			\uIf (\tcp*[f]{successful collision}) {$temp \neq \NULL\ \&\&\ temp \neq \init$}
				{$result := temp$}
		}
		\uElse (\tcp*[f]{crash while accessing central stack}) {
			Node $*nd$ := $RD[pid].nd$ \;
			\uIf (\tcp*[f]{If operation response was persisted}) {$RD[pid].result \neq \init$}
				{$result := RD[pid].result$}
			\uElseIf (\tcp*[f]{pop from an empty stack}) {$nd = \init$} {$result$ := \emptyst}
			%\uIf {$nd.popby = pid$} {\KwRet $nd.value$}
			\uElseIf (\tcp*[f]{$nd$ was removed from the stack}) {$\neg \search(nd)$} {
				$nd.popby.\CAS(\NULL, pid)$ \tcp*{try to announce yourself as the winner}
				\uIf (\tcp*[f]{if you are the winner}) {$nd.popby = pid$} {$result$ := $nd.value$}
			}
		}
		\uIf (\tcp*[f]{operation was completed}) {$result \neq \init$} {
			$RD[pid].result := result$ \tcc*{persist response and return}
			\KwRet $result$
		}
		\uElse (\tcp*[f]{operation was not completed})
			{Proceed from line \ref{pop-entry} \tcp*[f]{re-execute}}
	\end{procedure}

	\caption{Recoverable Elimination-Stack: \pop\ routines.}
	\label{alg:eliminition-stack-pop}
	\end{algorithm}
	
\end{figure*}


\end{document}
