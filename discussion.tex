\section{Discussion}

Our approach relies on key properties of the the implementation,
regardless of the memory primitives it applies.
In particular, our approach avoids the ABA problem by relying on the
fact that each operation attempt uses a unique info record,
and therefore, a CAS can be replaced with LL/SC (with minor modifications).
In such case, the recoverable version is still correct,
as the new implementation delivers the same guarantees as the original.

The simple but efficient approach we demonstrate for BST can be used
for other implementation in which a process first declare its operation
by marking a node and adding an info record.
Moreover, in some cases it is very natural to transform the implementation
to such a form, that is, to perform operations by writing info records
that contains the entire data required for the operation
(as demonstrate for the elimination stack).
This way, we can both be able to recover,
as well as be able to design an helping mechanism.
Note that we require each attempt to create a new info record,
to avoid the ABA problem.
