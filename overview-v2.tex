\section{Overview of Our Approach}

Our methodology separately considers the operations provided by
a linearizable implementation $A$ of a data structure $Q$,
according to their properties.
In this initial description, we assume each operation becomes
visible and is linearized through at most a single \CAS;
this simplifies recovery by avoiding scenario of an
operation changing the data structure but is yet to complete
and be linearized.
% we didn't discuss helping yet!
% while some other helping process may complete and linearize it.
(Later in this section, discuss how to remove this assumption,
in the context a binary search tree.)

Consider an implementation of an operation $Op$ that does not change
the data structure, as is often the case with a search operation.
In this case, the operation remains intact, and the recovery
function simply restarts it from scratch.
(Later, in the context of the linked list, we explain how this
approach is extended to the case where stopping $Op$ at any point
and restarting it does not violate linearizability.)
% say this later, not here:
%i.e., it is indistinguishable to all other processes from a run in which it is executed only once.

For operations that do change the data structure, assume first that
there is a way to uniquely identify each instance of $Op$.
In this case, $Op$ can be made recoverable by adding an indication
once it is visible.
The recovery function then checks for this indication:
if found, the operation has completed, and its response is returned;
otherwise, the operation is restarted.
For example, assume process $p$ tries to add a new unique node $nd$
to the data structure using \CAS\
(other processes may add different nodes with the same data).
Once $nd$ is added, it can be found in the data structure,
indicating that the operation has completed.
However, this indication is lost when $nd$ is removed.
This problem can be overcome by flagging a special field in $nd$ before
removing it.
This way, we have two indicators,
one is set once $nd$ is added and the other when $nd$ is deleted.

A further complication is posed when the implementation employs a helping
mechanism, namely, several processes attempt to perform an operation,
%%% what do mean in the next sentence? each attempt is not uniquely identified?
which is not uniquely identified.
When $p$ crashes during an execution of $Op$ and recovers, it might be
able to check whether $Op$ is complete, but it is unable to
know whether it is the one to perform it, or some other process.
As a result, $p$ may not be able to recover the right response value.
% but why it is a problem? if the operation is uniquely identified, then who cares who performed it?
This is resolved by adding an \emph{owner} field (for each operation type),
used for agreeing which process performed the operation;
after a process $p$ completes an operation, \CAS\ its id to \emph{owner},
to declare itself as the winner.
If $p$ crashes during an execution of $Op$ and recovers,
$p$ first checks whether the operation is visible, and if so,
$p$ tries to set $owner$ to $p$ with \CAS\ and then responds according to
the id stored in $owner$.
%For example, trying to delete a node $nd$ from the data-structure is such an operation.
%An indication for it being visible is that $nd$ is no longer part of the data-structure.
%However, several process may try and delete $nd$ concurrently.
(This is demonstrated in the \emph{delete} of our Linked-List implementation.)

In all cases, it is necessary to persist the response value before returning it.
In some cases, $p$ may also need to store some recovery data;
for example, the node to be added or removed.
This information is stored in a designate memory location, for each process;
after the process stores this information and before the process starts
performing the operation,
a checkpoint is set to signify that the data is relevant for the current operation.

We next further explain our approach with a relatively simple example,
of a linked-list set, and then explain some extensions with a
\emph{binary search tree}.
Later in the paper (Section~\ref{section:elimination-stack}),
we present a comprehensive example
of taking a sophisticated concurrent data structure,
an \emph{elimination stack}, and making it recoverable.

\subsection*{Linked List}

Algorithms~\ref{alg:linked-list} and~\ref{alg:linked-list2} show
Harris' classic lock-free \emph{linked list}
set~\cite{DBLP:conf/wdag/Harris01}.\footnote{
    The \delete\ pseudo-code is slightly optimized so that if the key is found and
    is later logically deleted, then \delete\ returns \True\ if the logical deletion
    was performed by the current process, and \False\ otherwise
    (lines \ref{delete-while-loop}-\ref{delete:logical-delete}, \ref{delete:return-res}).}
It supports \find, \insertlst\ and \delete\;
the latter two use the \search\ helper function in order to find the node with
the smallest key greater than or equal to the input key (denoted \emph{curr})
and its predecessor in the list (denoted \emph{pred}).
\delete\ first logically deletes a node by marking it
as deleted and then physically removes it from the list.
\ohad{[add here description how this is done, or reference to details in section?!][add a bit more detail]}
%While traversing the list, \search\ attempts to physically remove any marked node
%it encounters from the list.



\begin{figure}[!t]
	\removelatexerror

\begin{algorithm}[H]

	\footnotesize

	\begin{flushleft}	
	\textbf{Type} Node \{MarkableNodeRef $*next$, int \emph{key}\}\\
	\textbf{Shared variables:} Node $*head$ \\	
	\end{flushleft}	

%%%%%%%%%%%%%%%%%%%%%% Linked List - Search %%%%%%%%%%%%%%%%%%%%%%%%

	\begin{procedure}[H]
  		\caption{() $\langle$Node $*$, Node $*$$\rangle$ \search\ (T $key$)}

		Node $*pred, *curr, *succ$ \;
		retry: \While{\True} { \label{find-outer-loop}
			$pred := head$ \;
			$curr := pred.next$ \label{search-read-cur}\;	
			\While{\True}{
				$succ := curr.next$ \;
				\tcp{if $curr$ was logically deleted}
				\uIf {$marked(succ)$} {
					\uIf (\tcp*[f]{Help physical delete}) {$pred.next.\CAS\ $ ($<0,curr>$,$<0,succ>$) = \False}{
					\textbf{go to} retry \tcp*{Help failed}
					}
					$curr := succ$ \tcp*{Help succeeded}
				} \uElse {
					\uIf (\tcp*[f]{First unmarked node with key $\geq key$}) {$curr.key \geq key$} {
					\KwRet $\langle pred,curr \rangle$
					}
					$pred := curr$ \tcp*{Advance $pred$}
					$curr := succ$ \tcp*{Advance $curr$}
				}
			}
		}
	\end{procedure}

%%%%%%%%%%%%%%%%%%%%%% Linked List - Find %%%%%%%%%%%%%%%%%%%%%%%%

	\begin{procedure}[H]
  		\caption{() boolean \find\ (T $key$)}

		Node $*curr := head$ \label{linked-list-find-start} \;
		\tcc{Search for first node with key $\geq$ $key$}
		\While {$curr.key < key$}{
			$curr = curr.next$ \label{find-read-cur}\;	
		}
		\KwRet ($curr.key = key\ \&\&\ \neg marked(curr.next)$) \label{linked-list-find-end}
	\end{procedure}

\caption{Linked list: \search\ and \find} \label{alg:linked-list}
\end{algorithm}

\end{figure}




\begin{figure}[!t]
\removelatexerror
	
\begin{algorithm}[H]
		
	\footnotesize
	\begin{flushleft}	
	\end{flushleft}

%%%%%%%%%%%%%%%%%%%%%% Linked List - Insert %%%%%%%%%%%%%%%%%%%%%%%%

	\begin{procedure}[H]
  		\caption{() boolean \insertlst\ (T $key$)}
  		%\textbf{Private variables:}
  		Node $*pred, *curr$ \label{ll:insert-entry} \;
		Node $newnd$ := \textbf{new} Node ($key$) \;
		\While{\True}{ \label{ll:insert-while}
			\tcp{Search the right location to insert}
			$\langle pred, curr \rangle := \search(key)$ \label{ll:insert-search} \;
			\uIf (\tcp*[f]{$key$ in the list}) {$curr.key = key$ \label{ll:insert-if-in-list}} {
				\KwRet \False \label{ll:insert-return-false}
			} \uElse {
				$newnd.next$ := $<0,curr>$ \;
				\tcp{Try to add $newnd$}
				\uIf {$pred.next.\CAS$ ($<0,curr>$, $<0,newnd>$)} {\KwRet \True} \label{ll:insert-CAS}
			}		
		}
	\end{procedure}


%%%%%%%%%%%%%%%%%%%%%% Linked List - Delete %%%%%%%%%%%%%%%%%%%%%%%%

	\begin{procedure}[H]
		\caption{() boolean \delete\ (T $key$)}

		Node $*pred, *curr, *succ$ $\quad$  boolean $res := \False$ \label{ll:delete-entry} \;
		$\langle pred, curr \rangle := \search (key)$ \tcp*{Search for $key$} \label{ll:delete-search}
		\uIf {$curr.key \neq key$ \label{ll:delete-if-in-list}} {
			\KwRet \False \tcp*[f]{$key$ not in the list} \label{ll:delete-return-false}
		}
		\uElse {
			\While (\tcp*[f]{Repeatedly attempt logical delete}) {$\lnot marked(curr.next)$  \label{ll:delete-while}}{
				$succ := curr.next$ \label{ll:delete-set-succ-to-next}\;
				$res := curr.next.\CAS$ ($<0,succ>$, $<1,succ>$) \label{ll:delete-logical-delete}\;
			}
			\tcp{Physical deletion attempt}
			$succ := curr.next$ \label{ll:delete-read-succ-after-CAS}\;
			$pred.next.\CAS$ ($<0,curr>$, $<0,succ>$) \label{ll:delete-physical-delete}
			\KwRet $res$ \label{ll:delete-return-res}
			}

	\end{procedure}

\caption{Linked list: \insertlst\ and \delete} \label{alg:linked-list2}
\end{algorithm}

\end{figure}



To find a key $k$, a process simply looks for an unmarked node with key $k$
(lines~\ref{linked-list-find-start}-\ref{linked-list-find-end}).
To insert a key $k$, process $p$ calls \search\ in order to find the position
in the list where $k$ should be added (line~\ref{ll:insert-search})
If $k$ is already in the list, \insertlst\ returns \False\
(lines \ref{ll:insert-if-in-list}-\ref{ll:insert-return-false}),
otherwise it tries to set \emph{pred}.\emph{next} to point to a new node
containing $k$ using \CAS\ (line~\ref{ll:insert-CAS});
this fails if \emph{pred} has been logically deleted in the interim.
To delete a key $k$, $p$ calls \search\, returning \False\ if $k$ not in the set
(lines \ref{ll:delete-if-in-list}-\ref{ll:delete-return-false}).
Otherwise, $p$ repeatedly tries to logically delete it by marking the \emph{next}
field of its node using \CAS, until the node is marked
(lines \ref{ll:delete-while}-\ref{ll:delete-logical-delete}).
After the node is marked, $p$  tries to physically remove it
(lines \ref{ll:delete-set-succ-to-next}-\ref{ll:delete-physical-delete}).

The operations are linearized as follows:
A \emph{Find} is linearized with the last read of \emph{curr}
(line~\ref{find-read-cur}), \find\ returns \True\ if node \emph{curr} is unmarked
and \False\ otherwise.
An \emph{Insert} is linearized either when the \search\ called by \insertlst\
reads (line~\ref{search-read-cur}) an unmarked node with key $k$,
when \insertlst\ returns \False,
or with a successful \CAS\ inserting $k$ to the list (line~\ref{ll:insert-CAS}),
when \insertlst\ returns \True.
A \emph{delete} is linearized either when the \search\ instance it calls
reads in line~\ref{search-read-cur} an unmarked list node with a key greater
than $k$ (in which case \delete\ returns \False),
upon a successful logical deletion by the current operation
(in line~\ref{ll:delete-logical-delete}, \delete\ returns \True),
or upon a logical deletion of the node by another concurrently executing
process (in line~\ref{delete:logical-delete}, \delete\ returns \False).

Thus, successful \insertlst\ and \delete\ are linearized when
they change the data-structure in a manner visible to other processes,
e.g., after the successful \CAS\ in line~\ref{ll:insert-CAS} of \insertlst,
or after a successful logical deletion in line~\ref{ll:delete-logical-delete} of \delete.
The instructions in lines~\ref{ll:insert-CAS} and~\ref{ll:delete-logical-delete}
are the {\em realization} \CAS\ for \insertlst\ and \delete, respectively.

Consider now the situation when some process $p$ recovers from a crash-failure.
By the way linearization points are assigned to operations and by the definition
of realization \CAS\ for \insertlst\ and \delete,
the algorithm satisfies \textit{strict linearizability} \cite{Aguilera2003StrictLA}:
a crashed operation can be either removed from the history or linearized
between its invocation and crash.
If $p$ failed during an operation $Op$,
either the realization \CAS\ for $Op$ has already been executed
in which case the effect of $Op$ becomes visible
and $Op$ is linearized in that \CAS,
or $p$ crashes at some earlier point of $Op$'s exeution
in which case none of the steps $Op$ performed before crashing
will ever become visible and $Op$ is not linearized.

However, the response of the failed operation may be lost.
For example,
in a scenario in which process $p$ performs $\delete(k)$
and crashes immediately after applying a \CAS\ to mark the node containing
key $k$, $p$ has no way of knowing when it recovers whether its failed operation
had any effect on the set. Specifically, $p$ cannot know whether its \delete\
has executed its realization \CAS, and if so what response it should return.

We now explain our approach to make this algorithm recoverable.
(See detailed description and code in Section~\ref{section:linked-list}.)
Our approach leaves \find{} as is, since it is a read-only operation.
\search{} is not read-only, but it can be stopped and restarted at any point,
without violating the linearizability of the resulted history,
and hence need not be changed.

\insertlst{} is uniquely identify by the node $nd$ it tries to add,
as different instances creates different nodes.
Moreover,
once the operation is visible and linearized in a successful realization \CAS,
$nd$ can be found in the list, indicating the operation is visible.
Finally, $nd$ is marked before it is physically removing from the list,
indicating that $nd$ was in the list.
This means there are two indications, one of which holding if
a realization \CAS\ was successfully performed before the crash,
and none of which holds, otherwise.

Our approach makes these operations recoverable,
by testing these indications upon recovery.
For \delete, note that $nd$ logical deletion (in a successful realization \CAS)
is indicated in $nd$ itself, which is marked, and will stay so forever.
Therefore, if $p$ crashes during a delete of $nd$, then $p$ can conclude,
upon recovery, whether $nd$ was deleted by checking if it is marked.

However, $p$ can not tell whether it is the process to delete $nd$.
We add a \emph{deleter} field to each node, so that once the node is deleted,
the first process to \CAS{} its id to \emph{deleter} is the one to perform the \delete.

\remove{
It is easy to make an operation crash-persistent by using crash-persistent CAS,
since an operation is linearized at its first successful CAS that
does not help other operations.
However, the operation is not necessarily detectable:
if there is a crash right after this CAS, then, upon recovery,
the crashed process needs to know whether the CAS was successful or not,
in order to conclude the right response value.
Moreover, two or more processes can try to perform the same operation,
for example, helping to delete a node; upon recovery the process not only
has to know whether the node was deleted,
but also to determine it has deleted the node or some process.

We address these problems by exploiting two properties of the implementation.

First, data is added to the linked-list nodes, ensuring they are unique,
even if they hold the same key.
A node \emph{nd} inserted to the list with a CAS will stay accessible until it is deleted
and the recovery function of the Insert operation, can check with the insert was
successful by testing the former two conditions. % did I miss something? which two conditions?

For \delete, we notice a logically deleted node (whose marked bit is set)
remains so forever, and therefore if process $p$ crashes while deleting \emph{nd},
it can check the marked bit upon recovery to detect whether \emph{nd} is logically deleted.
However, this information is insufficient, since several processes may try
to delete \emph{nd} concurrently, and exactly one should succeed.
For that, each node is equipped with an extra \emph{deleter} field,
initially $\bot$, used by the deleting processes to agree which process
succeeded to delete \emph{nd}.
The first process that writes its id to \emph{deleter} (with a CAS)
a \emph{nd} is logically deleted, is the one to succeeds in deleting the node,
while all others fail and return false.
Notice that crashed operations are also trying to write to \emph{deleter}
and that this filed is written to exactly once.
A recovering process can check if it succeeded in the delete by checking
if \emph{deleter} holds its id.

As we discuss later, the same technique can be applied to any implementation
with the following features:
An operation that takes effect using a single CAS,
and needs to guarantee exactly one process is the one to perform it,
can use a \emph{new} field (as in delete) to determine which process executed it;
this relies on the ability to determine, upon recover, whether the CAS was successful.
Additionally, an \emph{insert} to a data structure in which a new,
unique object is used whenever a new key is added (in a node),
can be recovered by looking for the new object,
or for an indication it has been removed from the data structure.
One such implementation is the \emph{queue}~\cite{MichaelS-PODC1996}.
% but we do not discuss this implementation but two others, more complicated

}


\subsection*{Binary Search Tree}

In the lock-free \emph{binary search tree} (BSF)
of~\cite{DBLP:conf/podc/EllenFRB10},
processes help each other to carry their operations.
The implementation, described in detail in Section~\ref{section:BSF}.
each operation has an associated info record,
initialized in a single CAS by the calling process,
and using by helping processes to track their progress.

When a process $p$ needs to modify a node $nd$ it marks $nd$ with
a pointer to its info record; this marking remains until $p$'s operation
completes (either by $p$ or by a helping process).
This implies that upon recovery after a crash,
$p$ can check to see if the node is still marked with its own operation,
and if so, try to complete the operation, starting from right after the marking CAS;
otherwise, $p$ restarts the operation.

As is often the case when helping is used,
several processes may try to perform the same operation (concurrently),
leading to redundant efforts that yield only a single completed operation.
Even more problematic in the context of the crash-recovery model,
and because helping is anonymous,
repeated attempts of $p$ to perform an operation (due to crashes)
may be indistinguishable from an execution where other processes are helping $p$
complete its operation.

The above recovery scheme may still allow a scenario in which $p$ crashes
right after completing its operation and before unmarking the node,
causing $p$ to restart the operation upon recovery and applying it twice.
To avoid this scenario, a process updates a \emph{done} field in
the info record, which signifies the operation is done,
before unmarking the node in the cleaning phase.
Thus, if the node is not marked with the operation upon recovery,
then the new field allows the process to conclude whether
the operation has been completed or failed.
If a process crashes after updating the done field and before cleaning,
then upon recovery, unless another process performed the cleaning,
it will observe that the node is marked and complete the cleaning phase.

%%% this may belong in the elimination stack section
\subsection*{Elimination Stack}

After presenting these specific data structures, linked list and BST,
we can explain the two principles underlying our general approach,
in the context of a more complex data structure---an
\emph{elimination stack}~\cite{DBLP:journals/jpdc/HendlerSY10}.

An elimination stack combines of two components:
a central stack and an elimination array.
%%% Explain the implementation more carefully.
The central stack is implemented in a way that resembles Harris'
linked list, described above:
to push or pop the process tries to atomically swing the head pointer using CAS.
Therefore, the same approach taken in the Linked-List can be used for this case also.
The Pop operation will use a new $popby$ field, similar to the deleter field,
in order to determine which process is the one to pop the node.

However, unlike the linked list,
where a node is first marked before removing it from the list,
in the stack a Pop operation is done using a single CAS without marking.
Hence, if a process $p$ crashes right after successfully pushing a new node
and the node is later removed from the [[stack??]], $p$ can not tell
between the resulting situation and one in which the Push has failed.

This problem is resolved by having a process mark a node [[as part of the list??]]
before trying to pop it; marking does not mean that the node has been removed,
as the Pop operation may fail.
%%% the following seems too technical and an optimization.
For efficiency, the $popby$ field is used also for this marking.
A process first writes NULL to it, and in case of a successful Pop try to write its id.
This is all done using CAS, so to avoid overwriting.

In the elimination array each entry holds an \emph{exchanger} object that matches
two processes---one doing a push and another doing a pop---to exchange their values.
We change the implementation such that a process first creates an info
record containing its values, and then processes exchange info records
instead of values:
The process doing a push writes its info record by a CAS on an exchanger,
while the process doing a pop takes this it by replacing it, with a CAS,
with its own info record.

To allow recovery, the info record includes s a state, identifying whether
$p$ is the first or second process to write to an exchanger,
i.e., whether it was performing a push or a pop;
in the latter case, the record also contains a reference to the info record $p$
is trying to collide with.
If $p$ crashes and the info record indicates it is part of an ongoing
exchange, then $p$ tries to complete the exchange;
otherwise, either its exchange attempt failed,
or that it succeeded and completed by some other process.
The latter case is indicated in the info record, which also contains the
response value.

The info records also support a simple and efficient helping mechanism.
After a process writes its info record second to the exchanger,
any other process complete the exchange by reading this record and
updating both records with the right response values.
That is, other processes do not wait for the two colliding processes
and complete the exchange on their behalf.
Then, they can reset the exchanger, making it ready for reuse.
