\section{Introduction (based on our PODC)}

Shared-memory multiprocessors are asynchronous in nature.
Asynchrony is related to reliability,
since algorithms that provide nonblocking progress properties
(e.g., lock-freedom and wait-freedom \cite{herlihy91waitfree})
in an asynchronous environment with reliable processes
continue to provide the same progress properties in the presence
of \emph{crash failures}.
This happens because a process that crashes
permanently during the execution of the algorithm is indistinguishable
to the other processes from one that is merely very slow.
Owing to its simplicity and intimate relationship with asynchrony,
the crash-failure model is almost ubiquitous in the treatment of
concurrent algorithms.

The attention to the crash-failure model has so far mostly neglected
the \emph{crash-recovery} model, in which a failed process may be
resurrected after it crashes.
Recent developments foreshadow the emergence of new systems,
in which byte-addressable \emph{non-volatile main memory} (\emph{NVRAM}),
combining the performance benefits of conventional main memory
with the durability of secondary storage,
co-exists with (or eventually even replaces) traditional volatile memory.
Consequently, there is increased interest in \emph{recoverable concurrent
objects} (also called
\emph{persistent}~\cite{CoburnCAGGJW-Asplos2011,ChenQ-VLDB2015}
or \emph{durable}~\cite{VenkataramanTRC-FAST2011}):
% YF: persistent is misleading
% HA: But it's a term used in the literature; added citations.
objects that are made robust to crash-failures by allowing their operations
to recover from such failures.

This paper consider an abstract individual-process crash-recovery model
for non-volatile memory~\cite{AttiyaBH-PODC2018}.
Processes communicate via non-volatile shared-memory variables.
Each process also has local variables stored in volatile processor registers.
[[YF: What about program counter? HA: in this paper, it can be volatile.]]
At any point, a process may incur a crash-failure,
causing all its local variables to be reset to arbitrary values.
Operation response values are returned via volatile processor registers,
which may become inaccessible to the calling process if it fails
just before persisting the response value.
A data structure has an associated \emph{recovery function} that is 
responsible for restoring it upon the recovery from a crash-failure.
The recovery function completes the current outstanding operation on 
the data structure, if there was any, returning either its \emph{response}
or a \emph{fail} indication, if it was unsuccessful. 
Both responses are consistent with the resulting state of the data structure,
to which the operation was applied (in the former case) 
or not (in the latter case).

We present a principled approach to deriving recoverable versions
and describe in in detail in two widely-used concurrent data structures:
a linked list~\cite{} and elimination stack~\cite{}.
%%%%%%%%%%% NEED MORE %%%%%%%%%%%%%%%%%%

%%%%%%%%%%% START OF RELATED WORK %%%%%%%%%%%%%%%%
Several correctness conditions for the crash-recovery model were
defined in recent years (see, e.g., \cite{Aguilera2003StrictLA,GolabR16,DBLP:conf/opodis/BerryhillGT15,DBLP:conf/icdcs/GuerraouiL04,DBLP:conf/wdag/IzraelevitzMS16}). The goal of these conditions is to maintain the state of concurrent objects
consistent in the face of crash failures.
