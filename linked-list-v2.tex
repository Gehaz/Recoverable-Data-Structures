\section{Linked-List Based Set}
\label{section-linked-list}

In this section, we present a recoverable version of the linked-list set algorithm of Harris \cite{DBLP:conf/wdag/Harris01}.\footnote{Some implementation details 
follow the algorithm's presentation in~\cite{DBLP:books/daglib/0020056}.}
 
The linked-list set supports the \find, \insertlst\ and \delete\ operations. 
The algorithm maintains a linked list of nodes sorted in increasing order of keys. 
\y{The list} always \y{contains} a \emph{head} and \emph{tail} sentinel nodes, 
containing keys $-\infty, \infty$, respectively.
The pseudo-code for the algorithm is presented 
in Algorithms \ref{alg:linked-list-part1}-\ref{alg:linked-list-part2}.
Pseudo-code in blue font was added for recoverability. 
the \emph{next} field of \y{each node 
%is a MarkableNodeRef type, consisting of a reference 
consists of a reference} to the next node and a \emph{marked} bit that is set when the node is logically deleted. 
Both components can be manipulated atomically, either together or individually, 
using a singe-word \CAS\ operation\footnote{In the Java implementation of the algorithm 
presented in \cite{DBLP:books/daglib/0020056}, \emph{next} fields are represented 
by AtomicMarkableReference \y{objects. In the implementation of an AtomicMarkableReference object}
the marked bit occupies the least significant bit of the reference.}. 
The \emph{marked} predicate can be applied 
to \y{the next field of a node to determine whether or not the node is marked}.

%\subsection{Detailed Description of the Algorithm}

We now describe an extension of the Harris algorithm 
that allows a process recovering from a crash-failure 
to determine whether \ohad{its failed operation completed, and return the correct response.}
%\y{the realization \CAS\ of its failed operation occured before the failure and return the correct response.}
\y{The recovery function of the \find\ operation simply re-invokes \find\ 
(hence its pseudo-code is not shown).}
To support the recovery of \insertlst\ and \delete\ operations, 
a (persistent) shared-memory array \emph{RD} was added, 
where \emph{RD}[\emph{p}] stores recovery data for process $p$. 
More specifically, variable \emph{RD}[\emph{p}] contains a pointer 
to an \Info\ structure storing recovery data for the process' current recoverable operation. 
Each \Info\ structure stores two fields - a reference \emph{nd} to a node 
and a \emph{result} field used for persisting the response value for the operation before returing.

We now describe the additions to the \insertlst\ operation that were introduced 
for supporting recoverability. First, \insertlst\ installs a fresh \Info\ structure \y{into $RD$}
in line \ref{insert-install-IS}, whose \emph{nd} field points 
to a newly allocated node structure (the \emph{deleter} field is only 
used by \delete\ operations and is described later). 
It then updates $p$'s check-point variable (in line \ref{insert-set-checkpoint}) 
by invoking the curPC macro, returning the current value of the program counter 
\y{(for simplicity, we assume that it is \ref{insert-set-checkpoint}, in this case)}. 
\y{By executing this line, $p$ {\em persistently reports}
that the info structure for its current operation has been installed.}
Finally, once \insertlst\ determines its response, 
it persists it just before returning (in lines \ref{insert-persist-false-response} 
and \ref{insert-persist-true-response}).

We now describe the \insertrecover\ function. Let $W$ denote the instance 
of \insertlst\ from whose failure  \insertrecover\ attempts to recover. 
\insertrecover\ starts by reading $p$'s check-point variable 
in line \ref{insert-recover-read-CP} in order to check 
whether $p$'s current info structure was installed by $W$. 
If the failure occurred before $W$ \y{persistently reported that} 
it installed its info structure 
(in line \ref{insert-set-checkpoint}), then $W$ is re-executed from scratch. 
\y{Otherwise, the following actions take place.}
If a response was already written to $W$'s \Info\ structure, 
then \insertrecover\ simply returns this response 
(lines \ref{insert-recover-response-persisted}-\ref{insert-recover-return-persisted-response}). 
We are left with the case that a response was not yet written by $W$ to its \Info\ structure, 
\y{so either $W$ did not execute a successful realization \CAS\ (in line \ref{insert-CAS}) 
or its realization \CAS\ succeeded 
%(hence $W$ was linearized) 
but $W$ failed before writing 
the response in line \ref{insert-persist-true-response}. }
In order to determine which of these two scenarios occurred, 
\insertrecover\ searches the list for \emph{key} (in line \ref{insert-recover-search}). 
%As we prove, 
\y{If either} the key is found in the list inside the node allocated 
by $W$ or if that node was marked for deletion (line \ref{insert-if-was-linearized}), 
\y{then $W$ had executed its realization \CAS\ before the failure occured
(i.e. $W$ succeeded in inserting its node in the linked list),} 
and so the recovery function persists the response 
and returns \True (lines \ref{insert-linearized-persist-true}-\ref{insert-linearized-return-true}). 
\y{Notice that indeed if the node has been successfully linked in the list,  
the only way for the \search\ not to find it is if it has, in the meantime, been
deleted. However, in that case, the node will have, first, been marked for deletion, and
therefore the condition of the if statement of line~\ref{insert-if-was-linearized} will be evaluated to true.}
Otherwise, recovery proceeds from line \ref{insert-entry}
in order to re-attempt insertion.\footnote{\textcolor{red}{BY ``recovery proceeds from line X'' we do not mean 
to say that there is a jump to line X, since in order to do so execution context 
before line X must be saved and then restored. Instead, we mean that the pseudo-code performed 
by the recovery function from this point on is the same as that of the recoverable \y{operation}
starting from line X. One efficient way of implementing this is to have 
both the \y{recoverable operation} and the recovery function invoke a parameterless macro call 
that embeds this pseudo-code during compilation pre-processing.}}

Next, we describe how recoverability \y{is ensured for} \delete\ operations. 
Similarly to the recoverable \insertlst\ operation, 
a recoverable \delete\ first installs a fresh \Info\ structure \y{into $RD$}
in line \ref{delete-entry}. It then updates $p$'s check-point variable 
(in line \ref{delete-set-checkpoint}) \y{to {\em persistently report} 
that its info structure has been installed.} 
If \emph{key} is not found 
(line \ref{search-in-delete}), the response \y{(which is in this case \False) is persisted 
and then it is returned} (lines \ref{delete-write-false}-\ref{delete-return-false}). 
If \emph{key} \emph{is} found (in node \emph{curr}), 
a reference to \emph{curr} is persisted to the \emph{nd} field of $p$'s \Info\ structure.
Following this, $p$ proceeds as \y{in the original algorithm} by repeatedly trying 
\y{to mark \emph{curr} using \CAS\ in lines \ref{delete-while-loop}-\ref{delete:logical-delete}
(i.e. it repeatedly executes \CAS\ until it logically deletes the node).} 
Once it is marked, $p$  tries to physically remove \emph{curr} 
in lines \ref{delete-read-succ-after-CAS}-\ref{delete-physical-delete}. 

The key technical difficulty for supporting recoverability of \delete\ operations 
is the following. If $p$ fails immediately before or after the \CAS\ 
of line \ref{delete:logical-delete}, recovers and then finds that \emph{curr}
was logically deleted, how can it know whether it was the (single) process 
that succeeded \y{in deleting \emph{curr}}? Our solution to this problem is 
to ``attribute'' a node's deletion to a single process using a new node-field 
called \emph{deleter}, initialized to $\bot$. After $p$ finds that \emph{curr} 
is logically deleted (marked) in line \ref{delete-while-loop}, 
\emph{regardless of whether it was marked by $p$ or by another process}, 
$p$ tries to establish itself as the deleter of \emph{curr} by atomically 
changing \emph{curr.deleter} from $\bot$ to its ID using \CAS\ 
(line \ref{delete-CAS-deleter}).
%This way, if $p$ crashes during a \delete\ operation, it can use the \emph{deleter} field in order to determine its response value.
Finally, $p$ persists and returns the result of this \CAS\ 
\y{as the response of the \delete\ operation}
in lines \ref{delete-write-after-CAS}-\ref{delete:return-res}.

We now describe the \deleterecover\ function. Let $D$ denote the instance of \delete\ 
from whose failure  \deleterecover\ attempts to recover. 
\deleterecover\ starts by reading $p$'s check-point variable 
in line \ref{delete-recover-read-CP}. If the failure occurred 
before $D$ \y{persistently reported that} it installed its info structure (in line \ref{delete-set-checkpoint}), 
then $D$ is re-executed from scratch. 
\y{Otherwise, the following actions take place.}
If a response was already written 
to $D$'s \Info\ structure, then \delete\ simply returns this response 
(lines \ref{delete-recover-response-persisted}-\ref{delete-recover-return-persisted-response}).
Otherwise, if the \emph{nd} field of $p$'s \Info\ structure was previously set 
and node $nd$ is logically deleted (line \ref{delete-recover-if-should-perform-deleter-CAS}), 
$p$ attempts to establish itself as the deleter of $nd$ using \CAS,
\ohad{and then persists and returns the result according to the id written in \emph{curr.deleter}} 
%and then persists and returns the result of that \CAS\ 
(lines \ref{delete-recover-deleter-CAS}-\ref{delete-recover-return-res})
\y{as the response of its \delete\ operation.} 
Finally, if the condition of line \ref{delete-recover-if-should-perform-deleter-CAS} 
does not hold, $p$ re-attempts the deletion (line \ref{delete-recover-reattempt-deletion}).



\subsection{Correctness Argument}

In the following, we give a high-level argument for the correctness of the algorithm.
We say that a node is in the linked-list if it is reachable
by following $next$ pointers starting from $head$.
We say that a node is in the implemented set if it is in the linked-list and is not marked.
%For simplicity of presentation, we sometimes use $curr$ (or $newnd$) 
%to refer to the node pointed to by it.} 

The proof relies on the following observations.
As long as a node $nd$ is in the linked-list there is exactly one node in the list pointing to it.
In addition, $nd$ can be marked exactly once, and it stays so forever, as any \CAS\ is executed with an unmarked node as its first argument.
For the same reason, $nd$ can be physically deleted exactly once, since no node in the list points to it once it is deleted, and we never add a marked node back to the list.

\find\ operation is implemented in a read-only manner, and thus it can be re-executed without effecting any other concurrent operation.
In addition, \search\ routine, even though not read-only, simply traverses the list 
while trying to physically delete any marked node it encounters. As any marked node can be physically deleted exactly once, re-executing \search\ can not cause a deletion of an unmarked node, or a deletion of the same node more then once. This implies re-executing \search\ can only help physical deletion of more nodes, and does not effect the list in any other way.

\insertlst\ and \delete\ first install info structure and set a check-point. Clearly, a crash before the check-point implies the operation did not effect the list or any other operation, and in such case the \recover\ function simply re-execute the operation. This argument holds for any number of crashes, as long as the check-point is yet to be set.
%We are left with the case of a crash after the check-point. We prove it for each operation.

Assume process $p$ performs an $\insertlst(key)$ operation. If $p$ does not crash, then it repeatedly search for the right location for the new node $newnd$, 
and tries to insert it.
In case $p$ updates $RD[p].result$ to \False\ in line \ref{insert-persist-false-response}, then the preceding \search\ in line \ref{insert-search} finds a node $curr$ in the set with data $key$ (when \search\ read $curr$ it is not marked). Therefore, there is a point along the \insertlst\ operation where the set contains $key$, and the operation is linearized at this point. If $p$ crash after updating $result$ then eventually, in order to complete \insertrecover\ $p$ must read $RD[p].result$ and return \False.

In case $p$ performs its realization \CAS\ in line \ref{insert-CAS}, then $newnd$ is in the set. The only way to physically remove it from the list is by first marking it. Therefore, after the realization \CAS, either $newnd$ is in the list, or it is marked, and one of these conditions must hold. As a result, in any crash after the realization \CAS\ the \insertrecover\ function returns \True, either in line \ref{insert-recover-return-persisted-response} (because $result$ has already been updated), or in line \ref{insert-if-was-linearized} which evaluates to \True. In particular, each \insertlst\ operation can perform at most a single realization \CAS, as any crash after it results a response of \True, without performing any more \CAS\ instances.

We are left with the case where $p$ crash before performing its realization \CAS\ or updating $result$. In such case, $newnd$ was not added to the list yet, and in particular no process can mark it. As a result, \insertlst\ did not effect any other operation, and indeed \insertrecover\ re-executes it.

Assume process $p$ performs $\delete(key)$ operation. If $p$ updates $result$ to \False\ in line \ref{delete-write-false}, then the preceding \search\ found two adjustments nodes $pred$ and $curr$ in the list, where $pred.key < key < curr.key$. Since we keep the list sorter in an increasing manner, it follows the list does not contains $key$ at the point when $pred$ points to $curr$. In particular, there is point along the interval of \delete\ where $key$ is not in the set, and the operation is linearized at this point. Any crash after line \ref{delete-write-false} results the \deleterecover\ function must read $result$ and return \False.

Assume now $p$ does not write in line \ref{delete-write-false}. A node $curr$ is written to $RD[p].nd$ in line \ref{delete-update-nd} only if \search\ observes $curr$ is in the list and not marked. Clearly, a crash before updating $RD[p].nd$ implies the operation did not mark any node, nor effected any other operation, and \deleterecover\ simply re-execute \delete. On the other hand, once $p$ updates $RD[p].nd$, it keeps trying to mark $nd$.
If $p$ crash and recovers, and finds $nd$ is not marked (line \ref{delete-recover-if-should-perform-deleter-CAS}), then in particular $p$'s operation did not mark $nd$, nor effected the list or any other operation. In such case, \deleterecover\ re-executes \delete. This argument holds for any number of crash and recover, as long as $p$ observe $nd$ is not marked.

If $p$ observes $nd$ is marked, either in the \delete\ or in \deleterecover, we conclude the marking was done after the read of $nd$ in \search. Moreover, once $nd$ is marked, in order to complete the \delete\ operation, $p$ must performs \CAS, trying to write its name to $nd.deleter$, either in \delete, or in \deleterecover. Since $deleter$ is initialised to \init, only the first such \CAS\ succeeds.
Let $q$ be the first process to perform such \CAS, if exists. Then, it set $nd.deleter$ to $q$, and this does not change. Once $deleter$ is set, if $p=q$, then it can only return \True, even in case it crash, as line \ref{delete-recover-read-deleter} always evaluates to \True. Otherwise, $p \neq q$, and by similar argument it can only return \False.
As a result, any process trying to delete $nd$ and observe it is marked (at any point), must have the marking step in its \delete\ operation interval. In addition, exactly one such process, denoted $q$, returns \True, while any other returns \False. We linearize the \delete\ operation of $q$ at the time of the marking, and any other \delete\ returning \False\ is linearized right after it (in an arbitrary order).







\remove{
We will argue each $\insertlst(key)$ can add at most a single node to the list. Moreover, it returns \False\ only if there is a point in time along its interval where there is a node in the set with data $key$.
Similarly, $\delete(key)$ can mark at most a single node with data $key$. In case it returns \False\ there is a point in time along its interval where there is no node in the set with data $key$. In addition, we show no two process can return \True\ when trying to delete the same node.



We will argue that


The proof relies on the following arguments. For a node $nd$ that was added to the list, at each point of the execution, if $nd$ has not been physically deleted then it is reachable. Moreover, there is a single node in the list pointing to $nd$. 

\find\ operation is implemented in a read-only manner, and thus does not effect any other concurrent operation. As a result, re-executing the operation in case of a crash does not effect any other process. 

The proof is based on the following argument. Each node can be added (marked) exactly once, and this is attribute to a single \insertlst\ (\delete) operation. Moreover, if \insertlst\ (\delete) with input $key$ returns \False, there is a point along the interval execution where there is (there is no) node in the set with $key$.

Each \insertlst(key) operation can add at most a single node, and each \delete(key) operation can logically delete at most a single node. Moreover, in case no node has been added (logically deleted), there is a point it time along the operation interval where key is in the set (not in the set). In addition, once a node is marked it is considered as not in the list, and no future \find\ operation 

We will show that each \find(key)\ operation either adds a new node exactly once and return \True, or returns \False, in which case there is a point along its interval where $key$ is the the set. 

\find\ operation is implemented in a read-only manner, and thus does not effect any other concurrent operation. As a result, re-executing the operation in case of a crash does not 
satisfies NRL, and the linearization point is set in the interval of the last complete execution of the operation.
The \search\ procedure simply \y{traverses} the list, 
while trying to physically delete marked nodes.
Once $curr.next$ is marked, \y{just one} process can perform the physical \y{deletion}. 
This follows from the fact that, at any point in time, there is a single node 
in the list which points to $curr$. Once curr is physically deleted,
no node in the list points to curr, and thus any \CAS\ operation with curr 
as the first parameter will fail. This observation relays on the fact 
that any new allocated node has a different address than $curr$. 
As a result, repeating the attempt to physically delete a node effect the list at most once.

\here{YOULA: It is unclear what we mean when we say "the list consistency".
We usually define the consistency of a data structure based on linearizability,
but linearizability is defined for a history. So, we probably need to fix a configuration,
define the history up to that configuration, define what we mean when we say "data structure"
for this configuration, and then define what we mean when we say "consistent data
strucutre in that configuration. Even then, it is not clear what we mean when we say 
that the data structure is consistent during the execution of a search 
because that the way I can think for this to make sense is that it is consistent 
in all configurations that the search is active.  Based on what I wrote above, 
I removed the reference to the consistency of the list.}

Assume a process $p$ performs an $\insertlst(key)$ operation.
First, $p$ writes to $RD[p]$ and set a check-point, reporting it is about to perform an \insertlst. A crash before setting the check-point implies the operation did not effect the linked-list yet, and \insertrecover\ re-execute the operation.
Once $p$ set the check-point it repeatedly search for the right location for the new node, 
and tries to insert it by performing a \CAS\, unless $key$ is already in the list (As in the original algorithm).
As long as $p$ does not performs its realization \CAS\ $newnd$ is not in the list, and therefore the operation is not visible to any other process. Hence, repeating the operation in such case does not violate linearizability. In addition, once $p$ updates $RD[p].result$ with a response we conclude the operation already had a linearization point (as in the original algorithm), and indeed, in this case \insertrecover\ returns the response stored in $RD[p].result$.

We are left with 
If $p$ does not crash, then, as in the original algorithm, 
it repeatedly tries to find the right location for the new node, 
and insert it by performing a \CAS\.
%changing $pred.next$ to point to $newnd$. 
In addition, a crash after updating $RD[p].result$ 
is after the operation has performed its realization \CAS,
%had its linearization point, 
and \insertrecover\ procedure will return 
the right response. \y{It remains to consider a crash that occurs before the} update to $RD[p].result$. 
There are two scenarios to consider.

Assume $p$ \y{crashes} without performing a successful \CAS\ in line \ref{insert-CAS}. 
Notice that $p$ is the only process to have a reference to $newnd$, 
and it is yet to update any node with this reference, 
and thus no linked-list node points to $newnd$. 
As a result, the operation did not \y{affect} any other process, 
nor it will be in the future. 
Hence, considering the operation as not having a linearization point 
does not violate the list consistency. Indeed, since no node points to $newnd$, 
upon recovery $p$ will see that $newnd$ is not in the list and also not marked, 
and thus will return FAIL. Notice this argument holds whether key is already in the tree, 
or not, as the operation in both cases did not affect the system.
\here{YOULA: I do not agree with the argument above. I think that most of the 
statements in this paragraph are wrong. What does it mean "it will return FAIL"?
According to the pseudocode, this is never the case (all operations return either true or false).
If the key is not in the list, the operation is repeated, no? So, eventually it will add 
the new key in the list and it will affect other operations.
Also, notice that in either case, if the operation is completed, it is linearized. 
So, I suggest to avoid talking about linearization in the discussion above.}

Assume now $p$ \y{crashes after successfully performing its realization} \CAS\ in line \ref{insert-CAS}.
\y{Right after the \CAS, $newnd$ is in the implemented set, as $pred.next$ points to it
and $newnd$ is initialized to be unmarked.}
Also notice \y{that the successful execution of the realization \CAS\ for $p$'s \insertlst\
does not cause the deletion of} any other node: $pred.next$ pointed to $curr$, 
and after the \CAS\ it points to $newnd$ which points to $curr$. 
\y{
We argue that if a node is in the implemented set at some point in time 
and it is not at some later point in time, then at all times after
the point that it is not in the set anymore, the node is marked.
The above argument relies on the fact that a marked node can not be unmarked, 
and that an Insert and Delete can not mistakenly remove nodes from the list. 
Therefore, if $newnode$ is no longer in the list, it must be marked.
As a result, when $p$ executes the Recover procedure, either it will see $newnd$ in the list, 
or it will find it marked. In either case, the condition of the if statement of line~\ref{???}
\here{YOULA: Ohad can you please fill in the reference here?} 
will be evaluated to true and the response of the operation will be true.}
%that it is no longer part of the list, and it must be some other process deleted it, 
%and hence $newnd.next$ is marked. In any case, $p$ will return true as required.

Assume a process $p$ performs a $delete(key)$ operation. First, $p$ writes to RD[p], 
\y{reporting that} it is about to perform a Delete. As before, a crash after writing to RD[p].result 
will return the right response. Also, a \y{crash} before updating any of RD[p].result 
or RD[p].nd implies $p$ is yet to try and mark any node, 
and thus the operation did not affect the system so far, 
nor it will be in the future. Therefore, we can consider the operation 
as not having a linearization point (even in case key is not the list), 
and indeed, the Recover procedure returns FAIL in such case.
\here{YOULA: I do not agree with the argument above for the same reasons 
as those I mentioned previously for \insertlst. }

Assume thus $p$ \y{wrote to $RD[p].nd$ before it crashes}. 
It follows that $p$ completed \y{\search\ } 
and found a node $curr$ storing key. \y{\search\ } guarantees  that there is a point 
in time \y{(during its execution)} where $curr$ is in the list and $curr.next$ is not marked. 
\y{If $p$ recovers and observes} that $curr$ is unmarked, then in returns FAIL. 
\here{YOULA: I do not understand. It never returns FAIL. Has this been left from some previous version?}
Since a marked node can not be unmarked, as there is no \CAS\ changing a marked node, 
it follows that $p$ did not marked $curr$. \here{Therefore, the operation did not affect any process, 
nor it will be, and we consider it as having no linearization point. YOULA: I do not follow:
depending on whether $p$ will manage to record itself as the deleter, it will 
return true or false (in either case it will be linearized).}
Otherwise, \y{the recovery function observes} $curr$ as marked, and we can conclude 
the marking point of $curr$ is along the delete operation. We now prove 
we can linearize the operation, according to its response.
\here{YOULA: I do not follow the last sentence.}

Let $q$ be the process to mark $curr$. Since once $curr.next$ is marked \y{it will never change}, 
the reference of $curr.next$ is fixed to $succ$ \y{(i.e. to the value that variable $succ$ in the
\delete\ instance of $q$ had at the point that $q$ performed the marking of $curr$). }
This also implies $q$ is unique and well defined, and any future \CAS\ on $curr.next$ will fail. 
As a result, any process leaving the while loop in line \ref{delete-while-loop} reads the same value 
in line \ref{delete-read-succ-after-CAS}, which is this $succ$. The attempt to physically delete $curr$ 
in line \ref{delete-physical-delete} will succeed only if $pred.next$ points to $curr$, 
and as we said, \y{$curr.next$} points to $succ$, and any other attempt will fail. 
Thus, if this attempt succeeds, it deletes only $curr$, and can not delete additional nodes.

In line \ref{delete-CAS-deleter} process $p$ tries to write its id to $curr.deleter$. 
\y{As it is initialized to $\bot$}, only the first process to perform this \CAS\ will succeed. 
Also, \y{every} $p$ must go through line \ref{delete-CAS-deleter} in order to complete its operation, 
\here{as the recovery function \y{redirects} the process to this line. YOULA: Not accurate: 
the recovery function contains a similar CAS in line 106. It "redirects" the process (if I understand
correctly what redirects mean) only if line 110 is executed.}
\y{Therefore, if there exists a process 
that has completed its delete operation on $curr$, 
there must be a successful \CAS\ to $curr.deleter$.} Let $q'$ be the first process to perform this \CAS. 
As proved above, $q'$ tries to delete $curr$, and the point in time 
where $curr$ is marked must be contained in its operation interval. 
Moreover, $q'$ is the only process to write to $curr.deleter$, 
and the first one to do so, thus $q'$ is the only process to obtain true when testing 
($curr.deleter = q'$) in line \ref{delete-write-after-CAS} (and thus to also return true), 
while any other process will \y{return} false. We linearize the operation of $q'$ at the point 
of the marking, and any other attempt to delete $curr$ is linearized after it (in an arbitrary order).

A corollary of the analysis is that processes trying to delete the same node $curr$, 
all keep trying to mark $curr$ (and therefore it is like if they help each other). 
However, the process that successfully performs the marking 
is not necessarily the one to return true. Also, in the original algorithm, 
if a process fails to mark a node, it starts the delete operation from the beginning. 
In our implementation, the process can keep trying to mark the node without the need 
to perform a \y{\search\ } again after each failed \CAS. We guarantee that once $curr$ is marked, 
exactly one process will return true, while the rest can consider $curr$ as being deleted 
(in the course of their delete execution), and thus there is a point along their execution 
at which $key$ is not in the tree, and they can return false.


\here{YOULA: I think it would be nice if we add 1-2 paragraphs to explain why nothing goes wrong 
in case a failure occurs while $p$ executes the recovery function of the \insertlst\ or
 the \delete.}


}



