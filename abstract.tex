\begin{abstract}
Recent developments foreshadow the emergence of new systems,
in which byte-addressable \emph{non-volatile main memory} (\emph{NVRAM}),
combining the performance benefits of conventional main memory
with the durability of secondary storage, co-exists with
(or eventually even replaces) traditional volatile memory.
Consequently, there is increased interest in \emph{recoverable}
concurrent objects: objects that are made robust to crash-failures
by allowing their operations to recover from such failures.
This paper presents a principled approach to deriving recoverable versions
of several widely-used concurrent data structures.
\y{Specifically, the approach can be applied in a wide range of
well-known concurrent data structure implementations to make them
recoverable, including stacks, queues, linked lists, trees and elimination stacks. }
%in particular, a linked list, a concurrent BST, and an elimination stack.
\end{abstract} 
