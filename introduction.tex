\section{Introduction}

Recent years has seen the emergence of systems in which
byte-addressable \emph{non-volatile main memory} (\emph{NVRAM}),
combining the performance benefits of conventional main memory
with the durability of secondary storage,
co-exists with (or eventually even replaces) traditional volatile memory.
This has lead to increased interest in the \emph{crash-recovery} model,
in which a failed process may be resurrected after it crashes.
Of particular interest is the design of \emph{recoverable concurrent
objects} (also called
\emph{persistent}~\cite{CoburnCAGGJW-Asplos2011,ChenQ-VLDB2015}
or \emph{durable}~\cite{VenkataramanTRC-FAST2011}):
% YF: persistent is misleading
% HA: But it's a term used in the literature; added citations.
objects that are made robust to crash-failures by allowing their operations
to recover after such failures.

It is challenging to design data structures that persist in the presence of
crashes and recoveries, and several concurrent implementations were proposed. %%% list them here
While many of these exploit specific aspects of an object to
optimize the implementation, it is important to develop generic
approaches for deriving recoverable implementations from their
non-recoverable counterparts.
Such an approach should preserve the structure and efficiency of
the implementation, as much as possible,
while avoiding the need to design new algorithms.

This paper presents a principled approach to deriving recoverable
objects and describe it in detail through three widely-used
concurrent data structures:
a linked list~\cite{DBLP:conf/wdag/Harris01},
binary search tree~\cite{}
and elimination stack~\cite{DBLP:journals/jpdc/HendlerSY10}.

Our results are presented in the context of an abstract individual-process
crash-recovery model for non-volatile memory~\cite{AttiyaBH-PODC2018},
in which processes communicate via non-volatile shared-memory variables.
%%% I have taken out the issue of volatile / non-volatile local variables
%%% and the exclusion of the program counter.
%%% It should be mentioned later in the introduction.
%Each process also has local variables stored in volatile processor registers
%[[YF: What about program counter? HA: in this paper, it can be volatile.]]
At any point, a process may incur a crash-failure,
causing all its local variables, [[[except for its program counter,???]]]
to be reset to arbitrary values.
Operation response values are returned via volatile processor registers,
which may become inaccessible to the calling process if it fails
just before persisting the response value.
[[[Each data structure operation???]]] has an associated \emph{recovery
function} that is
responsible for restoring it upon the recovery from a crash-failure.
The recovery function completes the current outstanding operation on
the data structure, if there was any, returning either its \emph{response}
or a \emph{fail} indication, if it was unsuccessful.
Both responses are consistent with the resulting state of the data structure,
to which the operation was applied (in the former case)
or not (in the latter case).

\subsection*{Related Work}

Several different correctness conditions and implementations have been
proposed for recoverable data
structures~\cite{CoburnCAGGJW-Asplos2011,ChenQ-VLDB2015,VenkataramanTRC-FAST2011,%
Aguilera2003StrictLA,GolabR16,DBLP:conf/opodis/BerryhillGT15,DBLP:conf/icdcs/GuerraouiL04,DBLP:conf/wdag/IzraelevitzMS16}.
However, this work concentrates on maintaining the consistency of
the concurrent object in the face of crash failures,
and does not consider \emph{detectability}, that is, the ability to
conclude upon recovery whether the crashed operation took effect.

There is a universal implementation~\cite{DBLP:journals/pacmpl/CohenFL17}
which is both durable [[undefined?]] and lock-free,
it uses only read, write and CAS.
This implementation applies at most one \emph{persistence[[?]] fence}
(flushing the contents of the memory) per operation, which is optimal,
but it keeps the entire history of the object in a designated shared queue;
it also keeps a per-process persistent log, such that these logs together
keep the entire history, and different logs may have a big overlap.
Furthermore, to determine the response of an operation,
it is necessary to read the entire history up to its linearization point.
This construction can easily be made detectable since an operation was
linearized if and only if it appears in the shared queue
(representing the object's history).
Specifically, after the system completes its recovery routine,
in the recovery from an $Update(op)$, 
the process can determine the response value from the shared queue;
this assumes that each $op$ is uniquely identified.
This work considers a \emph{system-wide} crash model, in which
all processes crash together and a single recovery function is
executed upon recovery,
in order to consistently reconstruct the queue data structure.
It ensures \emph{durable linearizability}~\cite{VenkataramanTRC-FAST2011}:
after a full system crash, the state of the object must reflect a consistent operation
sub-history that includes all operations completed by the time of the crash,
i.e., crashed operations may get lost.

\emph{Nesting-safe recoverable linearizability} (\emph{NRL})~\cite{AttiyaBH-PODC2018}
is a novel crash-recovery model together with a correctness condition.
It associates each recoverable operation with a recovery function, invoked 
after a crash in the operation to help a process to complete the operation, 
as well as restore the response value if needed.
They give recoverable implementations for read, write and CAS.
As suggested by its name, NRL support \emph{nesting},
so taking an algorithm that uses only read, write and CAS,
and replacing each primitive with its recoverable version yields an NRL implementation.
Some minor changes are still needed in order to use this transformation,
but they hold for all the implementations presented in that paper.
However, this transformation is quite costly, in terms of both time and space.

Indeed, implementations using only read, write and CAS can be made
recoverable and detectable~\cite{DBLP:journals/corr/abs-1806-04780},
by partitioning the code into \emph{capsules},
each containing a single CAS followed by some number of reads,
and replacing each CAS with its recoverable version.
This allows to recover from a crash inside the capsule.
\emph{Normalized} implementations~\cite{TimnatP-PPoPP2014} can be
further optimized so that an operation contains only two capsules.
However, not all implementation are given in a normalized form, and
converting an implementation into a normalized form may be costly by itself.
This general transformation has several drawbacks:
For example, replacing a CAS with its recoverable variant requires each
CAS to have distinct arguments, ensured by adding unique sequence numbers,
which are stored in the CAS location.
This means that CAS is applied to words with unbounded length,
even if the original implementation applied CAS over a finite domain.
Furthermore, although two capsules are used for each normalized operation,
these two capsules are repeatedly executed in attempt to complete the operation.
Thus, the implementation is lock-free.

In many cases we can avoid it,
without the extra capsule complexity for a failed attempt.
Moreover, we would like the transformation to change as little
as necessary from the original code,
even at the price of having a costly recover function.
Assuming crashes are rare,
this may yield a more efficient implementation in practice. 